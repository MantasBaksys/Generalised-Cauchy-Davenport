\documentclass[11pt,a4paper]{article}
\usepackage[T1]{fontenc}
\usepackage{amssymb}
\usepackage{isabelle,isabellesym}
\usepackage[english]{babel}  % for guillemots

% this should be the last package used
\usepackage{pdfsetup}

% urls in roman style, theory text in math-similar italics
\urlstyle{rm}
\isabellestyle{it}

\begin{document}

\title{A Generalization of the Cauchy--Davenport theorem}
\author{Mantas Bak\v{s}ys \\
University of Cambridge\\
\texttt{mb2412@cam.ac.uk}}

\maketitle
                                                             
\begin{abstract}
The Cauchy--Davenport theorem is a fundamental result in additive combinatorics.
It was originally independently discovered by Cauchy \cite{cauchy1812recherches} and Davenport \cite{davenport} and has been formalized in the AFP entry \cite{Kneser_Cauchy_Davenport-AFP} as a corollary of Kneser's theorem.
More recently, many generalizations of this theorem have been found. In this entry, we formalise a generalization due to DeVos \cite{DeVos2016OnAG}, which proves the theorem in a non-abelian setting.
\end{abstract}
\newpage
\tableofcontents

\newpage

% include generated text of all theories
%
\begin{isabellebody}%
\setisabellecontext{Generalized{\isacharunderscore}{\kern0pt}Cauchy{\isacharunderscore}{\kern0pt}Davenport{\isacharunderscore}{\kern0pt}preliminaries}%
%
\isadelimdocument
%
\endisadelimdocument
%
\isatagdocument
%
\isamarkupsection{Generalized Cauchy--Davenport Theorem: preliminaries%
}
\isamarkuptrue%
%
\endisatagdocument
{\isafolddocument}%
%
\isadelimdocument
%
\endisadelimdocument
%
\isadelimtheory
%
\endisadelimtheory
%
\isatagtheory
\isacommand{theory}\isamarkupfalse%
\ Generalized{\isacharunderscore}{\kern0pt}Cauchy{\isacharunderscore}{\kern0pt}Davenport{\isacharunderscore}{\kern0pt}preliminaries\isanewline
\ \ \isakeyword{imports}\ \isanewline
\ Complex{\isacharunderscore}{\kern0pt}Main\isanewline
\ {\isachardoublequoteopen}Jacobson{\isacharunderscore}{\kern0pt}Basic{\isacharunderscore}{\kern0pt}Algebra{\isachardot}{\kern0pt}Group{\isacharunderscore}{\kern0pt}Theory{\isachardoublequoteclose}\isanewline
\isanewline
\isakeyword{begin}%
\endisatagtheory
{\isafoldtheory}%
%
\isadelimtheory
%
\endisadelimtheory
%
\isadelimdocument
%
\endisadelimdocument
%
\isatagdocument
%
\isamarkupsubsection{Well-ordering lemmas%
}
\isamarkuptrue%
%
\endisatagdocument
{\isafolddocument}%
%
\isadelimdocument
%
\endisadelimdocument
\isacommand{lemma}\isamarkupfalse%
\ wf{\isacharunderscore}{\kern0pt}prod{\isacharunderscore}{\kern0pt}lex{\isacharunderscore}{\kern0pt}fibers{\isacharunderscore}{\kern0pt}inter{\isacharcolon}{\kern0pt}\ \isanewline
\ \ \isakeyword{fixes}\ r\ {\isacharcolon}{\kern0pt}{\isacharcolon}{\kern0pt}\ {\isachardoublequoteopen}{\isacharparenleft}{\kern0pt}{\isacharprime}{\kern0pt}a\ {\isasymtimes}\ {\isacharprime}{\kern0pt}a{\isacharparenright}{\kern0pt}\ set{\isachardoublequoteclose}\ \isakeyword{and}\ s\ {\isacharcolon}{\kern0pt}{\isacharcolon}{\kern0pt}\ {\isachardoublequoteopen}{\isacharparenleft}{\kern0pt}{\isacharprime}{\kern0pt}b\ {\isasymtimes}\ {\isacharprime}{\kern0pt}b{\isacharparenright}{\kern0pt}\ set{\isachardoublequoteclose}\ \isakeyword{and}\ f\ {\isacharcolon}{\kern0pt}{\isacharcolon}{\kern0pt}\ {\isachardoublequoteopen}{\isacharprime}{\kern0pt}c\ {\isasymRightarrow}\ {\isacharprime}{\kern0pt}a{\isachardoublequoteclose}\ \isakeyword{and}\ g\ {\isacharcolon}{\kern0pt}{\isacharcolon}{\kern0pt}\ {\isachardoublequoteopen}{\isacharprime}{\kern0pt}c\ {\isasymRightarrow}\ {\isacharprime}{\kern0pt}b{\isachardoublequoteclose}\ \isakeyword{and}\isanewline
\ \ t\ {\isacharcolon}{\kern0pt}{\isacharcolon}{\kern0pt}\ {\isachardoublequoteopen}{\isacharparenleft}{\kern0pt}{\isacharprime}{\kern0pt}c\ {\isasymtimes}\ {\isacharprime}{\kern0pt}c{\isacharparenright}{\kern0pt}\ set{\isachardoublequoteclose}\isanewline
\ \ \isakeyword{assumes}\ h{\isadigit{1}}{\isacharcolon}{\kern0pt}\ {\isachardoublequoteopen}wf\ {\isacharparenleft}{\kern0pt}{\isacharparenleft}{\kern0pt}inv{\isacharunderscore}{\kern0pt}image\ r\ f{\isacharparenright}{\kern0pt}\ {\isasyminter}\ t{\isacharparenright}{\kern0pt}{\isachardoublequoteclose}\ \isakeyword{and}\isanewline
\ \ h{\isadigit{2}}{\isacharcolon}{\kern0pt}\ {\isachardoublequoteopen}{\isasymAnd}\ a{\isachardot}{\kern0pt}\ a\ {\isasymin}\ range\ f\ {\isasymLongrightarrow}\ wf\ {\isacharparenleft}{\kern0pt}{\isacharparenleft}{\kern0pt}{\isacharbraceleft}{\kern0pt}x{\isachardot}{\kern0pt}\ f\ x\ {\isacharequal}{\kern0pt}\ a{\isacharbraceright}{\kern0pt}\ {\isasymtimes}\ {\isacharbraceleft}{\kern0pt}x{\isachardot}{\kern0pt}\ f\ x\ {\isacharequal}{\kern0pt}\ a{\isacharbraceright}{\kern0pt}\ {\isasyminter}\ {\isacharparenleft}{\kern0pt}inv{\isacharunderscore}{\kern0pt}image\ s\ g{\isacharparenright}{\kern0pt}{\isacharparenright}{\kern0pt}\ {\isasyminter}\ t{\isacharparenright}{\kern0pt}{\isachardoublequoteclose}\ \isakeyword{and}\isanewline
\ \ h{\isadigit{3}}{\isacharcolon}{\kern0pt}\ {\isachardoublequoteopen}trans\ t{\isachardoublequoteclose}\isanewline
\ \ \isakeyword{shows}\ {\isachardoublequoteopen}wf\ {\isacharparenleft}{\kern0pt}{\isacharparenleft}{\kern0pt}inv{\isacharunderscore}{\kern0pt}image\ {\isacharparenleft}{\kern0pt}r\ {\isacharless}{\kern0pt}{\isacharasterisk}{\kern0pt}lex{\isacharasterisk}{\kern0pt}{\isachargreater}{\kern0pt}\ s{\isacharparenright}{\kern0pt}\ {\isacharparenleft}{\kern0pt}{\isasymlambda}\ c{\isachardot}{\kern0pt}\ {\isacharparenleft}{\kern0pt}f\ c{\isacharcomma}{\kern0pt}\ g\ c{\isacharparenright}{\kern0pt}{\isacharparenright}{\kern0pt}{\isacharparenright}{\kern0pt}\ {\isasyminter}\ t{\isacharparenright}{\kern0pt}{\isachardoublequoteclose}\isanewline
%
\isadelimproof
%
\endisadelimproof
%
\isatagproof
\isacommand{proof}\isamarkupfalse%
{\isacharminus}{\kern0pt}\isanewline
\ \ \isacommand{have}\isamarkupfalse%
\ h{\isadigit{4}}{\isacharcolon}{\kern0pt}\ {\isachardoublequoteopen}{\isacharparenleft}{\kern0pt}{\isasymUnion}\ a\ {\isasymin}\ range\ f{\isachardot}{\kern0pt}\ {\isacharparenleft}{\kern0pt}{\isacharbraceleft}{\kern0pt}x{\isachardot}{\kern0pt}\ f\ x\ {\isacharequal}{\kern0pt}\ a{\isacharbraceright}{\kern0pt}\ {\isasymtimes}\ {\isacharbraceleft}{\kern0pt}x{\isachardot}{\kern0pt}\ f\ x\ {\isacharequal}{\kern0pt}\ a{\isacharbraceright}{\kern0pt}\ {\isasyminter}\ {\isacharparenleft}{\kern0pt}inv{\isacharunderscore}{\kern0pt}image\ s\ g{\isacharparenright}{\kern0pt}{\isacharparenright}{\kern0pt}\ {\isasyminter}\ t{\isacharparenright}{\kern0pt}\ {\isacharequal}{\kern0pt}\ \isanewline
\ \ \ \ {\isacharparenleft}{\kern0pt}{\isasymUnion}\ a\ {\isasymin}\ range\ f{\isachardot}{\kern0pt}\ {\isacharparenleft}{\kern0pt}{\isacharbraceleft}{\kern0pt}x{\isachardot}{\kern0pt}\ f\ x\ {\isacharequal}{\kern0pt}\ a{\isacharbraceright}{\kern0pt}\ {\isasymtimes}\ {\isacharbraceleft}{\kern0pt}x{\isachardot}{\kern0pt}\ f\ x\ {\isacharequal}{\kern0pt}\ a{\isacharbraceright}{\kern0pt}\ {\isasyminter}\ {\isacharparenleft}{\kern0pt}inv{\isacharunderscore}{\kern0pt}image\ s\ g{\isacharparenright}{\kern0pt}{\isacharparenright}{\kern0pt}{\isacharparenright}{\kern0pt}\ {\isasyminter}\ t{\isachardoublequoteclose}\ \isacommand{by}\isamarkupfalse%
\ blast\ \isanewline
\ \ \isacommand{have}\isamarkupfalse%
\ {\isachardoublequoteopen}{\isacharparenleft}{\kern0pt}inv{\isacharunderscore}{\kern0pt}image\ {\isacharparenleft}{\kern0pt}r\ {\isacharless}{\kern0pt}{\isacharasterisk}{\kern0pt}lex{\isacharasterisk}{\kern0pt}{\isachargreater}{\kern0pt}\ s{\isacharparenright}{\kern0pt}\ {\isacharparenleft}{\kern0pt}{\isasymlambda}\ c{\isachardot}{\kern0pt}\ {\isacharparenleft}{\kern0pt}f\ c{\isacharcomma}{\kern0pt}\ g\ c{\isacharparenright}{\kern0pt}{\isacharparenright}{\kern0pt}{\isacharparenright}{\kern0pt}\ {\isasyminter}\ t\ {\isacharequal}{\kern0pt}\ {\isacharparenleft}{\kern0pt}inv{\isacharunderscore}{\kern0pt}image\ r\ f\ {\isasyminter}\ t{\isacharparenright}{\kern0pt}\ {\isasymunion}\isanewline
\ \ \ \ {\isacharparenleft}{\kern0pt}{\isasymUnion}\ a\ {\isasymin}\ range\ f{\isachardot}{\kern0pt}\ {\isacharbraceleft}{\kern0pt}x{\isachardot}{\kern0pt}\ f\ x\ {\isacharequal}{\kern0pt}\ a{\isacharbraceright}{\kern0pt}\ {\isasymtimes}\ {\isacharbraceleft}{\kern0pt}x{\isachardot}{\kern0pt}\ f\ x\ {\isacharequal}{\kern0pt}\ a{\isacharbraceright}{\kern0pt}\ {\isasyminter}\ {\isacharparenleft}{\kern0pt}inv{\isacharunderscore}{\kern0pt}image\ s\ g{\isacharparenright}{\kern0pt}\ {\isasyminter}\ t{\isacharparenright}{\kern0pt}{\isachardoublequoteclose}\isanewline
\ \ \isacommand{proof}\isamarkupfalse%
\isanewline
\ \ \ \ \isacommand{show}\isamarkupfalse%
\ {\isachardoublequoteopen}inv{\isacharunderscore}{\kern0pt}image\ {\isacharparenleft}{\kern0pt}r\ {\isacharless}{\kern0pt}{\isacharasterisk}{\kern0pt}lex{\isacharasterisk}{\kern0pt}{\isachargreater}{\kern0pt}\ s{\isacharparenright}{\kern0pt}\ {\isacharparenleft}{\kern0pt}{\isasymlambda}c{\isachardot}{\kern0pt}\ {\isacharparenleft}{\kern0pt}f\ c{\isacharcomma}{\kern0pt}\ g\ c{\isacharparenright}{\kern0pt}{\isacharparenright}{\kern0pt}\ {\isasyminter}\ t\isanewline
\ \ \ \ {\isasymsubseteq}\ inv{\isacharunderscore}{\kern0pt}image\ r\ f\ {\isasyminter}\ t\ {\isasymunion}\ {\isacharparenleft}{\kern0pt}{\isasymUnion}a{\isasymin}range\ f{\isachardot}{\kern0pt}\ {\isacharbraceleft}{\kern0pt}x{\isachardot}{\kern0pt}\ f\ x\ {\isacharequal}{\kern0pt}\ a{\isacharbraceright}{\kern0pt}\ {\isasymtimes}\ {\isacharbraceleft}{\kern0pt}x{\isachardot}{\kern0pt}\ f\ x\ {\isacharequal}{\kern0pt}\ a{\isacharbraceright}{\kern0pt}\ {\isasyminter}\ inv{\isacharunderscore}{\kern0pt}image\ s\ g\ {\isasyminter}\ t{\isacharparenright}{\kern0pt}{\isachardoublequoteclose}\isanewline
\ \ \ \ \isacommand{proof}\isamarkupfalse%
\isanewline
\ \ \ \ \ \ \isacommand{fix}\isamarkupfalse%
\ y\ \isacommand{assume}\isamarkupfalse%
\ hy{\isacharcolon}{\kern0pt}\ {\isachardoublequoteopen}y\ {\isasymin}\ inv{\isacharunderscore}{\kern0pt}image\ {\isacharparenleft}{\kern0pt}r\ {\isacharless}{\kern0pt}{\isacharasterisk}{\kern0pt}lex{\isacharasterisk}{\kern0pt}{\isachargreater}{\kern0pt}\ s{\isacharparenright}{\kern0pt}\ {\isacharparenleft}{\kern0pt}{\isasymlambda}c{\isachardot}{\kern0pt}\ {\isacharparenleft}{\kern0pt}f\ c{\isacharcomma}{\kern0pt}\ g\ c{\isacharparenright}{\kern0pt}{\isacharparenright}{\kern0pt}\ {\isasyminter}\ t{\isachardoublequoteclose}\isanewline
\ \ \ \ \ \ \isacommand{then}\isamarkupfalse%
\ \isacommand{obtain}\isamarkupfalse%
\ a\ b\ \isakeyword{where}\ hx{\isacharcolon}{\kern0pt}\ {\isachardoublequoteopen}y\ {\isacharequal}{\kern0pt}\ {\isacharparenleft}{\kern0pt}a{\isacharcomma}{\kern0pt}\ b{\isacharparenright}{\kern0pt}{\isachardoublequoteclose}\ \isakeyword{and}\ {\isachardoublequoteopen}{\isacharparenleft}{\kern0pt}f\ a{\isacharcomma}{\kern0pt}\ f\ b{\isacharparenright}{\kern0pt}\ {\isasymin}\ r\ {\isasymor}\ {\isacharparenleft}{\kern0pt}f\ a\ {\isacharequal}{\kern0pt}\ f\ b\ {\isasymand}\ {\isacharparenleft}{\kern0pt}g\ a{\isacharcomma}{\kern0pt}\ g\ b{\isacharparenright}{\kern0pt}\ {\isasymin}\ s{\isacharparenright}{\kern0pt}{\isachardoublequoteclose}\isanewline
\ \ \ \ \ \ \ \ \isacommand{using}\isamarkupfalse%
\ in{\isacharunderscore}{\kern0pt}inv{\isacharunderscore}{\kern0pt}image\ in{\isacharunderscore}{\kern0pt}lex{\isacharunderscore}{\kern0pt}prod\ Int{\isacharunderscore}{\kern0pt}iff\ SigmaE\ UNIV{\isacharunderscore}{\kern0pt}Times{\isacharunderscore}{\kern0pt}UNIV\ inf{\isacharunderscore}{\kern0pt}top{\isacharunderscore}{\kern0pt}right\ \isacommand{by}\isamarkupfalse%
\ {\isacharparenleft}{\kern0pt}smt\ {\isacharparenleft}{\kern0pt}z{\isadigit{3}}{\isacharparenright}{\kern0pt}{\isacharparenright}{\kern0pt}\isanewline
\ \ \ \ \ \ \isacommand{then}\isamarkupfalse%
\ \isacommand{show}\isamarkupfalse%
\ {\isachardoublequoteopen}y\ {\isasymin}\ inv{\isacharunderscore}{\kern0pt}image\ r\ f\ {\isasyminter}\ t\ {\isasymunion}\ {\isacharparenleft}{\kern0pt}{\isasymUnion}a{\isasymin}range\ f{\isachardot}{\kern0pt}\ {\isacharbraceleft}{\kern0pt}x{\isachardot}{\kern0pt}\ f\ x\ {\isacharequal}{\kern0pt}\ a{\isacharbraceright}{\kern0pt}\ {\isasymtimes}\ {\isacharbraceleft}{\kern0pt}x{\isachardot}{\kern0pt}\ f\ x\ {\isacharequal}{\kern0pt}\ a{\isacharbraceright}{\kern0pt}\ {\isasyminter}\ inv{\isacharunderscore}{\kern0pt}image\ s\ g\ {\isasyminter}\ t{\isacharparenright}{\kern0pt}{\isachardoublequoteclose}\ \isanewline
\ \ \ \ \ \ \ \ \isacommand{using}\isamarkupfalse%
\ hy\ \isacommand{by}\isamarkupfalse%
\ auto\isanewline
\ \ \ \ \isacommand{qed}\isamarkupfalse%
\isanewline
\ \ \ \ \isacommand{show}\isamarkupfalse%
\ {\isachardoublequoteopen}inv{\isacharunderscore}{\kern0pt}image\ r\ f\ {\isasyminter}\ t\ {\isasymunion}\ {\isacharparenleft}{\kern0pt}{\isasymUnion}a{\isasymin}range\ f{\isachardot}{\kern0pt}\ {\isacharbraceleft}{\kern0pt}x{\isachardot}{\kern0pt}\ f\ x\ {\isacharequal}{\kern0pt}\ a{\isacharbraceright}{\kern0pt}\ {\isasymtimes}\ {\isacharbraceleft}{\kern0pt}x{\isachardot}{\kern0pt}\ f\ x\ {\isacharequal}{\kern0pt}\ a{\isacharbraceright}{\kern0pt}\ {\isasyminter}\ inv{\isacharunderscore}{\kern0pt}image\ s\ g\ {\isasyminter}\ t{\isacharparenright}{\kern0pt}\ {\isasymsubseteq}\ \isanewline
\ \ \ \ \ \ inv{\isacharunderscore}{\kern0pt}image\ {\isacharparenleft}{\kern0pt}r\ {\isacharless}{\kern0pt}{\isacharasterisk}{\kern0pt}lex{\isacharasterisk}{\kern0pt}{\isachargreater}{\kern0pt}\ s{\isacharparenright}{\kern0pt}\ {\isacharparenleft}{\kern0pt}{\isasymlambda}c{\isachardot}{\kern0pt}\ {\isacharparenleft}{\kern0pt}f\ c{\isacharcomma}{\kern0pt}\ g\ c{\isacharparenright}{\kern0pt}{\isacharparenright}{\kern0pt}\ {\isasyminter}\ t{\isachardoublequoteclose}\ \isacommand{using}\isamarkupfalse%
\ Int{\isacharunderscore}{\kern0pt}iff\ SUP{\isacharunderscore}{\kern0pt}le{\isacharunderscore}{\kern0pt}iff\ SigmaD{\isadigit{1}}\ SigmaD{\isadigit{2}}\ \isanewline
\ \ \ \ \ \ in{\isacharunderscore}{\kern0pt}inv{\isacharunderscore}{\kern0pt}image\ in{\isacharunderscore}{\kern0pt}lex{\isacharunderscore}{\kern0pt}prod\ mem{\isacharunderscore}{\kern0pt}Collect{\isacharunderscore}{\kern0pt}eq\ subrelI\ \isacommand{by}\isamarkupfalse%
\ force\isanewline
\ \ \isacommand{qed}\isamarkupfalse%
\isanewline
\ \ \isacommand{moreover}\isamarkupfalse%
\ \isacommand{have}\isamarkupfalse%
\ {\isachardoublequoteopen}{\isacharparenleft}{\kern0pt}{\isacharparenleft}{\kern0pt}inv{\isacharunderscore}{\kern0pt}image\ r\ f{\isacharparenright}{\kern0pt}\ {\isasyminter}\ t{\isacharparenright}{\kern0pt}\ O\isanewline
\ \ \ \ {\isacharparenleft}{\kern0pt}{\isasymUnion}\ a\ {\isasymin}\ range\ f{\isachardot}{\kern0pt}\ {\isacharparenleft}{\kern0pt}{\isacharbraceleft}{\kern0pt}x{\isachardot}{\kern0pt}\ f\ x\ {\isacharequal}{\kern0pt}\ a{\isacharbraceright}{\kern0pt}\ {\isasymtimes}\ {\isacharbraceleft}{\kern0pt}x{\isachardot}{\kern0pt}\ f\ x\ {\isacharequal}{\kern0pt}\ a{\isacharbraceright}{\kern0pt}\ {\isasyminter}\ {\isacharparenleft}{\kern0pt}inv{\isacharunderscore}{\kern0pt}image\ s\ g{\isacharparenright}{\kern0pt}{\isacharparenright}{\kern0pt}\ {\isasyminter}\ t{\isacharparenright}{\kern0pt}\ {\isasymsubseteq}\ {\isacharparenleft}{\kern0pt}inv{\isacharunderscore}{\kern0pt}image\ r\ f{\isacharparenright}{\kern0pt}\ {\isasyminter}\ t{\isachardoublequoteclose}\isanewline
\ \ \ \isacommand{using}\isamarkupfalse%
\ h{\isadigit{3}}\ trans{\isacharunderscore}{\kern0pt}O{\isacharunderscore}{\kern0pt}subset\ \isacommand{by}\isamarkupfalse%
\ fastforce\isanewline
\ \ \isacommand{moreover}\isamarkupfalse%
\ \isacommand{have}\isamarkupfalse%
\ {\isachardoublequoteopen}wf\ {\isacharparenleft}{\kern0pt}{\isasymUnion}\ a\ {\isasymin}\ range\ f{\isachardot}{\kern0pt}\ {\isacharbraceleft}{\kern0pt}x{\isachardot}{\kern0pt}\ f\ x\ {\isacharequal}{\kern0pt}\ a{\isacharbraceright}{\kern0pt}\ {\isasymtimes}\ {\isacharbraceleft}{\kern0pt}x{\isachardot}{\kern0pt}\ f\ x\ {\isacharequal}{\kern0pt}\ a{\isacharbraceright}{\kern0pt}\ {\isasyminter}\ {\isacharparenleft}{\kern0pt}inv{\isacharunderscore}{\kern0pt}image\ s\ g{\isacharparenright}{\kern0pt}\ {\isasyminter}\ t{\isacharparenright}{\kern0pt}{\isachardoublequoteclose}\isanewline
\ \ \ \ \isacommand{apply}\isamarkupfalse%
{\isacharparenleft}{\kern0pt}rule\ wf{\isacharunderscore}{\kern0pt}UN{\isacharcomma}{\kern0pt}\ auto\ simp\ add{\isacharcolon}{\kern0pt}\ h{\isadigit{2}}{\isacharparenright}{\kern0pt}\isanewline
\ \ \ \ \isacommand{done}\isamarkupfalse%
\isanewline
\ \ \isacommand{ultimately}\isamarkupfalse%
\ \isacommand{show}\isamarkupfalse%
\ {\isachardoublequoteopen}wf\ {\isacharparenleft}{\kern0pt}inv{\isacharunderscore}{\kern0pt}image\ {\isacharparenleft}{\kern0pt}r\ {\isacharless}{\kern0pt}{\isacharasterisk}{\kern0pt}lex{\isacharasterisk}{\kern0pt}{\isachargreater}{\kern0pt}\ s{\isacharparenright}{\kern0pt}\ {\isacharparenleft}{\kern0pt}{\isasymlambda}\ c{\isachardot}{\kern0pt}\ {\isacharparenleft}{\kern0pt}f\ c{\isacharcomma}{\kern0pt}\ g\ c{\isacharparenright}{\kern0pt}{\isacharparenright}{\kern0pt}\ {\isasyminter}\ t{\isacharparenright}{\kern0pt}{\isachardoublequoteclose}\ \isanewline
\ \ \ \ \isacommand{using}\isamarkupfalse%
\ wf{\isacharunderscore}{\kern0pt}union{\isacharunderscore}{\kern0pt}compatible{\isacharbrackleft}{\kern0pt}OF\ h{\isadigit{1}}{\isacharbrackright}{\kern0pt}\ \isacommand{by}\isamarkupfalse%
\ fastforce\isanewline
\isacommand{qed}\isamarkupfalse%
%
\endisatagproof
{\isafoldproof}%
%
\isadelimproof
\isanewline
%
\endisadelimproof
\isanewline
\isacommand{lemma}\isamarkupfalse%
\ wf{\isacharunderscore}{\kern0pt}prod{\isacharunderscore}{\kern0pt}lex{\isacharunderscore}{\kern0pt}fibers{\isacharcolon}{\kern0pt}\ \isanewline
\ \ \isakeyword{fixes}\ r\ {\isacharcolon}{\kern0pt}{\isacharcolon}{\kern0pt}\ {\isachardoublequoteopen}{\isacharparenleft}{\kern0pt}{\isacharprime}{\kern0pt}a\ {\isasymtimes}\ {\isacharprime}{\kern0pt}a{\isacharparenright}{\kern0pt}\ set{\isachardoublequoteclose}\ \isakeyword{and}\ s\ {\isacharcolon}{\kern0pt}{\isacharcolon}{\kern0pt}\ {\isachardoublequoteopen}{\isacharparenleft}{\kern0pt}{\isacharprime}{\kern0pt}b\ {\isasymtimes}\ {\isacharprime}{\kern0pt}b{\isacharparenright}{\kern0pt}\ set{\isachardoublequoteclose}\ \isakeyword{and}\ f\ {\isacharcolon}{\kern0pt}{\isacharcolon}{\kern0pt}\ {\isachardoublequoteopen}{\isacharprime}{\kern0pt}c\ {\isasymRightarrow}\ {\isacharprime}{\kern0pt}a{\isachardoublequoteclose}\ \isakeyword{and}\ g\ {\isacharcolon}{\kern0pt}{\isacharcolon}{\kern0pt}\ {\isachardoublequoteopen}{\isacharprime}{\kern0pt}c\ {\isasymRightarrow}\ {\isacharprime}{\kern0pt}b{\isachardoublequoteclose}\isanewline
\ \ \isakeyword{assumes}\ h{\isadigit{1}}{\isacharcolon}{\kern0pt}\ {\isachardoublequoteopen}wf\ {\isacharparenleft}{\kern0pt}inv{\isacharunderscore}{\kern0pt}image\ r\ f{\isacharparenright}{\kern0pt}{\isachardoublequoteclose}\ \isakeyword{and}\isanewline
\ \ h{\isadigit{2}}{\isacharcolon}{\kern0pt}\ {\isachardoublequoteopen}{\isasymAnd}\ a{\isachardot}{\kern0pt}\ a\ {\isasymin}\ range\ f\ {\isasymLongrightarrow}\ wf\ {\isacharparenleft}{\kern0pt}{\isacharbraceleft}{\kern0pt}x{\isachardot}{\kern0pt}\ f\ x\ {\isacharequal}{\kern0pt}\ a{\isacharbraceright}{\kern0pt}\ {\isasymtimes}\ {\isacharbraceleft}{\kern0pt}x{\isachardot}{\kern0pt}\ f\ x\ {\isacharequal}{\kern0pt}\ a{\isacharbraceright}{\kern0pt}\ {\isasyminter}\ {\isacharparenleft}{\kern0pt}inv{\isacharunderscore}{\kern0pt}image\ s\ g{\isacharparenright}{\kern0pt}{\isacharparenright}{\kern0pt}{\isachardoublequoteclose}\isanewline
\ \ \isakeyword{shows}\ {\isachardoublequoteopen}wf\ {\isacharparenleft}{\kern0pt}inv{\isacharunderscore}{\kern0pt}image\ {\isacharparenleft}{\kern0pt}r\ {\isacharless}{\kern0pt}{\isacharasterisk}{\kern0pt}lex{\isacharasterisk}{\kern0pt}{\isachargreater}{\kern0pt}\ s{\isacharparenright}{\kern0pt}\ {\isacharparenleft}{\kern0pt}{\isasymlambda}\ c{\isachardot}{\kern0pt}\ {\isacharparenleft}{\kern0pt}f\ c{\isacharcomma}{\kern0pt}\ g\ c{\isacharparenright}{\kern0pt}{\isacharparenright}{\kern0pt}{\isacharparenright}{\kern0pt}{\isachardoublequoteclose}\isanewline
%
\isadelimproof
\ \ %
\endisadelimproof
%
\isatagproof
\isacommand{using}\isamarkupfalse%
\ assms\ trans{\isacharunderscore}{\kern0pt}def\ wf{\isacharunderscore}{\kern0pt}prod{\isacharunderscore}{\kern0pt}lex{\isacharunderscore}{\kern0pt}fibers{\isacharunderscore}{\kern0pt}inter{\isacharbrackleft}{\kern0pt}of\ r\ f\ UNIV\ s\ g{\isacharbrackright}{\kern0pt}\ inf{\isacharunderscore}{\kern0pt}top{\isacharunderscore}{\kern0pt}right\isanewline
\ \ \isacommand{by}\isamarkupfalse%
\ {\isacharparenleft}{\kern0pt}metis\ {\isacharparenleft}{\kern0pt}mono{\isacharunderscore}{\kern0pt}tags{\isacharcomma}{\kern0pt}\ lifting{\isacharparenright}{\kern0pt}\ iso{\isacharunderscore}{\kern0pt}tuple{\isacharunderscore}{\kern0pt}UNIV{\isacharunderscore}{\kern0pt}I{\isacharparenright}{\kern0pt}%
\endisatagproof
{\isafoldproof}%
%
\isadelimproof
\isanewline
%
\endisadelimproof
\isanewline
\isacommand{context}\isamarkupfalse%
\ monoid\isanewline
\isanewline
\isakeyword{begin}%
\isadelimdocument
%
\endisadelimdocument
%
\isatagdocument
%
\isamarkupsubsection{Pointwise set multiplication in monoid: definition and key-lemmas%
}
\isamarkuptrue%
%
\endisatagdocument
{\isafolddocument}%
%
\isadelimdocument
%
\endisadelimdocument
\isacommand{inductive{\isacharunderscore}{\kern0pt}set}\isamarkupfalse%
\ smul\ {\isacharcolon}{\kern0pt}{\isacharcolon}{\kern0pt}\ {\isachardoublequoteopen}{\isacharprime}{\kern0pt}a\ set\ {\isasymRightarrow}\ {\isacharprime}{\kern0pt}a\ set\ {\isasymRightarrow}\ {\isacharprime}{\kern0pt}a\ set{\isachardoublequoteclose}\ \isakeyword{for}\ A\ B\ \isanewline
\ \ \isakeyword{where}\isanewline
\ \ \ \ smulI{\isacharbrackleft}{\kern0pt}intro{\isacharbrackright}{\kern0pt}{\isacharcolon}{\kern0pt}\ {\isachardoublequoteopen}{\isasymlbrakk}a\ {\isasymin}\ A{\isacharsemicolon}{\kern0pt}\ a\ {\isasymin}\ M{\isacharsemicolon}{\kern0pt}\ b\ {\isasymin}\ B{\isacharsemicolon}{\kern0pt}\ b\ {\isasymin}\ M{\isasymrbrakk}\ {\isasymLongrightarrow}\ a\ {\isasymcdot}\ b\ {\isasymin}\ smul\ A\ B{\isachardoublequoteclose}\isanewline
\isanewline
\isacommand{syntax}\isamarkupfalse%
\ {\isachardoublequoteopen}smul{\isachardoublequoteclose}\ {\isacharcolon}{\kern0pt}{\isacharcolon}{\kern0pt}\ \ {\isachardoublequoteopen}{\isacharprime}{\kern0pt}a\ set\ {\isasymRightarrow}\ {\isacharprime}{\kern0pt}a\ set\ {\isasymRightarrow}\ {\isacharprime}{\kern0pt}a\ set{\isachardoublequoteclose}\ {\isacharparenleft}{\kern0pt}{\isachardoublequoteopen}{\isacharparenleft}{\kern0pt}{\isacharunderscore}{\kern0pt}\ {\isasymcdots}\ {\isacharunderscore}{\kern0pt}{\isacharparenright}{\kern0pt}{\isachardoublequoteclose}{\isacharparenright}{\kern0pt}\isanewline
\isanewline
\isacommand{lemma}\isamarkupfalse%
\ smul{\isacharunderscore}{\kern0pt}eq{\isacharcolon}{\kern0pt}\ {\isachardoublequoteopen}smul\ A\ B\ {\isacharequal}{\kern0pt}\ {\isacharbraceleft}{\kern0pt}c{\isachardot}{\kern0pt}\ {\isasymexists}a\ {\isasymin}\ A\ {\isasyminter}\ M{\isachardot}{\kern0pt}\ {\isasymexists}b\ {\isasymin}\ B\ {\isasyminter}\ M{\isachardot}{\kern0pt}\ c\ {\isacharequal}{\kern0pt}\ a\ {\isasymcdot}\ b{\isacharbraceright}{\kern0pt}{\isachardoublequoteclose}\isanewline
%
\isadelimproof
\ \ %
\endisadelimproof
%
\isatagproof
\isacommand{by}\isamarkupfalse%
\ {\isacharparenleft}{\kern0pt}auto\ simp{\isacharcolon}{\kern0pt}\ smul{\isachardot}{\kern0pt}simps\ elim{\isacharbang}{\kern0pt}{\isacharcolon}{\kern0pt}\ smul{\isachardot}{\kern0pt}cases{\isacharparenright}{\kern0pt}%
\endisatagproof
{\isafoldproof}%
%
\isadelimproof
\isanewline
%
\endisadelimproof
\isanewline
\isacommand{lemma}\isamarkupfalse%
\ smul{\isacharcolon}{\kern0pt}\ {\isachardoublequoteopen}smul\ A\ B\ {\isacharequal}{\kern0pt}\ {\isacharparenleft}{\kern0pt}{\isasymUnion}a\ {\isasymin}\ A\ {\isasyminter}\ M{\isachardot}{\kern0pt}\ {\isasymUnion}b\ {\isasymin}\ B\ {\isasyminter}\ M{\isachardot}{\kern0pt}\ {\isacharbraceleft}{\kern0pt}a\ {\isasymcdot}\ b{\isacharbraceright}{\kern0pt}{\isacharparenright}{\kern0pt}{\isachardoublequoteclose}\isanewline
%
\isadelimproof
\ \ %
\endisadelimproof
%
\isatagproof
\isacommand{by}\isamarkupfalse%
\ {\isacharparenleft}{\kern0pt}auto\ simp{\isacharcolon}{\kern0pt}\ smul{\isacharunderscore}{\kern0pt}eq{\isacharparenright}{\kern0pt}%
\endisatagproof
{\isafoldproof}%
%
\isadelimproof
\isanewline
%
\endisadelimproof
\isanewline
\isacommand{lemma}\isamarkupfalse%
\ smul{\isacharunderscore}{\kern0pt}subset{\isacharunderscore}{\kern0pt}carrier{\isacharcolon}{\kern0pt}\ {\isachardoublequoteopen}smul\ A\ B\ {\isasymsubseteq}\ M{\isachardoublequoteclose}\isanewline
%
\isadelimproof
\ \ %
\endisadelimproof
%
\isatagproof
\isacommand{by}\isamarkupfalse%
\ {\isacharparenleft}{\kern0pt}auto\ simp{\isacharcolon}{\kern0pt}\ smul{\isacharunderscore}{\kern0pt}eq{\isacharparenright}{\kern0pt}%
\endisatagproof
{\isafoldproof}%
%
\isadelimproof
\isanewline
%
\endisadelimproof
\isanewline
\isacommand{lemma}\isamarkupfalse%
\ smul{\isacharunderscore}{\kern0pt}Int{\isacharunderscore}{\kern0pt}carrier\ {\isacharbrackleft}{\kern0pt}simp{\isacharbrackright}{\kern0pt}{\isacharcolon}{\kern0pt}\ {\isachardoublequoteopen}smul\ A\ B\ {\isasyminter}\ M\ {\isacharequal}{\kern0pt}\ smul\ A\ B{\isachardoublequoteclose}\isanewline
%
\isadelimproof
\ \ %
\endisadelimproof
%
\isatagproof
\isacommand{by}\isamarkupfalse%
\ {\isacharparenleft}{\kern0pt}simp\ add{\isacharcolon}{\kern0pt}\ Int{\isacharunderscore}{\kern0pt}absorb{\isadigit{2}}\ smul{\isacharunderscore}{\kern0pt}subset{\isacharunderscore}{\kern0pt}carrier{\isacharparenright}{\kern0pt}%
\endisatagproof
{\isafoldproof}%
%
\isadelimproof
\isanewline
%
\endisadelimproof
\isanewline
\isacommand{lemma}\isamarkupfalse%
\ smul{\isacharunderscore}{\kern0pt}mono{\isacharcolon}{\kern0pt}\ {\isachardoublequoteopen}{\isasymlbrakk}A{\isacharprime}{\kern0pt}\ {\isasymsubseteq}\ A{\isacharsemicolon}{\kern0pt}\ B{\isacharprime}{\kern0pt}\ {\isasymsubseteq}\ B{\isasymrbrakk}\ {\isasymLongrightarrow}\ smul\ A{\isacharprime}{\kern0pt}\ B{\isacharprime}{\kern0pt}\ {\isasymsubseteq}\ smul\ A\ B{\isachardoublequoteclose}\isanewline
%
\isadelimproof
\ \ %
\endisadelimproof
%
\isatagproof
\isacommand{by}\isamarkupfalse%
\ {\isacharparenleft}{\kern0pt}auto\ simp{\isacharcolon}{\kern0pt}\ smul{\isacharunderscore}{\kern0pt}eq{\isacharparenright}{\kern0pt}%
\endisatagproof
{\isafoldproof}%
%
\isadelimproof
\isanewline
%
\endisadelimproof
\isanewline
\isacommand{lemma}\isamarkupfalse%
\ smul{\isacharunderscore}{\kern0pt}insert{\isadigit{1}}{\isacharcolon}{\kern0pt}\ {\isachardoublequoteopen}NO{\isacharunderscore}{\kern0pt}MATCH\ {\isacharbraceleft}{\kern0pt}{\isacharbraceright}{\kern0pt}\ A\ {\isasymLongrightarrow}\ smul\ {\isacharparenleft}{\kern0pt}insert\ x\ A{\isacharparenright}{\kern0pt}\ B\ {\isacharequal}{\kern0pt}\ smul\ {\isacharbraceleft}{\kern0pt}x{\isacharbraceright}{\kern0pt}\ B\ {\isasymunion}\ smul\ A\ B{\isachardoublequoteclose}\isanewline
%
\isadelimproof
\ \ %
\endisadelimproof
%
\isatagproof
\isacommand{by}\isamarkupfalse%
\ {\isacharparenleft}{\kern0pt}auto\ simp{\isacharcolon}{\kern0pt}\ smul{\isacharunderscore}{\kern0pt}eq{\isacharparenright}{\kern0pt}%
\endisatagproof
{\isafoldproof}%
%
\isadelimproof
\isanewline
%
\endisadelimproof
\isanewline
\isacommand{lemma}\isamarkupfalse%
\ smul{\isacharunderscore}{\kern0pt}insert{\isadigit{2}}{\isacharcolon}{\kern0pt}\ {\isachardoublequoteopen}NO{\isacharunderscore}{\kern0pt}MATCH\ {\isacharbraceleft}{\kern0pt}{\isacharbraceright}{\kern0pt}\ B\ {\isasymLongrightarrow}\ smul\ A\ {\isacharparenleft}{\kern0pt}insert\ x\ B{\isacharparenright}{\kern0pt}\ {\isacharequal}{\kern0pt}\ smul\ A\ {\isacharbraceleft}{\kern0pt}x{\isacharbraceright}{\kern0pt}\ {\isasymunion}\ smul\ A\ B{\isachardoublequoteclose}\isanewline
%
\isadelimproof
\ \ %
\endisadelimproof
%
\isatagproof
\isacommand{by}\isamarkupfalse%
\ {\isacharparenleft}{\kern0pt}auto\ simp{\isacharcolon}{\kern0pt}\ smul{\isacharunderscore}{\kern0pt}eq{\isacharparenright}{\kern0pt}%
\endisatagproof
{\isafoldproof}%
%
\isadelimproof
\isanewline
%
\endisadelimproof
\isanewline
\isacommand{lemma}\isamarkupfalse%
\ smul{\isacharunderscore}{\kern0pt}subset{\isacharunderscore}{\kern0pt}Un{\isadigit{1}}{\isacharcolon}{\kern0pt}\ {\isachardoublequoteopen}smul\ {\isacharparenleft}{\kern0pt}A\ {\isasymunion}\ A{\isacharprime}{\kern0pt}{\isacharparenright}{\kern0pt}\ B\ {\isacharequal}{\kern0pt}\ smul\ A\ B\ {\isasymunion}\ smul\ A{\isacharprime}{\kern0pt}\ B{\isachardoublequoteclose}\isanewline
%
\isadelimproof
\ \ %
\endisadelimproof
%
\isatagproof
\isacommand{by}\isamarkupfalse%
\ {\isacharparenleft}{\kern0pt}auto\ simp{\isacharcolon}{\kern0pt}\ smul{\isacharunderscore}{\kern0pt}eq{\isacharparenright}{\kern0pt}%
\endisatagproof
{\isafoldproof}%
%
\isadelimproof
\isanewline
%
\endisadelimproof
\isanewline
\isacommand{lemma}\isamarkupfalse%
\ smul{\isacharunderscore}{\kern0pt}subset{\isacharunderscore}{\kern0pt}Un{\isadigit{2}}{\isacharcolon}{\kern0pt}\ {\isachardoublequoteopen}smul\ A\ {\isacharparenleft}{\kern0pt}B\ {\isasymunion}\ B{\isacharprime}{\kern0pt}{\isacharparenright}{\kern0pt}\ {\isacharequal}{\kern0pt}\ smul\ A\ B\ {\isasymunion}\ smul\ A\ B{\isacharprime}{\kern0pt}{\isachardoublequoteclose}\isanewline
%
\isadelimproof
\ \ %
\endisadelimproof
%
\isatagproof
\isacommand{by}\isamarkupfalse%
\ {\isacharparenleft}{\kern0pt}auto\ simp{\isacharcolon}{\kern0pt}\ smul{\isacharunderscore}{\kern0pt}eq{\isacharparenright}{\kern0pt}%
\endisatagproof
{\isafoldproof}%
%
\isadelimproof
\isanewline
%
\endisadelimproof
\isanewline
\isacommand{lemma}\isamarkupfalse%
\ smul{\isacharunderscore}{\kern0pt}subset{\isacharunderscore}{\kern0pt}Union{\isadigit{1}}{\isacharcolon}{\kern0pt}\ {\isachardoublequoteopen}smul\ {\isacharparenleft}{\kern0pt}{\isasymUnion}\ A{\isacharparenright}{\kern0pt}\ B\ {\isacharequal}{\kern0pt}\ {\isacharparenleft}{\kern0pt}{\isasymUnion}\ a\ {\isasymin}\ A{\isachardot}{\kern0pt}\ smul\ a\ B{\isacharparenright}{\kern0pt}{\isachardoublequoteclose}\isanewline
%
\isadelimproof
\ \ %
\endisadelimproof
%
\isatagproof
\isacommand{by}\isamarkupfalse%
\ {\isacharparenleft}{\kern0pt}auto\ simp{\isacharcolon}{\kern0pt}\ smul{\isacharunderscore}{\kern0pt}eq{\isacharparenright}{\kern0pt}%
\endisatagproof
{\isafoldproof}%
%
\isadelimproof
\isanewline
%
\endisadelimproof
\isanewline
\isacommand{lemma}\isamarkupfalse%
\ smul{\isacharunderscore}{\kern0pt}subset{\isacharunderscore}{\kern0pt}Union{\isadigit{2}}{\isacharcolon}{\kern0pt}\ {\isachardoublequoteopen}smul\ A\ {\isacharparenleft}{\kern0pt}{\isasymUnion}\ B{\isacharparenright}{\kern0pt}\ {\isacharequal}{\kern0pt}\ {\isacharparenleft}{\kern0pt}{\isasymUnion}\ b\ {\isasymin}\ B{\isachardot}{\kern0pt}\ smul\ A\ b{\isacharparenright}{\kern0pt}{\isachardoublequoteclose}\isanewline
%
\isadelimproof
\ \ %
\endisadelimproof
%
\isatagproof
\isacommand{by}\isamarkupfalse%
\ {\isacharparenleft}{\kern0pt}auto\ simp{\isacharcolon}{\kern0pt}\ smul{\isacharunderscore}{\kern0pt}eq{\isacharparenright}{\kern0pt}%
\endisatagproof
{\isafoldproof}%
%
\isadelimproof
\isanewline
%
\endisadelimproof
\isanewline
\isacommand{lemma}\isamarkupfalse%
\ smul{\isacharunderscore}{\kern0pt}subset{\isacharunderscore}{\kern0pt}insert{\isacharcolon}{\kern0pt}\ {\isachardoublequoteopen}smul\ A\ B\ {\isasymsubseteq}\ smul\ A\ {\isacharparenleft}{\kern0pt}insert\ x\ B{\isacharparenright}{\kern0pt}{\isachardoublequoteclose}\ {\isachardoublequoteopen}smul\ A\ B\ {\isasymsubseteq}\ smul\ {\isacharparenleft}{\kern0pt}insert\ x\ A{\isacharparenright}{\kern0pt}\ B{\isachardoublequoteclose}\isanewline
%
\isadelimproof
\ \ %
\endisadelimproof
%
\isatagproof
\isacommand{by}\isamarkupfalse%
\ {\isacharparenleft}{\kern0pt}auto\ simp{\isacharcolon}{\kern0pt}\ smul{\isacharunderscore}{\kern0pt}eq{\isacharparenright}{\kern0pt}%
\endisatagproof
{\isafoldproof}%
%
\isadelimproof
\isanewline
%
\endisadelimproof
\isanewline
\isacommand{lemma}\isamarkupfalse%
\ smul{\isacharunderscore}{\kern0pt}subset{\isacharunderscore}{\kern0pt}Un{\isacharcolon}{\kern0pt}\ {\isachardoublequoteopen}smul\ A\ B\ {\isasymsubseteq}\ smul\ A\ {\isacharparenleft}{\kern0pt}B{\isasymunion}C{\isacharparenright}{\kern0pt}{\isachardoublequoteclose}\ {\isachardoublequoteopen}smul\ A\ B\ {\isasymsubseteq}\ smul\ {\isacharparenleft}{\kern0pt}A{\isasymunion}C{\isacharparenright}{\kern0pt}\ B{\isachardoublequoteclose}\isanewline
%
\isadelimproof
\ \ %
\endisadelimproof
%
\isatagproof
\isacommand{by}\isamarkupfalse%
\ {\isacharparenleft}{\kern0pt}auto\ simp{\isacharcolon}{\kern0pt}\ smul{\isacharunderscore}{\kern0pt}eq{\isacharparenright}{\kern0pt}%
\endisatagproof
{\isafoldproof}%
%
\isadelimproof
\isanewline
%
\endisadelimproof
\isanewline
\isacommand{lemma}\isamarkupfalse%
\ smul{\isacharunderscore}{\kern0pt}empty\ {\isacharbrackleft}{\kern0pt}simp{\isacharbrackright}{\kern0pt}{\isacharcolon}{\kern0pt}\ {\isachardoublequoteopen}smul\ A\ {\isacharbraceleft}{\kern0pt}{\isacharbraceright}{\kern0pt}\ {\isacharequal}{\kern0pt}\ {\isacharbraceleft}{\kern0pt}{\isacharbraceright}{\kern0pt}{\isachardoublequoteclose}\ {\isachardoublequoteopen}smul\ {\isacharbraceleft}{\kern0pt}{\isacharbraceright}{\kern0pt}\ A\ {\isacharequal}{\kern0pt}\ {\isacharbraceleft}{\kern0pt}{\isacharbraceright}{\kern0pt}{\isachardoublequoteclose}\isanewline
%
\isadelimproof
\ \ %
\endisadelimproof
%
\isatagproof
\isacommand{by}\isamarkupfalse%
\ {\isacharparenleft}{\kern0pt}auto\ simp{\isacharcolon}{\kern0pt}\ smul{\isacharunderscore}{\kern0pt}eq{\isacharparenright}{\kern0pt}%
\endisatagproof
{\isafoldproof}%
%
\isadelimproof
\isanewline
%
\endisadelimproof
\isanewline
\isacommand{lemma}\isamarkupfalse%
\ smul{\isacharunderscore}{\kern0pt}empty{\isacharprime}{\kern0pt}{\isacharcolon}{\kern0pt}\isanewline
\ \ \isakeyword{assumes}\ {\isachardoublequoteopen}A\ {\isasyminter}\ M\ {\isacharequal}{\kern0pt}\ {\isacharbraceleft}{\kern0pt}{\isacharbraceright}{\kern0pt}{\isachardoublequoteclose}\isanewline
\ \ \isakeyword{shows}\ {\isachardoublequoteopen}smul\ B\ A\ {\isacharequal}{\kern0pt}\ {\isacharbraceleft}{\kern0pt}{\isacharbraceright}{\kern0pt}{\isachardoublequoteclose}\ {\isachardoublequoteopen}smul\ A\ B\ {\isacharequal}{\kern0pt}\ {\isacharbraceleft}{\kern0pt}{\isacharbraceright}{\kern0pt}{\isachardoublequoteclose}\isanewline
%
\isadelimproof
\ \ %
\endisadelimproof
%
\isatagproof
\isacommand{using}\isamarkupfalse%
\ assms\ \isacommand{by}\isamarkupfalse%
\ {\isacharparenleft}{\kern0pt}auto\ simp{\isacharcolon}{\kern0pt}\ smul{\isacharunderscore}{\kern0pt}eq{\isacharparenright}{\kern0pt}%
\endisatagproof
{\isafoldproof}%
%
\isadelimproof
\isanewline
%
\endisadelimproof
\isanewline
\isacommand{lemma}\isamarkupfalse%
\ smul{\isacharunderscore}{\kern0pt}is{\isacharunderscore}{\kern0pt}empty{\isacharunderscore}{\kern0pt}iff\ {\isacharbrackleft}{\kern0pt}simp{\isacharbrackright}{\kern0pt}{\isacharcolon}{\kern0pt}\ {\isachardoublequoteopen}smul\ A\ B\ {\isacharequal}{\kern0pt}\ {\isacharbraceleft}{\kern0pt}{\isacharbraceright}{\kern0pt}\ {\isasymlongleftrightarrow}\ A\ {\isasyminter}\ M\ {\isacharequal}{\kern0pt}\ {\isacharbraceleft}{\kern0pt}{\isacharbraceright}{\kern0pt}\ {\isasymor}\ B\ {\isasyminter}\ M\ {\isacharequal}{\kern0pt}\ {\isacharbraceleft}{\kern0pt}{\isacharbraceright}{\kern0pt}{\isachardoublequoteclose}\isanewline
%
\isadelimproof
\ \ %
\endisadelimproof
%
\isatagproof
\isacommand{by}\isamarkupfalse%
\ {\isacharparenleft}{\kern0pt}auto\ simp{\isacharcolon}{\kern0pt}\ smul{\isacharunderscore}{\kern0pt}eq{\isacharparenright}{\kern0pt}%
\endisatagproof
{\isafoldproof}%
%
\isadelimproof
\isanewline
%
\endisadelimproof
\isanewline
\isacommand{lemma}\isamarkupfalse%
\ smul{\isacharunderscore}{\kern0pt}D\ {\isacharbrackleft}{\kern0pt}simp{\isacharbrackright}{\kern0pt}{\isacharcolon}{\kern0pt}\ {\isachardoublequoteopen}smul\ A\ {\isacharbraceleft}{\kern0pt}{\isasymone}{\isacharbraceright}{\kern0pt}\ {\isacharequal}{\kern0pt}\ A\ {\isasyminter}\ M{\isachardoublequoteclose}\ {\isachardoublequoteopen}smul\ {\isacharbraceleft}{\kern0pt}{\isasymone}{\isacharbraceright}{\kern0pt}\ A\ {\isacharequal}{\kern0pt}\ A\ {\isasyminter}\ M{\isachardoublequoteclose}\isanewline
%
\isadelimproof
\ \ %
\endisadelimproof
%
\isatagproof
\isacommand{by}\isamarkupfalse%
\ {\isacharparenleft}{\kern0pt}auto\ simp{\isacharcolon}{\kern0pt}\ smul{\isacharunderscore}{\kern0pt}eq{\isacharparenright}{\kern0pt}%
\endisatagproof
{\isafoldproof}%
%
\isadelimproof
\isanewline
%
\endisadelimproof
\isanewline
\isacommand{lemma}\isamarkupfalse%
\ smul{\isacharunderscore}{\kern0pt}Int{\isacharunderscore}{\kern0pt}carrier{\isacharunderscore}{\kern0pt}eq\ {\isacharbrackleft}{\kern0pt}simp{\isacharbrackright}{\kern0pt}{\isacharcolon}{\kern0pt}\ {\isachardoublequoteopen}smul\ A\ {\isacharparenleft}{\kern0pt}B\ {\isasyminter}\ M{\isacharparenright}{\kern0pt}\ {\isacharequal}{\kern0pt}\ smul\ A\ B{\isachardoublequoteclose}\ {\isachardoublequoteopen}smul\ {\isacharparenleft}{\kern0pt}A\ {\isasyminter}\ M{\isacharparenright}{\kern0pt}\ B\ {\isacharequal}{\kern0pt}\ smul\ A\ B{\isachardoublequoteclose}\isanewline
%
\isadelimproof
\ \ %
\endisadelimproof
%
\isatagproof
\isacommand{by}\isamarkupfalse%
\ {\isacharparenleft}{\kern0pt}auto\ simp{\isacharcolon}{\kern0pt}\ smul{\isacharunderscore}{\kern0pt}eq{\isacharparenright}{\kern0pt}%
\endisatagproof
{\isafoldproof}%
%
\isadelimproof
\isanewline
%
\endisadelimproof
\isanewline
\isacommand{lemma}\isamarkupfalse%
\ smul{\isacharunderscore}{\kern0pt}assoc{\isacharcolon}{\kern0pt}\isanewline
\ \ \isakeyword{shows}\ {\isachardoublequoteopen}smul\ {\isacharparenleft}{\kern0pt}smul\ A\ B{\isacharparenright}{\kern0pt}\ C\ {\isacharequal}{\kern0pt}\ smul\ A\ {\isacharparenleft}{\kern0pt}smul\ B\ C{\isacharparenright}{\kern0pt}{\isachardoublequoteclose}\isanewline
%
\isadelimproof
\ \ %
\endisadelimproof
%
\isatagproof
\isacommand{by}\isamarkupfalse%
\ {\isacharparenleft}{\kern0pt}fastforce\ simp\ add{\isacharcolon}{\kern0pt}\ smul{\isacharunderscore}{\kern0pt}eq\ associative\ Bex{\isacharunderscore}{\kern0pt}def{\isacharparenright}{\kern0pt}%
\endisatagproof
{\isafoldproof}%
%
\isadelimproof
\isanewline
%
\endisadelimproof
\isanewline
\isacommand{lemma}\isamarkupfalse%
\ finite{\isacharunderscore}{\kern0pt}smul{\isacharcolon}{\kern0pt}\isanewline
\ \ \isakeyword{assumes}\ {\isachardoublequoteopen}finite\ A{\isachardoublequoteclose}\ {\isachardoublequoteopen}finite\ B{\isachardoublequoteclose}\ \ \isakeyword{shows}\ {\isachardoublequoteopen}finite\ {\isacharparenleft}{\kern0pt}smul\ A\ B{\isacharparenright}{\kern0pt}{\isachardoublequoteclose}\isanewline
%
\isadelimproof
\ \ %
\endisadelimproof
%
\isatagproof
\isacommand{using}\isamarkupfalse%
\ assms\ \isacommand{by}\isamarkupfalse%
\ {\isacharparenleft}{\kern0pt}auto\ simp{\isacharcolon}{\kern0pt}\ smul{\isacharunderscore}{\kern0pt}eq{\isacharparenright}{\kern0pt}%
\endisatagproof
{\isafoldproof}%
%
\isadelimproof
\isanewline
%
\endisadelimproof
\isanewline
\isacommand{lemma}\isamarkupfalse%
\ finite{\isacharunderscore}{\kern0pt}smul{\isacharprime}{\kern0pt}{\isacharcolon}{\kern0pt}\isanewline
\ \ \isakeyword{assumes}\ {\isachardoublequoteopen}finite\ {\isacharparenleft}{\kern0pt}A\ {\isasyminter}\ M{\isacharparenright}{\kern0pt}{\isachardoublequoteclose}\ {\isachardoublequoteopen}finite\ {\isacharparenleft}{\kern0pt}B\ {\isasyminter}\ M{\isacharparenright}{\kern0pt}{\isachardoublequoteclose}\isanewline
\ \ \ \ \isakeyword{shows}\ {\isachardoublequoteopen}finite\ {\isacharparenleft}{\kern0pt}smul\ A\ B{\isacharparenright}{\kern0pt}{\isachardoublequoteclose}\isanewline
%
\isadelimproof
\ \ %
\endisadelimproof
%
\isatagproof
\isacommand{using}\isamarkupfalse%
\ assms\ \isacommand{by}\isamarkupfalse%
\ {\isacharparenleft}{\kern0pt}auto\ simp{\isacharcolon}{\kern0pt}\ smul{\isacharunderscore}{\kern0pt}eq{\isacharparenright}{\kern0pt}%
\endisatagproof
{\isafoldproof}%
%
\isadelimproof
%
\endisadelimproof
%
\isadelimdocument
%
\endisadelimdocument
%
\isatagdocument
%
\isamarkupsubsection{Exponentiation in a monoid: definitions and lemmas%
}
\isamarkuptrue%
%
\endisatagdocument
{\isafolddocument}%
%
\isadelimdocument
%
\endisadelimdocument
\isacommand{primrec}\isamarkupfalse%
\ power\ {\isacharcolon}{\kern0pt}{\isacharcolon}{\kern0pt}\ {\isachardoublequoteopen}{\isacharprime}{\kern0pt}a\ {\isasymRightarrow}\ nat\ {\isasymRightarrow}\ {\isacharprime}{\kern0pt}a{\isachardoublequoteclose}\ {\isacharparenleft}{\kern0pt}\isakeyword{infix}\ {\isachardoublequoteopen}{\isacharcircum}{\kern0pt}{\isachardoublequoteclose}\ {\isadigit{1}}{\isadigit{0}}{\isadigit{0}}{\isacharparenright}{\kern0pt}\isanewline
\ \ \isakeyword{where}\isanewline
\ \ power{\isadigit{0}}{\isacharcolon}{\kern0pt}\ {\isachardoublequoteopen}power\ g\ {\isadigit{0}}\ {\isacharequal}{\kern0pt}\ {\isasymone}{\isachardoublequoteclose}\isanewline
{\isacharbar}{\kern0pt}\ power{\isacharunderscore}{\kern0pt}suc{\isacharcolon}{\kern0pt}\ {\isachardoublequoteopen}power\ g\ {\isacharparenleft}{\kern0pt}Suc\ n{\isacharparenright}{\kern0pt}\ {\isacharequal}{\kern0pt}\ power\ g\ n\ {\isasymcdot}\ g{\isachardoublequoteclose}\isanewline
\isanewline
\isacommand{lemma}\isamarkupfalse%
\ power{\isacharunderscore}{\kern0pt}one{\isacharcolon}{\kern0pt}\isanewline
\ \ \isakeyword{assumes}\ {\isachardoublequoteopen}g\ {\isasymin}\ M{\isachardoublequoteclose}\isanewline
\ \ \isakeyword{shows}\ {\isachardoublequoteopen}power\ g\ {\isadigit{1}}\ {\isacharequal}{\kern0pt}\ g{\isachardoublequoteclose}%
\isadelimproof
\ %
\endisadelimproof
%
\isatagproof
\isacommand{using}\isamarkupfalse%
\ power{\isacharunderscore}{\kern0pt}def\ power{\isadigit{0}}\ assms\ \isacommand{by}\isamarkupfalse%
\ simp%
\endisatagproof
{\isafoldproof}%
%
\isadelimproof
%
\endisadelimproof
\isanewline
\isanewline
\isacommand{lemma}\isamarkupfalse%
\ power{\isacharunderscore}{\kern0pt}mem{\isacharunderscore}{\kern0pt}carrier{\isacharcolon}{\kern0pt}\isanewline
\ \ \isakeyword{fixes}\ n\isanewline
\ \ \isakeyword{assumes}\ {\isachardoublequoteopen}g\ {\isasymin}\ M{\isachardoublequoteclose}\isanewline
\ \ \isakeyword{shows}\ {\isachardoublequoteopen}g\ {\isacharcircum}{\kern0pt}\ n\ {\isasymin}\ M{\isachardoublequoteclose}\isanewline
%
\isadelimproof
\ \ %
\endisadelimproof
%
\isatagproof
\isacommand{apply}\isamarkupfalse%
\ {\isacharparenleft}{\kern0pt}induction\ n{\isacharcomma}{\kern0pt}\ auto\ simp\ add{\isacharcolon}{\kern0pt}\ assms\ power{\isacharunderscore}{\kern0pt}def{\isacharparenright}{\kern0pt}\isanewline
\ \ \isacommand{done}\isamarkupfalse%
%
\endisatagproof
{\isafoldproof}%
%
\isadelimproof
\isanewline
%
\endisadelimproof
\isanewline
\isacommand{lemma}\isamarkupfalse%
\ power{\isacharunderscore}{\kern0pt}mult{\isacharcolon}{\kern0pt}\isanewline
\ \ \isakeyword{assumes}\ {\isachardoublequoteopen}g\ {\isasymin}\ M{\isachardoublequoteclose}\isanewline
\ \ \isakeyword{shows}\ {\isachardoublequoteopen}g\ {\isacharcircum}{\kern0pt}\ n\ {\isasymcdot}\ g\ {\isacharcircum}{\kern0pt}\ m\ {\isacharequal}{\kern0pt}\ g\ {\isacharcircum}{\kern0pt}\ {\isacharparenleft}{\kern0pt}n\ {\isacharplus}{\kern0pt}\ m{\isacharparenright}{\kern0pt}{\isachardoublequoteclose}\isanewline
%
\isadelimproof
%
\endisadelimproof
%
\isatagproof
\isacommand{proof}\isamarkupfalse%
{\isacharparenleft}{\kern0pt}induction\ m{\isacharparenright}{\kern0pt}\isanewline
\ \ \isacommand{case}\isamarkupfalse%
\ {\isadigit{0}}\isanewline
\ \ \isacommand{then}\isamarkupfalse%
\ \isacommand{show}\isamarkupfalse%
\ {\isacharquery}{\kern0pt}case\ \isacommand{using}\isamarkupfalse%
\ assms\ power{\isadigit{0}}\ right{\isacharunderscore}{\kern0pt}unit\ power{\isacharunderscore}{\kern0pt}mem{\isacharunderscore}{\kern0pt}carrier\ \isacommand{by}\isamarkupfalse%
\ simp\isanewline
\isacommand{next}\isamarkupfalse%
\isanewline
\ \ \isacommand{case}\isamarkupfalse%
\ {\isacharparenleft}{\kern0pt}Suc\ m{\isacharparenright}{\kern0pt}\isanewline
\ \ \isacommand{assume}\isamarkupfalse%
\ {\isachardoublequoteopen}g\ {\isacharcircum}{\kern0pt}\ n\ {\isasymcdot}\ g\ {\isacharcircum}{\kern0pt}\ m\ {\isacharequal}{\kern0pt}\ g\ {\isacharcircum}{\kern0pt}\ {\isacharparenleft}{\kern0pt}n\ {\isacharplus}{\kern0pt}\ m{\isacharparenright}{\kern0pt}{\isachardoublequoteclose}\isanewline
\ \ \isacommand{then}\isamarkupfalse%
\ \isacommand{show}\isamarkupfalse%
\ {\isacharquery}{\kern0pt}case\ \isacommand{using}\isamarkupfalse%
\ power{\isacharunderscore}{\kern0pt}def\ \isacommand{by}\isamarkupfalse%
\ {\isacharparenleft}{\kern0pt}smt\ {\isacharparenleft}{\kern0pt}verit{\isacharparenright}{\kern0pt}\ add{\isacharunderscore}{\kern0pt}Suc{\isacharunderscore}{\kern0pt}right\ assms\ associative\ \isanewline
\ \ \ \ power{\isacharunderscore}{\kern0pt}mem{\isacharunderscore}{\kern0pt}carrier\ power{\isacharunderscore}{\kern0pt}suc{\isacharparenright}{\kern0pt}\isanewline
\isacommand{qed}\isamarkupfalse%
%
\endisatagproof
{\isafoldproof}%
%
\isadelimproof
\isanewline
%
\endisadelimproof
\isanewline
\isacommand{lemma}\isamarkupfalse%
\ mult{\isacharunderscore}{\kern0pt}inverse{\isacharunderscore}{\kern0pt}power{\isacharcolon}{\kern0pt}\isanewline
\ \ \isakeyword{assumes}\ {\isachardoublequoteopen}g\ {\isasymin}\ M{\isachardoublequoteclose}\ \isakeyword{and}\ {\isachardoublequoteopen}invertible\ g{\isachardoublequoteclose}\isanewline
\ \ \isakeyword{shows}\ {\isachardoublequoteopen}g\ {\isacharcircum}{\kern0pt}\ n\ {\isasymcdot}\ {\isacharparenleft}{\kern0pt}{\isacharparenleft}{\kern0pt}inverse\ g{\isacharparenright}{\kern0pt}\ {\isacharcircum}{\kern0pt}\ n{\isacharparenright}{\kern0pt}\ {\isacharequal}{\kern0pt}\ {\isasymone}{\isachardoublequoteclose}\isanewline
%
\isadelimproof
%
\endisadelimproof
%
\isatagproof
\isacommand{proof}\isamarkupfalse%
{\isacharparenleft}{\kern0pt}induction\ n{\isacharparenright}{\kern0pt}\isanewline
\ \ \isacommand{case}\isamarkupfalse%
\ {\isadigit{0}}\isanewline
\ \ \isacommand{then}\isamarkupfalse%
\ \isacommand{show}\isamarkupfalse%
\ {\isacharquery}{\kern0pt}case\ \isacommand{using}\isamarkupfalse%
\ power{\isacharunderscore}{\kern0pt}{\isadigit{0}}\ \isacommand{by}\isamarkupfalse%
\ auto\isanewline
\isacommand{next}\isamarkupfalse%
\isanewline
\ \ \isacommand{case}\isamarkupfalse%
\ {\isacharparenleft}{\kern0pt}Suc\ n{\isacharparenright}{\kern0pt}\isanewline
\ \ \isacommand{assume}\isamarkupfalse%
\ hind{\isacharcolon}{\kern0pt}\ {\isachardoublequoteopen}g\ {\isacharcircum}{\kern0pt}\ n\ {\isasymcdot}\ local{\isachardot}{\kern0pt}inverse\ g\ {\isacharcircum}{\kern0pt}\ n\ {\isacharequal}{\kern0pt}\ {\isasymone}{\isachardoublequoteclose}\isanewline
\ \ \isacommand{then}\isamarkupfalse%
\ \isacommand{have}\isamarkupfalse%
\ {\isachardoublequoteopen}g\ {\isacharcircum}{\kern0pt}\ Suc\ n\ {\isasymcdot}\ inverse\ g\ {\isacharcircum}{\kern0pt}\ Suc\ n\ {\isacharequal}{\kern0pt}\ {\isacharparenleft}{\kern0pt}g\ {\isasymcdot}\ g\ {\isacharcircum}{\kern0pt}\ n{\isacharparenright}{\kern0pt}\ {\isasymcdot}\ {\isacharparenleft}{\kern0pt}inverse\ g\ {\isacharcircum}{\kern0pt}\ n\ {\isasymcdot}\ inverse\ g{\isacharparenright}{\kern0pt}{\isachardoublequoteclose}\ \isanewline
\ \ \ \ \isacommand{using}\isamarkupfalse%
\ power{\isacharunderscore}{\kern0pt}def\ power{\isacharunderscore}{\kern0pt}mult\ assms\ \isacommand{by}\isamarkupfalse%
\ {\isacharparenleft}{\kern0pt}smt\ {\isacharparenleft}{\kern0pt}z{\isadigit{3}}{\isacharparenright}{\kern0pt}\ add{\isachardot}{\kern0pt}commute\ invertible{\isacharunderscore}{\kern0pt}inverse{\isacharunderscore}{\kern0pt}closed\ \isanewline
\ \ \ \ invertible{\isacharunderscore}{\kern0pt}right{\isacharunderscore}{\kern0pt}inverse\ left{\isacharunderscore}{\kern0pt}unit\ monoid{\isachardot}{\kern0pt}associative\ monoid{\isacharunderscore}{\kern0pt}axioms\ power{\isacharunderscore}{\kern0pt}mem{\isacharunderscore}{\kern0pt}carrier\ power{\isacharunderscore}{\kern0pt}suc{\isacharparenright}{\kern0pt}\isanewline
\ \ \isacommand{then}\isamarkupfalse%
\ \isacommand{show}\isamarkupfalse%
\ {\isacharquery}{\kern0pt}case\ \isacommand{using}\isamarkupfalse%
\ associative\ power{\isacharunderscore}{\kern0pt}mem{\isacharunderscore}{\kern0pt}carrier\ assms\ hind\ \isacommand{by}\isamarkupfalse%
\ {\isacharparenleft}{\kern0pt}smt\ {\isacharparenleft}{\kern0pt}verit{\isacharcomma}{\kern0pt}\ del{\isacharunderscore}{\kern0pt}insts{\isacharparenright}{\kern0pt}\ \isanewline
\ \ \ \ composition{\isacharunderscore}{\kern0pt}closed\ invertible{\isacharunderscore}{\kern0pt}inverse{\isacharunderscore}{\kern0pt}closed\ invertible{\isacharunderscore}{\kern0pt}right{\isacharunderscore}{\kern0pt}inverse\ right{\isacharunderscore}{\kern0pt}unit{\isacharparenright}{\kern0pt}\isanewline
\isacommand{qed}\isamarkupfalse%
%
\endisatagproof
{\isafoldproof}%
%
\isadelimproof
\isanewline
%
\endisadelimproof
\isanewline
\isacommand{lemma}\isamarkupfalse%
\ inverse{\isacharunderscore}{\kern0pt}mult{\isacharunderscore}{\kern0pt}power{\isacharcolon}{\kern0pt}\isanewline
\ \ \isakeyword{assumes}\ {\isachardoublequoteopen}g\ {\isasymin}\ M{\isachardoublequoteclose}\ \isakeyword{and}\ {\isachardoublequoteopen}invertible\ g{\isachardoublequoteclose}\isanewline
\ \ \isakeyword{shows}\ {\isachardoublequoteopen}{\isacharparenleft}{\kern0pt}{\isacharparenleft}{\kern0pt}inverse\ g{\isacharparenright}{\kern0pt}\ {\isacharcircum}{\kern0pt}\ n{\isacharparenright}{\kern0pt}\ {\isasymcdot}\ g\ {\isacharcircum}{\kern0pt}\ n\ {\isacharequal}{\kern0pt}\ {\isasymone}{\isachardoublequoteclose}%
\isadelimproof
\ %
\endisadelimproof
%
\isatagproof
\isacommand{using}\isamarkupfalse%
\ assms\ \isacommand{by}\isamarkupfalse%
\ {\isacharparenleft}{\kern0pt}metis\ invertible{\isacharunderscore}{\kern0pt}inverse{\isacharunderscore}{\kern0pt}closed\ \isanewline
\ \ \ \ invertible{\isacharunderscore}{\kern0pt}inverse{\isacharunderscore}{\kern0pt}inverse\ invertible{\isacharunderscore}{\kern0pt}inverse{\isacharunderscore}{\kern0pt}invertible\ mult{\isacharunderscore}{\kern0pt}inverse{\isacharunderscore}{\kern0pt}power{\isacharparenright}{\kern0pt}%
\endisatagproof
{\isafoldproof}%
%
\isadelimproof
%
\endisadelimproof
\isanewline
\isanewline
\isacommand{lemma}\isamarkupfalse%
\ inverse{\isacharunderscore}{\kern0pt}mult{\isacharunderscore}{\kern0pt}power{\isacharunderscore}{\kern0pt}eq{\isacharcolon}{\kern0pt}\isanewline
\ \ \isakeyword{assumes}\ {\isachardoublequoteopen}g\ {\isasymin}\ M{\isachardoublequoteclose}\ \isakeyword{and}\ {\isachardoublequoteopen}invertible\ g{\isachardoublequoteclose}\isanewline
\ \ \isakeyword{shows}\ {\isachardoublequoteopen}inverse\ {\isacharparenleft}{\kern0pt}g\ {\isacharcircum}{\kern0pt}\ n{\isacharparenright}{\kern0pt}\ {\isacharequal}{\kern0pt}\ {\isacharparenleft}{\kern0pt}inverse\ g{\isacharparenright}{\kern0pt}\ {\isacharcircum}{\kern0pt}\ n{\isachardoublequoteclose}\isanewline
%
\isadelimproof
\ \ %
\endisadelimproof
%
\isatagproof
\isacommand{using}\isamarkupfalse%
\ assms\ inverse{\isacharunderscore}{\kern0pt}equality\ \isacommand{by}\isamarkupfalse%
\ {\isacharparenleft}{\kern0pt}simp\ add{\isacharcolon}{\kern0pt}\ inverse{\isacharunderscore}{\kern0pt}mult{\isacharunderscore}{\kern0pt}power\ mult{\isacharunderscore}{\kern0pt}inverse{\isacharunderscore}{\kern0pt}power\ power{\isacharunderscore}{\kern0pt}mem{\isacharunderscore}{\kern0pt}carrier{\isacharparenright}{\kern0pt}%
\endisatagproof
{\isafoldproof}%
%
\isadelimproof
\isanewline
%
\endisadelimproof
\isanewline
\isacommand{definition}\isamarkupfalse%
\ power{\isacharunderscore}{\kern0pt}int\ {\isacharcolon}{\kern0pt}{\isacharcolon}{\kern0pt}\ {\isachardoublequoteopen}{\isacharprime}{\kern0pt}a\ {\isasymRightarrow}\ int\ {\isasymRightarrow}\ {\isacharprime}{\kern0pt}a{\isachardoublequoteclose}\ {\isacharparenleft}{\kern0pt}\isakeyword{infixr}\ {\isachardoublequoteopen}powi{\isachardoublequoteclose}\ {\isadigit{8}}{\isadigit{0}}{\isacharparenright}{\kern0pt}\ \isakeyword{where}\isanewline
\ \ {\isachardoublequoteopen}power{\isacharunderscore}{\kern0pt}int\ g\ n\ {\isacharequal}{\kern0pt}\ {\isacharparenleft}{\kern0pt}if\ n\ {\isasymge}\ {\isadigit{0}}\ then\ g\ {\isacharcircum}{\kern0pt}\ {\isacharparenleft}{\kern0pt}nat\ n{\isacharparenright}{\kern0pt}\ else\ {\isacharparenleft}{\kern0pt}inverse\ g{\isacharparenright}{\kern0pt}\ {\isacharcircum}{\kern0pt}\ {\isacharparenleft}{\kern0pt}nat\ {\isacharparenleft}{\kern0pt}{\isacharminus}{\kern0pt}n{\isacharparenright}{\kern0pt}{\isacharparenright}{\kern0pt}{\isacharparenright}{\kern0pt}{\isachardoublequoteclose}\isanewline
\isanewline
\isacommand{definition}\isamarkupfalse%
\ nat{\isacharunderscore}{\kern0pt}powers\ {\isacharcolon}{\kern0pt}{\isacharcolon}{\kern0pt}\ {\isachardoublequoteopen}{\isacharprime}{\kern0pt}a\ {\isasymRightarrow}\ {\isacharprime}{\kern0pt}a\ set{\isachardoublequoteclose}\ \isakeyword{where}\ {\isachardoublequoteopen}nat{\isacharunderscore}{\kern0pt}powers\ g\ {\isacharequal}{\kern0pt}\ {\isacharparenleft}{\kern0pt}{\isacharparenleft}{\kern0pt}{\isasymlambda}\ n{\isachardot}{\kern0pt}\ g\ {\isacharcircum}{\kern0pt}\ n{\isacharparenright}{\kern0pt}\ {\isacharbackquote}{\kern0pt}\ UNIV{\isacharparenright}{\kern0pt}{\isachardoublequoteclose}\isanewline
\isanewline
\isacommand{lemma}\isamarkupfalse%
\ nat{\isacharunderscore}{\kern0pt}powers{\isacharunderscore}{\kern0pt}eq{\isacharunderscore}{\kern0pt}Union{\isacharcolon}{\kern0pt}\ {\isachardoublequoteopen}nat{\isacharunderscore}{\kern0pt}powers\ g\ {\isacharequal}{\kern0pt}\ {\isacharparenleft}{\kern0pt}{\isasymUnion}\ n{\isachardot}{\kern0pt}\ {\isacharbraceleft}{\kern0pt}g\ {\isacharcircum}{\kern0pt}\ n{\isacharbraceright}{\kern0pt}{\isacharparenright}{\kern0pt}{\isachardoublequoteclose}%
\isadelimproof
\ %
\endisadelimproof
%
\isatagproof
\isacommand{using}\isamarkupfalse%
\ nat{\isacharunderscore}{\kern0pt}powers{\isacharunderscore}{\kern0pt}def\ \isacommand{by}\isamarkupfalse%
\ auto%
\endisatagproof
{\isafoldproof}%
%
\isadelimproof
%
\endisadelimproof
\isanewline
\isanewline
\isacommand{definition}\isamarkupfalse%
\ powers\ {\isacharcolon}{\kern0pt}{\isacharcolon}{\kern0pt}\ {\isachardoublequoteopen}{\isacharprime}{\kern0pt}a\ {\isasymRightarrow}\ {\isacharprime}{\kern0pt}a\ set{\isachardoublequoteclose}\ \isakeyword{where}\ {\isachardoublequoteopen}powers\ g\ {\isacharequal}{\kern0pt}\ {\isacharparenleft}{\kern0pt}{\isacharparenleft}{\kern0pt}{\isasymlambda}\ n{\isachardot}{\kern0pt}\ g\ powi\ n{\isacharparenright}{\kern0pt}\ {\isacharbackquote}{\kern0pt}\ UNIV{\isacharparenright}{\kern0pt}{\isachardoublequoteclose}\isanewline
\isanewline
\isacommand{lemma}\isamarkupfalse%
\ nat{\isacharunderscore}{\kern0pt}powers{\isacharunderscore}{\kern0pt}subset{\isacharcolon}{\kern0pt}\isanewline
\ \ {\isachardoublequoteopen}nat{\isacharunderscore}{\kern0pt}powers\ g\ {\isasymsubseteq}\ powers\ g{\isachardoublequoteclose}\isanewline
%
\isadelimproof
%
\endisadelimproof
%
\isatagproof
\isacommand{proof}\isamarkupfalse%
\isanewline
\ \ \isacommand{fix}\isamarkupfalse%
\ x\ \isacommand{assume}\isamarkupfalse%
\ {\isachardoublequoteopen}x\ {\isasymin}\ nat{\isacharunderscore}{\kern0pt}powers\ g{\isachardoublequoteclose}\isanewline
\ \ \isacommand{then}\isamarkupfalse%
\ \isacommand{obtain}\isamarkupfalse%
\ n\ \isakeyword{where}\ {\isachardoublequoteopen}x\ {\isacharequal}{\kern0pt}\ g\ {\isacharcircum}{\kern0pt}\ n{\isachardoublequoteclose}\ \isakeyword{and}\ {\isachardoublequoteopen}nat\ n\ {\isacharequal}{\kern0pt}\ n{\isachardoublequoteclose}\ \isacommand{using}\isamarkupfalse%
\ nat{\isacharunderscore}{\kern0pt}powers{\isacharunderscore}{\kern0pt}def\ \isacommand{by}\isamarkupfalse%
\ auto\isanewline
\ \ \isacommand{then}\isamarkupfalse%
\ \isacommand{show}\isamarkupfalse%
\ {\isachardoublequoteopen}x\ {\isasymin}\ powers\ g{\isachardoublequoteclose}\ \isacommand{using}\isamarkupfalse%
\ powers{\isacharunderscore}{\kern0pt}def\ power{\isacharunderscore}{\kern0pt}int{\isacharunderscore}{\kern0pt}def\ \isanewline
\ \ \ \ \isacommand{by}\isamarkupfalse%
\ {\isacharparenleft}{\kern0pt}metis\ UNIV{\isacharunderscore}{\kern0pt}I\ image{\isacharunderscore}{\kern0pt}iff\ of{\isacharunderscore}{\kern0pt}nat{\isacharunderscore}{\kern0pt}{\isadigit{0}}{\isacharunderscore}{\kern0pt}le{\isacharunderscore}{\kern0pt}iff{\isacharparenright}{\kern0pt}\isanewline
\isacommand{qed}\isamarkupfalse%
%
\endisatagproof
{\isafoldproof}%
%
\isadelimproof
\isanewline
%
\endisadelimproof
\isanewline
\isacommand{lemma}\isamarkupfalse%
\ inverse{\isacharunderscore}{\kern0pt}nat{\isacharunderscore}{\kern0pt}powers{\isacharunderscore}{\kern0pt}subset{\isacharcolon}{\kern0pt}\isanewline
\ \ {\isachardoublequoteopen}nat{\isacharunderscore}{\kern0pt}powers\ {\isacharparenleft}{\kern0pt}inverse\ g{\isacharparenright}{\kern0pt}\ {\isasymsubseteq}\ powers\ g{\isachardoublequoteclose}\isanewline
%
\isadelimproof
%
\endisadelimproof
%
\isatagproof
\isacommand{proof}\isamarkupfalse%
\isanewline
\ \ \isacommand{fix}\isamarkupfalse%
\ x\ \isacommand{assume}\isamarkupfalse%
\ {\isachardoublequoteopen}x\ {\isasymin}\ nat{\isacharunderscore}{\kern0pt}powers\ {\isacharparenleft}{\kern0pt}inverse\ g{\isacharparenright}{\kern0pt}{\isachardoublequoteclose}\isanewline
\ \ \isacommand{then}\isamarkupfalse%
\ \isacommand{obtain}\isamarkupfalse%
\ n\ \isakeyword{where}\ hx{\isacharcolon}{\kern0pt}\ {\isachardoublequoteopen}x\ {\isacharequal}{\kern0pt}\ {\isacharparenleft}{\kern0pt}inverse\ g{\isacharparenright}{\kern0pt}\ {\isacharcircum}{\kern0pt}\ n{\isachardoublequoteclose}\ \isacommand{using}\isamarkupfalse%
\ nat{\isacharunderscore}{\kern0pt}powers{\isacharunderscore}{\kern0pt}def\ \isacommand{by}\isamarkupfalse%
\ blast\isanewline
\ \ \isacommand{then}\isamarkupfalse%
\ \isacommand{show}\isamarkupfalse%
\ {\isachardoublequoteopen}x\ {\isasymin}\ powers\ g{\isachardoublequoteclose}\isanewline
\ \ \isacommand{proof}\isamarkupfalse%
{\isacharparenleft}{\kern0pt}cases\ {\isachardoublequoteopen}n\ {\isacharequal}{\kern0pt}\ {\isadigit{0}}{\isachardoublequoteclose}{\isacharparenright}{\kern0pt}\isanewline
\ \ \ \ \isacommand{case}\isamarkupfalse%
\ True\isanewline
\ \ \ \ \isacommand{then}\isamarkupfalse%
\ \isacommand{show}\isamarkupfalse%
\ {\isacharquery}{\kern0pt}thesis\ \isacommand{using}\isamarkupfalse%
\ hx\ power{\isadigit{0}}\ powers{\isacharunderscore}{\kern0pt}def\isanewline
\ \ \ \ \ \ \isacommand{by}\isamarkupfalse%
\ {\isacharparenleft}{\kern0pt}metis\ nat{\isacharunderscore}{\kern0pt}powers{\isacharunderscore}{\kern0pt}def\ nat{\isacharunderscore}{\kern0pt}powers{\isacharunderscore}{\kern0pt}subset\ rangeI\ subsetD{\isacharparenright}{\kern0pt}\isanewline
\ \ \isacommand{next}\isamarkupfalse%
\isanewline
\ \ \ \ \isacommand{case}\isamarkupfalse%
\ False\isanewline
\ \ \ \ \isacommand{then}\isamarkupfalse%
\ \isacommand{have}\isamarkupfalse%
\ hpos{\isacharcolon}{\kern0pt}\ {\isachardoublequoteopen}{\isasymnot}\ {\isacharparenleft}{\kern0pt}{\isacharminus}{\kern0pt}\ int\ n{\isacharparenright}{\kern0pt}\ {\isasymge}\ {\isadigit{0}}{\isachardoublequoteclose}\ \isacommand{by}\isamarkupfalse%
\ auto\isanewline
\ \ \ \ \isacommand{then}\isamarkupfalse%
\ \isacommand{have}\isamarkupfalse%
\ {\isachardoublequoteopen}x\ {\isacharequal}{\kern0pt}\ g\ powi\ {\isacharparenleft}{\kern0pt}{\isacharminus}{\kern0pt}\ int\ n{\isacharparenright}{\kern0pt}{\isachardoublequoteclose}\ \isacommand{using}\isamarkupfalse%
\ hx\ hpos\ power{\isacharunderscore}{\kern0pt}int{\isacharunderscore}{\kern0pt}def\ \isacommand{by}\isamarkupfalse%
\ simp\isanewline
\ \ \ \ \isacommand{then}\isamarkupfalse%
\ \isacommand{show}\isamarkupfalse%
\ {\isacharquery}{\kern0pt}thesis\ \isacommand{using}\isamarkupfalse%
\ powers{\isacharunderscore}{\kern0pt}def\ \isacommand{by}\isamarkupfalse%
\ auto\isanewline
\ \ \isacommand{qed}\isamarkupfalse%
\isanewline
\isacommand{qed}\isamarkupfalse%
%
\endisatagproof
{\isafoldproof}%
%
\isadelimproof
\isanewline
%
\endisadelimproof
\isanewline
\isacommand{lemma}\isamarkupfalse%
\ powers{\isacharunderscore}{\kern0pt}eq{\isacharunderscore}{\kern0pt}union{\isacharunderscore}{\kern0pt}nat{\isacharunderscore}{\kern0pt}powers{\isacharcolon}{\kern0pt}\isanewline
\ \ {\isachardoublequoteopen}powers\ g\ {\isacharequal}{\kern0pt}\ nat{\isacharunderscore}{\kern0pt}powers\ g\ {\isasymunion}\ nat{\isacharunderscore}{\kern0pt}powers\ {\isacharparenleft}{\kern0pt}inverse\ g{\isacharparenright}{\kern0pt}{\isachardoublequoteclose}\isanewline
%
\isadelimproof
%
\endisadelimproof
%
\isatagproof
\isacommand{proof}\isamarkupfalse%
\isanewline
\ \ \isacommand{show}\isamarkupfalse%
\ {\isachardoublequoteopen}powers\ g\ {\isasymsubseteq}\ nat{\isacharunderscore}{\kern0pt}powers\ g\ {\isasymunion}\ nat{\isacharunderscore}{\kern0pt}powers\ {\isacharparenleft}{\kern0pt}local{\isachardot}{\kern0pt}inverse\ g{\isacharparenright}{\kern0pt}{\isachardoublequoteclose}\isanewline
\ \ \ \ \isacommand{using}\isamarkupfalse%
\ powers{\isacharunderscore}{\kern0pt}def\ nat{\isacharunderscore}{\kern0pt}powers{\isacharunderscore}{\kern0pt}def\ power{\isacharunderscore}{\kern0pt}int{\isacharunderscore}{\kern0pt}def\ \isacommand{by}\isamarkupfalse%
\ auto\isanewline
\isacommand{next}\isamarkupfalse%
\ \isanewline
\ \ \isacommand{show}\isamarkupfalse%
\ {\isachardoublequoteopen}nat{\isacharunderscore}{\kern0pt}powers\ g\ {\isasymunion}\ nat{\isacharunderscore}{\kern0pt}powers\ {\isacharparenleft}{\kern0pt}inverse\ g{\isacharparenright}{\kern0pt}\ {\isasymsubseteq}\ powers\ g{\isachardoublequoteclose}\isanewline
\ \ \ \ \isacommand{by}\isamarkupfalse%
\ {\isacharparenleft}{\kern0pt}simp\ add{\isacharcolon}{\kern0pt}\ inverse{\isacharunderscore}{\kern0pt}nat{\isacharunderscore}{\kern0pt}powers{\isacharunderscore}{\kern0pt}subset\ nat{\isacharunderscore}{\kern0pt}powers{\isacharunderscore}{\kern0pt}subset{\isacharparenright}{\kern0pt}\isanewline
\isacommand{qed}\isamarkupfalse%
%
\endisatagproof
{\isafoldproof}%
%
\isadelimproof
\isanewline
%
\endisadelimproof
\isanewline
\isacommand{lemma}\isamarkupfalse%
\ one{\isacharunderscore}{\kern0pt}mem{\isacharunderscore}{\kern0pt}nat{\isacharunderscore}{\kern0pt}powers{\isacharcolon}{\kern0pt}\ {\isachardoublequoteopen}{\isasymone}\ {\isasymin}\ nat{\isacharunderscore}{\kern0pt}powers\ g{\isachardoublequoteclose}\isanewline
%
\isadelimproof
\ \ %
\endisadelimproof
%
\isatagproof
\isacommand{using}\isamarkupfalse%
\ rangeI\ power{\isadigit{0}}\ nat{\isacharunderscore}{\kern0pt}powers{\isacharunderscore}{\kern0pt}def\ \isacommand{by}\isamarkupfalse%
\ metis%
\endisatagproof
{\isafoldproof}%
%
\isadelimproof
\isanewline
%
\endisadelimproof
\isanewline
\isacommand{lemma}\isamarkupfalse%
\ nat{\isacharunderscore}{\kern0pt}powers{\isacharunderscore}{\kern0pt}subset{\isacharunderscore}{\kern0pt}carrier{\isacharcolon}{\kern0pt}\isanewline
\ \ \isakeyword{assumes}\ {\isachardoublequoteopen}g\ {\isasymin}\ M{\isachardoublequoteclose}\isanewline
\ \ \isakeyword{shows}\ {\isachardoublequoteopen}nat{\isacharunderscore}{\kern0pt}powers\ g\ {\isasymsubseteq}\ M{\isachardoublequoteclose}\isanewline
%
\isadelimproof
\ \ %
\endisadelimproof
%
\isatagproof
\isacommand{using}\isamarkupfalse%
\ nat{\isacharunderscore}{\kern0pt}powers{\isacharunderscore}{\kern0pt}def\ power{\isacharunderscore}{\kern0pt}mem{\isacharunderscore}{\kern0pt}carrier\ assms\ \isacommand{by}\isamarkupfalse%
\ auto%
\endisatagproof
{\isafoldproof}%
%
\isadelimproof
\isanewline
%
\endisadelimproof
\isanewline
\isacommand{lemma}\isamarkupfalse%
\ nat{\isacharunderscore}{\kern0pt}powers{\isacharunderscore}{\kern0pt}mult{\isacharunderscore}{\kern0pt}closed{\isacharcolon}{\kern0pt}\isanewline
\ \ \isakeyword{assumes}\ {\isachardoublequoteopen}g\ {\isasymin}\ M{\isachardoublequoteclose}\isanewline
\ \ \isakeyword{shows}\ {\isachardoublequoteopen}{\isasymAnd}\ x\ y{\isachardot}{\kern0pt}\ x\ {\isasymin}\ nat{\isacharunderscore}{\kern0pt}powers\ g\ {\isasymLongrightarrow}\ y\ {\isasymin}\ nat{\isacharunderscore}{\kern0pt}powers\ g\ {\isasymLongrightarrow}\ x\ {\isasymcdot}\ y\ {\isasymin}\ nat{\isacharunderscore}{\kern0pt}powers\ g{\isachardoublequoteclose}\isanewline
%
\isadelimproof
\ \ %
\endisadelimproof
%
\isatagproof
\isacommand{using}\isamarkupfalse%
\ nat{\isacharunderscore}{\kern0pt}powers{\isacharunderscore}{\kern0pt}def\ power{\isacharunderscore}{\kern0pt}mult\ assms\ \isacommand{by}\isamarkupfalse%
\ auto%
\endisatagproof
{\isafoldproof}%
%
\isadelimproof
\isanewline
%
\endisadelimproof
\isanewline
\isacommand{lemma}\isamarkupfalse%
\ nat{\isacharunderscore}{\kern0pt}powers{\isacharunderscore}{\kern0pt}inv{\isacharunderscore}{\kern0pt}mult{\isacharcolon}{\kern0pt}\isanewline
\ \ \isakeyword{assumes}\ {\isachardoublequoteopen}g\ {\isasymin}\ M{\isachardoublequoteclose}\ \isakeyword{and}\ {\isachardoublequoteopen}invertible\ g{\isachardoublequoteclose}\isanewline
\ \ \isakeyword{shows}\ {\isachardoublequoteopen}{\isasymAnd}\ x\ y{\isachardot}{\kern0pt}\ x\ {\isasymin}\ nat{\isacharunderscore}{\kern0pt}powers\ g\ {\isasymLongrightarrow}\ y\ {\isasymin}\ nat{\isacharunderscore}{\kern0pt}powers\ {\isacharparenleft}{\kern0pt}inverse\ g{\isacharparenright}{\kern0pt}\ {\isasymLongrightarrow}\ x\ {\isasymcdot}\ y\ {\isasymin}\ powers\ g{\isachardoublequoteclose}\isanewline
%
\isadelimproof
%
\endisadelimproof
%
\isatagproof
\isacommand{proof}\isamarkupfalse%
{\isacharminus}{\kern0pt}\isanewline
\ \ \isacommand{fix}\isamarkupfalse%
\ x\ y\ \isacommand{assume}\isamarkupfalse%
\ {\isachardoublequoteopen}x\ {\isasymin}\ nat{\isacharunderscore}{\kern0pt}powers\ g{\isachardoublequoteclose}\ \isakeyword{and}\ {\isachardoublequoteopen}y\ {\isasymin}\ nat{\isacharunderscore}{\kern0pt}powers\ {\isacharparenleft}{\kern0pt}inverse\ g{\isacharparenright}{\kern0pt}{\isachardoublequoteclose}\isanewline
\ \ \isacommand{then}\isamarkupfalse%
\ \isacommand{obtain}\isamarkupfalse%
\ n\ m\ \isakeyword{where}\ hx{\isacharcolon}{\kern0pt}\ {\isachardoublequoteopen}x\ {\isacharequal}{\kern0pt}\ g\ {\isacharcircum}{\kern0pt}\ n{\isachardoublequoteclose}\ \isakeyword{and}\ hy{\isacharcolon}{\kern0pt}\ {\isachardoublequoteopen}y\ {\isacharequal}{\kern0pt}\ {\isacharparenleft}{\kern0pt}inverse\ g{\isacharparenright}{\kern0pt}\ {\isacharcircum}{\kern0pt}\ m{\isachardoublequoteclose}\ \isacommand{using}\isamarkupfalse%
\ nat{\isacharunderscore}{\kern0pt}powers{\isacharunderscore}{\kern0pt}def\ \isacommand{by}\isamarkupfalse%
\ blast\isanewline
\ \ \isacommand{show}\isamarkupfalse%
\ {\isachardoublequoteopen}x\ {\isasymcdot}\ y\ {\isasymin}\ powers\ g{\isachardoublequoteclose}\isanewline
\ \ \isacommand{proof}\isamarkupfalse%
{\isacharparenleft}{\kern0pt}cases\ {\isachardoublequoteopen}n\ {\isasymge}\ m{\isachardoublequoteclose}{\isacharparenright}{\kern0pt}\isanewline
\ \ \ \ \isacommand{case}\isamarkupfalse%
\ True\isanewline
\ \ \ \ \isacommand{then}\isamarkupfalse%
\ \isacommand{obtain}\isamarkupfalse%
\ k\ \isakeyword{where}\ {\isachardoublequoteopen}n\ {\isacharequal}{\kern0pt}\ k\ {\isacharplus}{\kern0pt}\ m{\isachardoublequoteclose}\ \isacommand{using}\isamarkupfalse%
\ add{\isachardot}{\kern0pt}commute\ le{\isacharunderscore}{\kern0pt}Suc{\isacharunderscore}{\kern0pt}ex\ \isacommand{by}\isamarkupfalse%
\ blast\isanewline
\ \ \ \ \isacommand{then}\isamarkupfalse%
\ \isacommand{have}\isamarkupfalse%
\ {\isachardoublequoteopen}g\ {\isacharcircum}{\kern0pt}\ n\ {\isasymcdot}\ {\isacharparenleft}{\kern0pt}inverse\ g{\isacharparenright}{\kern0pt}\ {\isacharcircum}{\kern0pt}\ m\ {\isacharequal}{\kern0pt}\ g\ {\isacharcircum}{\kern0pt}\ k{\isachardoublequoteclose}\ \isacommand{using}\isamarkupfalse%
\ mult{\isacharunderscore}{\kern0pt}inverse{\isacharunderscore}{\kern0pt}power\ assms\ associative\ \isanewline
\ \ \ \ \ \ \isacommand{by}\isamarkupfalse%
\ {\isacharparenleft}{\kern0pt}smt\ {\isacharparenleft}{\kern0pt}verit{\isacharparenright}{\kern0pt}\ invertible{\isacharunderscore}{\kern0pt}inverse{\isacharunderscore}{\kern0pt}closed\ local{\isachardot}{\kern0pt}power{\isacharunderscore}{\kern0pt}mult\ power{\isacharunderscore}{\kern0pt}mem{\isacharunderscore}{\kern0pt}carrier\ right{\isacharunderscore}{\kern0pt}unit{\isacharparenright}{\kern0pt}\isanewline
\ \ \ \ \isacommand{then}\isamarkupfalse%
\ \isacommand{show}\isamarkupfalse%
\ {\isacharquery}{\kern0pt}thesis\ \isacommand{using}\isamarkupfalse%
\ hx\ hy\ powers{\isacharunderscore}{\kern0pt}eq{\isacharunderscore}{\kern0pt}union{\isacharunderscore}{\kern0pt}nat{\isacharunderscore}{\kern0pt}powers\ nat{\isacharunderscore}{\kern0pt}powers{\isacharunderscore}{\kern0pt}def\ \isacommand{by}\isamarkupfalse%
\ auto\isanewline
\ \ \isacommand{next}\isamarkupfalse%
\isanewline
\ \ \ \ \isacommand{case}\isamarkupfalse%
\ False\isanewline
\ \ \ \ \isacommand{then}\isamarkupfalse%
\ \isacommand{obtain}\isamarkupfalse%
\ k\ \isakeyword{where}\ {\isachardoublequoteopen}m\ {\isacharequal}{\kern0pt}\ n\ {\isacharplus}{\kern0pt}\ k{\isachardoublequoteclose}\ \isacommand{by}\isamarkupfalse%
\ {\isacharparenleft}{\kern0pt}metis\ leI\ less{\isacharunderscore}{\kern0pt}imp{\isacharunderscore}{\kern0pt}add{\isacharunderscore}{\kern0pt}positive{\isacharparenright}{\kern0pt}\isanewline
\ \ \ \ \isacommand{then}\isamarkupfalse%
\ \isacommand{have}\isamarkupfalse%
\ {\isachardoublequoteopen}g\ {\isacharcircum}{\kern0pt}\ n\ {\isasymcdot}\ {\isacharparenleft}{\kern0pt}inverse\ g{\isacharparenright}{\kern0pt}\ {\isacharcircum}{\kern0pt}\ m\ {\isacharequal}{\kern0pt}\ {\isacharparenleft}{\kern0pt}inverse\ g{\isacharparenright}{\kern0pt}\ {\isacharcircum}{\kern0pt}\ k{\isachardoublequoteclose}\ \isacommand{using}\isamarkupfalse%
\ inverse{\isacharunderscore}{\kern0pt}mult{\isacharunderscore}{\kern0pt}power\ assms\ associative\ \isanewline
\ \ \ \ \ \ \isacommand{by}\isamarkupfalse%
\ {\isacharparenleft}{\kern0pt}smt\ {\isacharparenleft}{\kern0pt}verit{\isacharparenright}{\kern0pt}\ left{\isacharunderscore}{\kern0pt}unit\ local{\isachardot}{\kern0pt}power{\isacharunderscore}{\kern0pt}mult\ monoid{\isachardot}{\kern0pt}invertible{\isacharunderscore}{\kern0pt}inverse{\isacharunderscore}{\kern0pt}closed\ monoid{\isacharunderscore}{\kern0pt}axioms\ \isanewline
\ \ \ \ \ \ \ \ mult{\isacharunderscore}{\kern0pt}inverse{\isacharunderscore}{\kern0pt}power\ power{\isacharunderscore}{\kern0pt}mem{\isacharunderscore}{\kern0pt}carrier{\isacharparenright}{\kern0pt}\isanewline
\ \ \ \ \isacommand{then}\isamarkupfalse%
\ \isacommand{show}\isamarkupfalse%
\ {\isacharquery}{\kern0pt}thesis\ \isacommand{using}\isamarkupfalse%
\ hx\ hy\ powers{\isacharunderscore}{\kern0pt}eq{\isacharunderscore}{\kern0pt}union{\isacharunderscore}{\kern0pt}nat{\isacharunderscore}{\kern0pt}powers\ nat{\isacharunderscore}{\kern0pt}powers{\isacharunderscore}{\kern0pt}def\ \isacommand{by}\isamarkupfalse%
\ auto\isanewline
\ \ \isacommand{qed}\isamarkupfalse%
\isanewline
\isacommand{qed}\isamarkupfalse%
%
\endisatagproof
{\isafoldproof}%
%
\isadelimproof
\isanewline
%
\endisadelimproof
\isanewline
\isacommand{lemma}\isamarkupfalse%
\ inv{\isacharunderscore}{\kern0pt}nat{\isacharunderscore}{\kern0pt}powers{\isacharunderscore}{\kern0pt}mult{\isacharcolon}{\kern0pt}\isanewline
\ \ \isakeyword{assumes}\ {\isachardoublequoteopen}g\ {\isasymin}\ M{\isachardoublequoteclose}\ \isakeyword{and}\ {\isachardoublequoteopen}invertible\ g{\isachardoublequoteclose}\isanewline
\ \ \isakeyword{shows}\ {\isachardoublequoteopen}{\isasymAnd}\ x\ y{\isachardot}{\kern0pt}\ x\ {\isasymin}\ nat{\isacharunderscore}{\kern0pt}powers\ {\isacharparenleft}{\kern0pt}inverse\ g{\isacharparenright}{\kern0pt}\ {\isasymLongrightarrow}\ y\ {\isasymin}\ nat{\isacharunderscore}{\kern0pt}powers\ g\ {\isasymLongrightarrow}\ x\ {\isasymcdot}\ y\ {\isasymin}\ powers\ g{\isachardoublequoteclose}\isanewline
%
\isadelimproof
\ \ %
\endisadelimproof
%
\isatagproof
\isacommand{by}\isamarkupfalse%
\ {\isacharparenleft}{\kern0pt}metis\ assms\ invertible{\isacharunderscore}{\kern0pt}inverse{\isacharunderscore}{\kern0pt}closed\ invertible{\isacharunderscore}{\kern0pt}inverse{\isacharunderscore}{\kern0pt}inverse\ invertible{\isacharunderscore}{\kern0pt}inverse{\isacharunderscore}{\kern0pt}invertible\isanewline
\ \ \ \ nat{\isacharunderscore}{\kern0pt}powers{\isacharunderscore}{\kern0pt}inv{\isacharunderscore}{\kern0pt}mult\ powers{\isacharunderscore}{\kern0pt}eq{\isacharunderscore}{\kern0pt}union{\isacharunderscore}{\kern0pt}nat{\isacharunderscore}{\kern0pt}powers\ sup{\isacharunderscore}{\kern0pt}commute{\isacharparenright}{\kern0pt}%
\endisatagproof
{\isafoldproof}%
%
\isadelimproof
\isanewline
%
\endisadelimproof
\isanewline
\isacommand{lemma}\isamarkupfalse%
\ powers{\isacharunderscore}{\kern0pt}mult{\isacharunderscore}{\kern0pt}closed{\isacharcolon}{\kern0pt}\isanewline
\ \ \isakeyword{assumes}\ {\isachardoublequoteopen}g\ {\isasymin}\ M{\isachardoublequoteclose}\ \isakeyword{and}\ {\isachardoublequoteopen}invertible\ g{\isachardoublequoteclose}\isanewline
\ \ \isakeyword{shows}\ {\isachardoublequoteopen}{\isasymAnd}\ x\ y{\isachardot}{\kern0pt}\ x\ {\isasymin}\ powers\ g\ {\isasymLongrightarrow}\ y\ {\isasymin}\ powers\ g\ {\isasymLongrightarrow}\ x\ {\isasymcdot}\ y\ {\isasymin}\ powers\ g{\isachardoublequoteclose}\isanewline
%
\isadelimproof
\ \ %
\endisadelimproof
%
\isatagproof
\isacommand{using}\isamarkupfalse%
\ powers{\isacharunderscore}{\kern0pt}eq{\isacharunderscore}{\kern0pt}union{\isacharunderscore}{\kern0pt}nat{\isacharunderscore}{\kern0pt}powers\ assms\ \isanewline
\ \ \ \ nat{\isacharunderscore}{\kern0pt}powers{\isacharunderscore}{\kern0pt}mult{\isacharunderscore}{\kern0pt}closed\ nat{\isacharunderscore}{\kern0pt}powers{\isacharunderscore}{\kern0pt}inv{\isacharunderscore}{\kern0pt}mult\ inv{\isacharunderscore}{\kern0pt}nat{\isacharunderscore}{\kern0pt}powers{\isacharunderscore}{\kern0pt}mult\ \isacommand{by}\isamarkupfalse%
\ fastforce%
\endisatagproof
{\isafoldproof}%
%
\isadelimproof
\isanewline
%
\endisadelimproof
\isanewline
\isacommand{lemma}\isamarkupfalse%
\ nat{\isacharunderscore}{\kern0pt}powers{\isacharunderscore}{\kern0pt}submonoid{\isacharcolon}{\kern0pt}\isanewline
\ \ \isakeyword{assumes}\ {\isachardoublequoteopen}g\ {\isasymin}\ M{\isachardoublequoteclose}\isanewline
\ \ \isakeyword{shows}\ {\isachardoublequoteopen}submonoid\ {\isacharparenleft}{\kern0pt}nat{\isacharunderscore}{\kern0pt}powers\ g{\isacharparenright}{\kern0pt}\ M\ {\isacharparenleft}{\kern0pt}{\isasymcdot}{\isacharparenright}{\kern0pt}\ {\isasymone}{\isachardoublequoteclose}\isanewline
%
\isadelimproof
\ \ %
\endisadelimproof
%
\isatagproof
\isacommand{apply}\isamarkupfalse%
{\isacharparenleft}{\kern0pt}unfold{\isacharunderscore}{\kern0pt}locales{\isacharparenright}{\kern0pt}\isanewline
\ \ \isacommand{apply}\isamarkupfalse%
{\isacharparenleft}{\kern0pt}auto\ simp\ add{\isacharcolon}{\kern0pt}\ assms\ nat{\isacharunderscore}{\kern0pt}powers{\isacharunderscore}{\kern0pt}mult{\isacharunderscore}{\kern0pt}closed\ one{\isacharunderscore}{\kern0pt}mem{\isacharunderscore}{\kern0pt}nat{\isacharunderscore}{\kern0pt}powers\ nat{\isacharunderscore}{\kern0pt}powers{\isacharunderscore}{\kern0pt}subset{\isacharunderscore}{\kern0pt}carrier{\isacharparenright}{\kern0pt}\isanewline
\ \ \isacommand{done}\isamarkupfalse%
%
\endisatagproof
{\isafoldproof}%
%
\isadelimproof
\isanewline
%
\endisadelimproof
\isanewline
\isacommand{lemma}\isamarkupfalse%
\ nat{\isacharunderscore}{\kern0pt}powers{\isacharunderscore}{\kern0pt}monoid{\isacharcolon}{\kern0pt}\isanewline
\ \ \isakeyword{assumes}\ {\isachardoublequoteopen}g\ {\isasymin}\ M{\isachardoublequoteclose}\isanewline
\ \ \isakeyword{shows}\ {\isachardoublequoteopen}monoid\ {\isacharparenleft}{\kern0pt}nat{\isacharunderscore}{\kern0pt}powers\ g{\isacharparenright}{\kern0pt}\ {\isacharparenleft}{\kern0pt}{\isasymcdot}{\isacharparenright}{\kern0pt}\ {\isasymone}{\isachardoublequoteclose}\isanewline
%
\isadelimproof
\ \ %
\endisadelimproof
%
\isatagproof
\isacommand{using}\isamarkupfalse%
\ nat{\isacharunderscore}{\kern0pt}powers{\isacharunderscore}{\kern0pt}submonoid\ assms\ \isacommand{by}\isamarkupfalse%
\ {\isacharparenleft}{\kern0pt}smt\ {\isacharparenleft}{\kern0pt}verit{\isacharparenright}{\kern0pt}\ monoid{\isachardot}{\kern0pt}intro\ associative\ left{\isacharunderscore}{\kern0pt}unit\ \isanewline
\ \ \ \ \ \ one{\isacharunderscore}{\kern0pt}mem{\isacharunderscore}{\kern0pt}nat{\isacharunderscore}{\kern0pt}powers\ nat{\isacharunderscore}{\kern0pt}powers{\isacharunderscore}{\kern0pt}mult{\isacharunderscore}{\kern0pt}closed\ right{\isacharunderscore}{\kern0pt}unit\ submonoid{\isachardot}{\kern0pt}sub{\isacharparenright}{\kern0pt}%
\endisatagproof
{\isafoldproof}%
%
\isadelimproof
\isanewline
%
\endisadelimproof
\isanewline
\isacommand{lemma}\isamarkupfalse%
\ powers{\isacharunderscore}{\kern0pt}submonoid{\isacharcolon}{\kern0pt}\isanewline
\ \ \isakeyword{assumes}\ {\isachardoublequoteopen}g\ {\isasymin}\ M{\isachardoublequoteclose}\ \isakeyword{and}\ {\isachardoublequoteopen}invertible\ g{\isachardoublequoteclose}\isanewline
\ \ \isakeyword{shows}\ {\isachardoublequoteopen}submonoid\ {\isacharparenleft}{\kern0pt}powers\ g{\isacharparenright}{\kern0pt}\ M\ {\isacharparenleft}{\kern0pt}{\isasymcdot}{\isacharparenright}{\kern0pt}\ {\isasymone}{\isachardoublequoteclose}\isanewline
%
\isadelimproof
%
\endisadelimproof
%
\isatagproof
\isacommand{proof}\isamarkupfalse%
\isanewline
\ \ \isacommand{show}\isamarkupfalse%
\ {\isachardoublequoteopen}powers\ g\ {\isasymsubseteq}\ M{\isachardoublequoteclose}\ \isacommand{using}\isamarkupfalse%
\ powers{\isacharunderscore}{\kern0pt}eq{\isacharunderscore}{\kern0pt}union{\isacharunderscore}{\kern0pt}nat{\isacharunderscore}{\kern0pt}powers\ nat{\isacharunderscore}{\kern0pt}powers{\isacharunderscore}{\kern0pt}subset{\isacharunderscore}{\kern0pt}carrier\ assms\ \isacommand{by}\isamarkupfalse%
\ auto\isanewline
\isacommand{next}\isamarkupfalse%
\isanewline
\ \ \isacommand{show}\isamarkupfalse%
\ {\isachardoublequoteopen}{\isasymAnd}a\ b{\isachardot}{\kern0pt}\ a\ {\isasymin}\ powers\ g\ {\isasymLongrightarrow}\ b\ {\isasymin}\ powers\ g\ {\isasymLongrightarrow}\ a\ {\isasymcdot}\ b\ {\isasymin}\ powers\ g{\isachardoublequoteclose}\ \isanewline
\ \ \ \ \isacommand{using}\isamarkupfalse%
\ powers{\isacharunderscore}{\kern0pt}mult{\isacharunderscore}{\kern0pt}closed\ assms\ \isacommand{by}\isamarkupfalse%
\ auto\isanewline
\isacommand{next}\isamarkupfalse%
\isanewline
\ \ \isacommand{show}\isamarkupfalse%
\ {\isachardoublequoteopen}{\isasymone}\ {\isasymin}\ powers\ g{\isachardoublequoteclose}\ \isacommand{using}\isamarkupfalse%
\ powers{\isacharunderscore}{\kern0pt}eq{\isacharunderscore}{\kern0pt}union{\isacharunderscore}{\kern0pt}nat{\isacharunderscore}{\kern0pt}powers\ one{\isacharunderscore}{\kern0pt}mem{\isacharunderscore}{\kern0pt}nat{\isacharunderscore}{\kern0pt}powers\ \isacommand{by}\isamarkupfalse%
\ auto\isanewline
\isacommand{qed}\isamarkupfalse%
%
\endisatagproof
{\isafoldproof}%
%
\isadelimproof
\isanewline
%
\endisadelimproof
\isanewline
\isacommand{lemma}\isamarkupfalse%
\ powers{\isacharunderscore}{\kern0pt}monoid{\isacharcolon}{\kern0pt}\isanewline
\ \ \isakeyword{assumes}\ {\isachardoublequoteopen}g\ {\isasymin}\ M{\isachardoublequoteclose}\ \isakeyword{and}\ {\isachardoublequoteopen}invertible\ g{\isachardoublequoteclose}\isanewline
\ \ \isakeyword{shows}\ {\isachardoublequoteopen}monoid\ {\isacharparenleft}{\kern0pt}powers\ g{\isacharparenright}{\kern0pt}\ {\isacharparenleft}{\kern0pt}{\isasymcdot}{\isacharparenright}{\kern0pt}\ {\isasymone}{\isachardoublequoteclose}\isanewline
%
\isadelimproof
\ \ %
\endisadelimproof
%
\isatagproof
\isacommand{by}\isamarkupfalse%
\ {\isacharparenleft}{\kern0pt}smt\ {\isacharparenleft}{\kern0pt}verit{\isacharparenright}{\kern0pt}\ monoid{\isachardot}{\kern0pt}intro\ Un{\isacharunderscore}{\kern0pt}iff\ assms\ associative\ in{\isacharunderscore}{\kern0pt}mono\ invertible{\isacharunderscore}{\kern0pt}inverse{\isacharunderscore}{\kern0pt}closed\ \isanewline
\ \ \ \ \ \ monoid{\isachardot}{\kern0pt}left{\isacharunderscore}{\kern0pt}unit\ monoid{\isachardot}{\kern0pt}right{\isacharunderscore}{\kern0pt}unit\ nat{\isacharunderscore}{\kern0pt}powers{\isacharunderscore}{\kern0pt}monoid\ powers{\isacharunderscore}{\kern0pt}eq{\isacharunderscore}{\kern0pt}union{\isacharunderscore}{\kern0pt}nat{\isacharunderscore}{\kern0pt}powers\ \isanewline
\ \ \ \ \ \ powers{\isacharunderscore}{\kern0pt}mult{\isacharunderscore}{\kern0pt}closed\ powers{\isacharunderscore}{\kern0pt}submonoid\ submonoid{\isachardot}{\kern0pt}sub{\isacharunderscore}{\kern0pt}unit{\isacharunderscore}{\kern0pt}closed\ submonoid{\isachardot}{\kern0pt}subset{\isacharparenright}{\kern0pt}%
\endisatagproof
{\isafoldproof}%
%
\isadelimproof
\isanewline
%
\endisadelimproof
\isanewline
\isacommand{lemma}\isamarkupfalse%
\ mem{\isacharunderscore}{\kern0pt}nat{\isacharunderscore}{\kern0pt}powers{\isacharunderscore}{\kern0pt}invertible{\isacharcolon}{\kern0pt}\isanewline
\ \ \isakeyword{assumes}\ {\isachardoublequoteopen}g\ {\isasymin}\ M{\isachardoublequoteclose}\ \isakeyword{and}\ {\isachardoublequoteopen}invertible\ g{\isachardoublequoteclose}\ \isakeyword{and}\ {\isachardoublequoteopen}u\ {\isasymin}\ nat{\isacharunderscore}{\kern0pt}powers\ g{\isachardoublequoteclose}\isanewline
\ \ \isakeyword{shows}\ {\isachardoublequoteopen}monoid{\isachardot}{\kern0pt}invertible\ {\isacharparenleft}{\kern0pt}powers\ g{\isacharparenright}{\kern0pt}\ {\isacharparenleft}{\kern0pt}{\isasymcdot}{\isacharparenright}{\kern0pt}\ {\isasymone}\ u{\isachardoublequoteclose}\isanewline
%
\isadelimproof
%
\endisadelimproof
%
\isatagproof
\isacommand{proof}\isamarkupfalse%
{\isacharminus}{\kern0pt}\isanewline
\ \ \isacommand{obtain}\isamarkupfalse%
\ n\ \isakeyword{where}\ hu{\isacharcolon}{\kern0pt}\ {\isachardoublequoteopen}u\ {\isacharequal}{\kern0pt}\ g\ {\isacharcircum}{\kern0pt}\ n{\isachardoublequoteclose}\ \isacommand{using}\isamarkupfalse%
\ assms\ nat{\isacharunderscore}{\kern0pt}powers{\isacharunderscore}{\kern0pt}def\ \isacommand{by}\isamarkupfalse%
\ blast\isanewline
\ \ \isacommand{then}\isamarkupfalse%
\ \isacommand{have}\isamarkupfalse%
\ {\isachardoublequoteopen}inverse\ u\ {\isasymin}\ powers\ g{\isachardoublequoteclose}\ \isacommand{using}\isamarkupfalse%
\ assms\ inverse{\isacharunderscore}{\kern0pt}mult{\isacharunderscore}{\kern0pt}power{\isacharunderscore}{\kern0pt}eq\ \isanewline
\ \ \ \ \ \ powers{\isacharunderscore}{\kern0pt}eq{\isacharunderscore}{\kern0pt}union{\isacharunderscore}{\kern0pt}nat{\isacharunderscore}{\kern0pt}powers\ nat{\isacharunderscore}{\kern0pt}powers{\isacharunderscore}{\kern0pt}def\ \isacommand{by}\isamarkupfalse%
\ auto\isanewline
\ \ \isacommand{then}\isamarkupfalse%
\ \isacommand{show}\isamarkupfalse%
\ {\isacharquery}{\kern0pt}thesis\ \isacommand{using}\isamarkupfalse%
\ hu\ assms\ \isacommand{by}\isamarkupfalse%
\ {\isacharparenleft}{\kern0pt}metis\ in{\isacharunderscore}{\kern0pt}mono\ inverse{\isacharunderscore}{\kern0pt}mult{\isacharunderscore}{\kern0pt}power\ inverse{\isacharunderscore}{\kern0pt}mult{\isacharunderscore}{\kern0pt}power{\isacharunderscore}{\kern0pt}eq\ \isanewline
\ \ \ \ monoid{\isachardot}{\kern0pt}invertibleI\ monoid{\isachardot}{\kern0pt}nat{\isacharunderscore}{\kern0pt}powers{\isacharunderscore}{\kern0pt}subset\ monoid{\isachardot}{\kern0pt}powers{\isacharunderscore}{\kern0pt}monoid\ monoid{\isacharunderscore}{\kern0pt}axioms\ mult{\isacharunderscore}{\kern0pt}inverse{\isacharunderscore}{\kern0pt}power{\isacharparenright}{\kern0pt}\isanewline
\isacommand{qed}\isamarkupfalse%
%
\endisatagproof
{\isafoldproof}%
%
\isadelimproof
\isanewline
%
\endisadelimproof
\isanewline
\isacommand{lemma}\isamarkupfalse%
\ mem{\isacharunderscore}{\kern0pt}nat{\isacharunderscore}{\kern0pt}inv{\isacharunderscore}{\kern0pt}powers{\isacharunderscore}{\kern0pt}invertible{\isacharcolon}{\kern0pt}\isanewline
\ \ \isakeyword{assumes}\ {\isachardoublequoteopen}g\ {\isasymin}\ M{\isachardoublequoteclose}\ \isakeyword{and}\ {\isachardoublequoteopen}invertible\ g{\isachardoublequoteclose}\ \isakeyword{and}\ {\isachardoublequoteopen}u\ {\isasymin}\ nat{\isacharunderscore}{\kern0pt}powers\ {\isacharparenleft}{\kern0pt}inverse\ g{\isacharparenright}{\kern0pt}{\isachardoublequoteclose}\isanewline
\ \ \isakeyword{shows}\ {\isachardoublequoteopen}monoid{\isachardot}{\kern0pt}invertible\ {\isacharparenleft}{\kern0pt}powers\ g{\isacharparenright}{\kern0pt}\ {\isacharparenleft}{\kern0pt}{\isasymcdot}{\isacharparenright}{\kern0pt}\ {\isasymone}\ u{\isachardoublequoteclose}\isanewline
%
\isadelimproof
\ \ %
\endisadelimproof
%
\isatagproof
\isacommand{using}\isamarkupfalse%
\ assms\ \isacommand{by}\isamarkupfalse%
\ {\isacharparenleft}{\kern0pt}metis\ inf{\isacharunderscore}{\kern0pt}sup{\isacharunderscore}{\kern0pt}aci{\isacharparenleft}{\kern0pt}{\isadigit{5}}{\isacharparenright}{\kern0pt}\ invertible{\isacharunderscore}{\kern0pt}inverse{\isacharunderscore}{\kern0pt}closed\ invertible{\isacharunderscore}{\kern0pt}inverse{\isacharunderscore}{\kern0pt}inverse\ \isanewline
\ \ \ \ invertible{\isacharunderscore}{\kern0pt}inverse{\isacharunderscore}{\kern0pt}invertible\ mem{\isacharunderscore}{\kern0pt}nat{\isacharunderscore}{\kern0pt}powers{\isacharunderscore}{\kern0pt}invertible\ powers{\isacharunderscore}{\kern0pt}eq{\isacharunderscore}{\kern0pt}union{\isacharunderscore}{\kern0pt}nat{\isacharunderscore}{\kern0pt}powers{\isacharparenright}{\kern0pt}%
\endisatagproof
{\isafoldproof}%
%
\isadelimproof
\isanewline
%
\endisadelimproof
\isanewline
\isacommand{lemma}\isamarkupfalse%
\ powers{\isacharunderscore}{\kern0pt}group{\isacharcolon}{\kern0pt}\isanewline
\ \ \isakeyword{assumes}\ {\isachardoublequoteopen}g\ {\isasymin}\ M{\isachardoublequoteclose}\ \isakeyword{and}\ {\isachardoublequoteopen}invertible\ g{\isachardoublequoteclose}\isanewline
\ \ \isakeyword{shows}\ {\isachardoublequoteopen}group\ {\isacharparenleft}{\kern0pt}powers\ g{\isacharparenright}{\kern0pt}\ {\isacharparenleft}{\kern0pt}{\isasymcdot}{\isacharparenright}{\kern0pt}\ {\isasymone}{\isachardoublequoteclose}\isanewline
%
\isadelimproof
%
\endisadelimproof
%
\isatagproof
\isacommand{proof}\isamarkupfalse%
{\isacharparenleft}{\kern0pt}auto\ simp\ add{\isacharcolon}{\kern0pt}\ group{\isacharunderscore}{\kern0pt}def\ Group{\isacharunderscore}{\kern0pt}Theory{\isachardot}{\kern0pt}group{\isacharunderscore}{\kern0pt}axioms{\isacharunderscore}{\kern0pt}def\ assms\ powers{\isacharunderscore}{\kern0pt}monoid{\isacharparenright}{\kern0pt}\isanewline
\ \ \isacommand{show}\isamarkupfalse%
\ {\isachardoublequoteopen}{\isasymAnd}u{\isachardot}{\kern0pt}\ u\ {\isasymin}\ powers\ g\ {\isasymLongrightarrow}\ monoid{\isachardot}{\kern0pt}invertible\ {\isacharparenleft}{\kern0pt}powers\ g{\isacharparenright}{\kern0pt}\ {\isacharparenleft}{\kern0pt}{\isasymcdot}{\isacharparenright}{\kern0pt}\ {\isasymone}\ u{\isachardoublequoteclose}\ \isacommand{using}\isamarkupfalse%
\ assms\ \isanewline
\ \ \ \ mem{\isacharunderscore}{\kern0pt}nat{\isacharunderscore}{\kern0pt}inv{\isacharunderscore}{\kern0pt}powers{\isacharunderscore}{\kern0pt}invertible\ mem{\isacharunderscore}{\kern0pt}nat{\isacharunderscore}{\kern0pt}powers{\isacharunderscore}{\kern0pt}invertible\ powers{\isacharunderscore}{\kern0pt}eq{\isacharunderscore}{\kern0pt}union{\isacharunderscore}{\kern0pt}nat{\isacharunderscore}{\kern0pt}powers\ \isacommand{by}\isamarkupfalse%
\ auto\isanewline
\isacommand{qed}\isamarkupfalse%
%
\endisatagproof
{\isafoldproof}%
%
\isadelimproof
\isanewline
%
\endisadelimproof
\isanewline
\isacommand{lemma}\isamarkupfalse%
\ nat{\isacharunderscore}{\kern0pt}powers{\isacharunderscore}{\kern0pt}ne{\isacharunderscore}{\kern0pt}one{\isacharcolon}{\kern0pt}\isanewline
\ \ \isakeyword{assumes}\ {\isachardoublequoteopen}g\ {\isasymin}\ M{\isachardoublequoteclose}\ \isakeyword{and}\ {\isachardoublequoteopen}g\ {\isasymnoteq}\ {\isasymone}{\isachardoublequoteclose}\isanewline
\ \ \isakeyword{shows}\ {\isachardoublequoteopen}nat{\isacharunderscore}{\kern0pt}powers\ g\ {\isasymnoteq}\ {\isacharbraceleft}{\kern0pt}{\isasymone}{\isacharbraceright}{\kern0pt}{\isachardoublequoteclose}\isanewline
%
\isadelimproof
%
\endisadelimproof
%
\isatagproof
\isacommand{proof}\isamarkupfalse%
{\isacharminus}{\kern0pt}\isanewline
\ \ \isacommand{have}\isamarkupfalse%
\ {\isachardoublequoteopen}g\ {\isasymin}\ nat{\isacharunderscore}{\kern0pt}powers\ g{\isachardoublequoteclose}\ \isacommand{using}\isamarkupfalse%
\ power{\isacharunderscore}{\kern0pt}one\ nat{\isacharunderscore}{\kern0pt}powers{\isacharunderscore}{\kern0pt}def\ assms\ rangeI\ \isacommand{by}\isamarkupfalse%
\ metis\isanewline
\ \ \isacommand{then}\isamarkupfalse%
\ \isacommand{show}\isamarkupfalse%
\ {\isacharquery}{\kern0pt}thesis\ \isacommand{using}\isamarkupfalse%
\ assms\ \isacommand{by}\isamarkupfalse%
\ blast\isanewline
\isacommand{qed}\isamarkupfalse%
%
\endisatagproof
{\isafoldproof}%
%
\isadelimproof
\isanewline
%
\endisadelimproof
\isanewline
\isacommand{lemma}\isamarkupfalse%
\ powers{\isacharunderscore}{\kern0pt}ne{\isacharunderscore}{\kern0pt}one{\isacharcolon}{\kern0pt}\ \isanewline
\ \ \isakeyword{assumes}\ {\isachardoublequoteopen}g\ {\isasymin}\ M{\isachardoublequoteclose}\ \isakeyword{and}\ {\isachardoublequoteopen}g\ {\isasymnoteq}\ {\isasymone}{\isachardoublequoteclose}\isanewline
\ \ \isakeyword{shows}\ {\isachardoublequoteopen}powers\ g\ {\isasymnoteq}\ {\isacharbraceleft}{\kern0pt}{\isasymone}{\isacharbraceright}{\kern0pt}{\isachardoublequoteclose}%
\isadelimproof
\ %
\endisadelimproof
%
\isatagproof
\isacommand{using}\isamarkupfalse%
\ assms\ nat{\isacharunderscore}{\kern0pt}powers{\isacharunderscore}{\kern0pt}ne{\isacharunderscore}{\kern0pt}one\ \isanewline
\ \ \isacommand{by}\isamarkupfalse%
\ {\isacharparenleft}{\kern0pt}metis\ all{\isacharunderscore}{\kern0pt}not{\isacharunderscore}{\kern0pt}in{\isacharunderscore}{\kern0pt}conv\ nat{\isacharunderscore}{\kern0pt}powers{\isacharunderscore}{\kern0pt}subset\ one{\isacharunderscore}{\kern0pt}mem{\isacharunderscore}{\kern0pt}nat{\isacharunderscore}{\kern0pt}powers\ subset{\isacharunderscore}{\kern0pt}singleton{\isacharunderscore}{\kern0pt}iff{\isacharparenright}{\kern0pt}%
\endisatagproof
{\isafoldproof}%
%
\isadelimproof
%
\endisadelimproof
%
\isadelimdocument
%
\endisadelimdocument
%
\isatagdocument
%
\isamarkupsubsection{Definition of the order of an element in a group%
}
\isamarkuptrue%
%
\endisatagdocument
{\isafolddocument}%
%
\isadelimdocument
%
\endisadelimdocument
\isacommand{definition}\isamarkupfalse%
\ order\ \isanewline
\ \ \isakeyword{where}\ {\isachardoublequoteopen}order\ g\ {\isacharequal}{\kern0pt}\ {\isacharparenleft}{\kern0pt}if\ {\isacharparenleft}{\kern0pt}{\isasymexists}\ n{\isachardot}{\kern0pt}\ n\ {\isachargreater}{\kern0pt}\ {\isadigit{0}}\ {\isasymand}\ g\ {\isacharcircum}{\kern0pt}\ n\ {\isacharequal}{\kern0pt}\ {\isasymone}{\isacharparenright}{\kern0pt}\ then\ Min\ {\isacharbraceleft}{\kern0pt}n{\isachardot}{\kern0pt}\ g\ {\isacharcircum}{\kern0pt}\ n\ {\isacharequal}{\kern0pt}\ {\isasymone}\ {\isasymand}\ n\ {\isachargreater}{\kern0pt}\ {\isadigit{0}}{\isacharbraceright}{\kern0pt}\ else\ {\isadigit{0}}{\isacharparenright}{\kern0pt}{\isachardoublequoteclose}\isanewline
\isanewline
\isacommand{definition}\isamarkupfalse%
\ min{\isacharunderscore}{\kern0pt}order\ \isakeyword{where}\ {\isachardoublequoteopen}min{\isacharunderscore}{\kern0pt}order\ {\isacharequal}{\kern0pt}\ Min\ {\isacharparenleft}{\kern0pt}{\isacharparenleft}{\kern0pt}order\ {\isacharbackquote}{\kern0pt}\ M{\isacharparenright}{\kern0pt}\ {\isacharminus}{\kern0pt}\ {\isacharbraceleft}{\kern0pt}{\isadigit{0}}{\isacharbraceright}{\kern0pt}{\isacharparenright}{\kern0pt}{\isachardoublequoteclose}\isanewline
\isanewline
\isacommand{end}\isamarkupfalse%
%
\isadelimdocument
%
\endisadelimdocument
%
\isatagdocument
%
\isamarkupsubsection{Sumset scalar multiplication cardinality lemmas%
}
\isamarkuptrue%
%
\endisatagdocument
{\isafolddocument}%
%
\isadelimdocument
%
\endisadelimdocument
\isacommand{context}\isamarkupfalse%
\ group\isanewline
\isanewline
\isakeyword{begin}\isanewline
\isanewline
\isacommand{lemma}\isamarkupfalse%
\ card{\isacharunderscore}{\kern0pt}smul{\isacharunderscore}{\kern0pt}singleton{\isacharunderscore}{\kern0pt}right{\isacharunderscore}{\kern0pt}eq{\isacharcolon}{\kern0pt}\isanewline
\ \ \isakeyword{assumes}\ {\isachardoublequoteopen}finite\ A{\isachardoublequoteclose}\ \isakeyword{shows}\ {\isachardoublequoteopen}card\ {\isacharparenleft}{\kern0pt}smul\ A\ {\isacharbraceleft}{\kern0pt}a{\isacharbraceright}{\kern0pt}{\isacharparenright}{\kern0pt}\ {\isacharequal}{\kern0pt}\ {\isacharparenleft}{\kern0pt}if\ a\ {\isasymin}\ G\ then\ card\ {\isacharparenleft}{\kern0pt}A\ {\isasyminter}\ G{\isacharparenright}{\kern0pt}\ else\ {\isadigit{0}}{\isacharparenright}{\kern0pt}{\isachardoublequoteclose}\isanewline
%
\isadelimproof
%
\endisadelimproof
%
\isatagproof
\isacommand{proof}\isamarkupfalse%
\ {\isacharparenleft}{\kern0pt}cases\ {\isachardoublequoteopen}a\ {\isasymin}\ G{\isachardoublequoteclose}{\isacharparenright}{\kern0pt}\isanewline
\ \ \isacommand{case}\isamarkupfalse%
\ True\isanewline
\ \ \isacommand{then}\isamarkupfalse%
\ \isacommand{have}\isamarkupfalse%
\ {\isachardoublequoteopen}smul\ A\ {\isacharbraceleft}{\kern0pt}a{\isacharbraceright}{\kern0pt}\ {\isacharequal}{\kern0pt}\ {\isacharparenleft}{\kern0pt}{\isasymlambda}x{\isachardot}{\kern0pt}\ x\ {\isasymcdot}\ a{\isacharparenright}{\kern0pt}\ {\isacharbackquote}{\kern0pt}\ {\isacharparenleft}{\kern0pt}A\ {\isasyminter}\ G{\isacharparenright}{\kern0pt}{\isachardoublequoteclose}\isanewline
\ \ \ \ \isacommand{by}\isamarkupfalse%
\ {\isacharparenleft}{\kern0pt}auto\ simp{\isacharcolon}{\kern0pt}\ smul{\isacharunderscore}{\kern0pt}eq{\isacharparenright}{\kern0pt}\isanewline
\ \ \isacommand{moreover}\isamarkupfalse%
\ \isacommand{have}\isamarkupfalse%
\ {\isachardoublequoteopen}inj{\isacharunderscore}{\kern0pt}on\ {\isacharparenleft}{\kern0pt}{\isasymlambda}x{\isachardot}{\kern0pt}\ x\ {\isasymcdot}\ a{\isacharparenright}{\kern0pt}\ {\isacharparenleft}{\kern0pt}A\ {\isasyminter}\ G{\isacharparenright}{\kern0pt}{\isachardoublequoteclose}\isanewline
\ \ \ \ \isacommand{by}\isamarkupfalse%
\ {\isacharparenleft}{\kern0pt}auto\ simp{\isacharcolon}{\kern0pt}\ inj{\isacharunderscore}{\kern0pt}on{\isacharunderscore}{\kern0pt}def\ True{\isacharparenright}{\kern0pt}\isanewline
\ \ \isacommand{ultimately}\isamarkupfalse%
\ \isacommand{show}\isamarkupfalse%
\ {\isacharquery}{\kern0pt}thesis\isanewline
\ \ \ \ \isacommand{by}\isamarkupfalse%
\ {\isacharparenleft}{\kern0pt}metis\ True\ card{\isacharunderscore}{\kern0pt}image{\isacharparenright}{\kern0pt}\isanewline
\isacommand{qed}\isamarkupfalse%
\ {\isacharparenleft}{\kern0pt}auto\ simp{\isacharcolon}{\kern0pt}\ smul{\isacharunderscore}{\kern0pt}eq{\isacharparenright}{\kern0pt}%
\endisatagproof
{\isafoldproof}%
%
\isadelimproof
\isanewline
%
\endisadelimproof
\isanewline
\isacommand{lemma}\isamarkupfalse%
\ card{\isacharunderscore}{\kern0pt}smul{\isacharunderscore}{\kern0pt}singleton{\isacharunderscore}{\kern0pt}left{\isacharunderscore}{\kern0pt}eq{\isacharcolon}{\kern0pt}\isanewline
\ \ \isakeyword{assumes}\ {\isachardoublequoteopen}finite\ A{\isachardoublequoteclose}\ \isakeyword{shows}\ {\isachardoublequoteopen}card\ {\isacharparenleft}{\kern0pt}smul\ {\isacharbraceleft}{\kern0pt}a{\isacharbraceright}{\kern0pt}\ A{\isacharparenright}{\kern0pt}\ {\isacharequal}{\kern0pt}\ {\isacharparenleft}{\kern0pt}if\ a\ {\isasymin}\ G\ then\ card\ {\isacharparenleft}{\kern0pt}A\ {\isasyminter}\ G{\isacharparenright}{\kern0pt}\ else\ {\isadigit{0}}{\isacharparenright}{\kern0pt}{\isachardoublequoteclose}\isanewline
%
\isadelimproof
%
\endisadelimproof
%
\isatagproof
\isacommand{proof}\isamarkupfalse%
\ {\isacharparenleft}{\kern0pt}cases\ {\isachardoublequoteopen}a\ {\isasymin}\ G{\isachardoublequoteclose}{\isacharparenright}{\kern0pt}\isanewline
\ \ \isacommand{case}\isamarkupfalse%
\ True\isanewline
\ \ \isacommand{then}\isamarkupfalse%
\ \isacommand{have}\isamarkupfalse%
\ {\isachardoublequoteopen}smul\ {\isacharbraceleft}{\kern0pt}a{\isacharbraceright}{\kern0pt}\ A\ {\isacharequal}{\kern0pt}\ {\isacharparenleft}{\kern0pt}{\isasymlambda}x{\isachardot}{\kern0pt}\ a\ {\isasymcdot}\ x{\isacharparenright}{\kern0pt}\ {\isacharbackquote}{\kern0pt}\ {\isacharparenleft}{\kern0pt}A\ {\isasyminter}\ G{\isacharparenright}{\kern0pt}{\isachardoublequoteclose}\isanewline
\ \ \ \ \isacommand{by}\isamarkupfalse%
\ {\isacharparenleft}{\kern0pt}auto\ simp{\isacharcolon}{\kern0pt}\ smul{\isacharunderscore}{\kern0pt}eq{\isacharparenright}{\kern0pt}\isanewline
\ \ \isacommand{moreover}\isamarkupfalse%
\ \isacommand{have}\isamarkupfalse%
\ {\isachardoublequoteopen}inj{\isacharunderscore}{\kern0pt}on\ {\isacharparenleft}{\kern0pt}{\isasymlambda}x{\isachardot}{\kern0pt}\ a\ {\isasymcdot}\ x{\isacharparenright}{\kern0pt}\ {\isacharparenleft}{\kern0pt}A\ {\isasyminter}\ G{\isacharparenright}{\kern0pt}{\isachardoublequoteclose}\isanewline
\ \ \ \ \isacommand{by}\isamarkupfalse%
\ {\isacharparenleft}{\kern0pt}auto\ simp{\isacharcolon}{\kern0pt}\ inj{\isacharunderscore}{\kern0pt}on{\isacharunderscore}{\kern0pt}def\ True{\isacharparenright}{\kern0pt}\isanewline
\ \ \isacommand{ultimately}\isamarkupfalse%
\ \isacommand{show}\isamarkupfalse%
\ {\isacharquery}{\kern0pt}thesis\isanewline
\ \ \ \ \isacommand{by}\isamarkupfalse%
\ {\isacharparenleft}{\kern0pt}metis\ True\ card{\isacharunderscore}{\kern0pt}image{\isacharparenright}{\kern0pt}\isanewline
\isacommand{qed}\isamarkupfalse%
\ {\isacharparenleft}{\kern0pt}auto\ simp{\isacharcolon}{\kern0pt}\ smul{\isacharunderscore}{\kern0pt}eq{\isacharparenright}{\kern0pt}%
\endisatagproof
{\isafoldproof}%
%
\isadelimproof
\isanewline
%
\endisadelimproof
\isanewline
\isacommand{lemma}\isamarkupfalse%
\ card{\isacharunderscore}{\kern0pt}smul{\isacharunderscore}{\kern0pt}sing{\isacharunderscore}{\kern0pt}right{\isacharunderscore}{\kern0pt}le{\isacharcolon}{\kern0pt}\isanewline
\ \ \isakeyword{assumes}\ {\isachardoublequoteopen}finite\ A{\isachardoublequoteclose}\ \isakeyword{shows}\ {\isachardoublequoteopen}card\ {\isacharparenleft}{\kern0pt}smul\ A\ {\isacharbraceleft}{\kern0pt}a{\isacharbraceright}{\kern0pt}{\isacharparenright}{\kern0pt}\ {\isasymle}\ card\ A{\isachardoublequoteclose}\isanewline
%
\isadelimproof
\ \ %
\endisadelimproof
%
\isatagproof
\isacommand{by}\isamarkupfalse%
\ {\isacharparenleft}{\kern0pt}simp\ add{\isacharcolon}{\kern0pt}\ assms\ card{\isacharunderscore}{\kern0pt}mono\ card{\isacharunderscore}{\kern0pt}smul{\isacharunderscore}{\kern0pt}singleton{\isacharunderscore}{\kern0pt}right{\isacharunderscore}{\kern0pt}eq{\isacharparenright}{\kern0pt}%
\endisatagproof
{\isafoldproof}%
%
\isadelimproof
\isanewline
%
\endisadelimproof
\isanewline
\isacommand{lemma}\isamarkupfalse%
\ card{\isacharunderscore}{\kern0pt}smul{\isacharunderscore}{\kern0pt}sing{\isacharunderscore}{\kern0pt}left{\isacharunderscore}{\kern0pt}le{\isacharcolon}{\kern0pt}\isanewline
\ \ \isakeyword{assumes}\ {\isachardoublequoteopen}finite\ A{\isachardoublequoteclose}\ \isakeyword{shows}\ {\isachardoublequoteopen}card\ {\isacharparenleft}{\kern0pt}smul\ {\isacharbraceleft}{\kern0pt}a{\isacharbraceright}{\kern0pt}\ A{\isacharparenright}{\kern0pt}\ {\isasymle}\ card\ A{\isachardoublequoteclose}\isanewline
%
\isadelimproof
\ \ %
\endisadelimproof
%
\isatagproof
\isacommand{by}\isamarkupfalse%
\ {\isacharparenleft}{\kern0pt}simp\ add{\isacharcolon}{\kern0pt}\ assms\ card{\isacharunderscore}{\kern0pt}mono\ card{\isacharunderscore}{\kern0pt}smul{\isacharunderscore}{\kern0pt}singleton{\isacharunderscore}{\kern0pt}left{\isacharunderscore}{\kern0pt}eq{\isacharparenright}{\kern0pt}%
\endisatagproof
{\isafoldproof}%
%
\isadelimproof
\isanewline
%
\endisadelimproof
\isanewline
\isacommand{lemma}\isamarkupfalse%
\ card{\isacharunderscore}{\kern0pt}le{\isacharunderscore}{\kern0pt}smul{\isacharunderscore}{\kern0pt}right{\isacharcolon}{\kern0pt}\isanewline
\ \ \isakeyword{assumes}\ A{\isacharcolon}{\kern0pt}\ {\isachardoublequoteopen}finite\ A{\isachardoublequoteclose}\ {\isachardoublequoteopen}a\ {\isasymin}\ A{\isachardoublequoteclose}\ {\isachardoublequoteopen}a\ {\isasymin}\ G{\isachardoublequoteclose}\isanewline
\ \ \ \ \isakeyword{and}\ \ \ B{\isacharcolon}{\kern0pt}\ {\isachardoublequoteopen}finite\ B{\isachardoublequoteclose}\ {\isachardoublequoteopen}B\ {\isasymsubseteq}\ G{\isachardoublequoteclose}\isanewline
\ \ \isakeyword{shows}\ {\isachardoublequoteopen}card\ B\ {\isasymle}\ card\ {\isacharparenleft}{\kern0pt}smul\ A\ B{\isacharparenright}{\kern0pt}{\isachardoublequoteclose}\isanewline
%
\isadelimproof
%
\endisadelimproof
%
\isatagproof
\isacommand{proof}\isamarkupfalse%
\ {\isacharminus}{\kern0pt}\isanewline
\ \ \isacommand{have}\isamarkupfalse%
\ {\isachardoublequoteopen}B\ {\isasymsubseteq}\ {\isacharparenleft}{\kern0pt}{\isasymlambda}\ x{\isachardot}{\kern0pt}\ \ {\isacharparenleft}{\kern0pt}inverse\ a{\isacharparenright}{\kern0pt}\ {\isasymcdot}\ x{\isacharparenright}{\kern0pt}\ {\isacharbackquote}{\kern0pt}\ smul\ A\ B{\isachardoublequoteclose}\isanewline
\ \ \ \ \isacommand{using}\isamarkupfalse%
\ A\ B\isanewline
\ \ \ \ \isacommand{apply}\isamarkupfalse%
\ {\isacharparenleft}{\kern0pt}clarsimp\ simp{\isacharcolon}{\kern0pt}\ smul\ image{\isacharunderscore}{\kern0pt}iff{\isacharparenright}{\kern0pt}\isanewline
\ \ \ \ \isacommand{using}\isamarkupfalse%
\ Int{\isacharunderscore}{\kern0pt}absorb{\isadigit{2}}\ Int{\isacharunderscore}{\kern0pt}iff\ invertible\ invertible{\isacharunderscore}{\kern0pt}left{\isacharunderscore}{\kern0pt}inverse{\isadigit{2}}\ \isacommand{by}\isamarkupfalse%
\ metis\isanewline
\ \ \isacommand{with}\isamarkupfalse%
\ A\ B\ \isacommand{show}\isamarkupfalse%
\ {\isacharquery}{\kern0pt}thesis\isanewline
\ \ \ \ \isacommand{by}\isamarkupfalse%
\ {\isacharparenleft}{\kern0pt}meson\ \ finite{\isacharunderscore}{\kern0pt}smul\ surj{\isacharunderscore}{\kern0pt}card{\isacharunderscore}{\kern0pt}le{\isacharparenright}{\kern0pt}\isanewline
\isacommand{qed}\isamarkupfalse%
%
\endisatagproof
{\isafoldproof}%
%
\isadelimproof
\isanewline
%
\endisadelimproof
\isanewline
\isacommand{lemma}\isamarkupfalse%
\ card{\isacharunderscore}{\kern0pt}le{\isacharunderscore}{\kern0pt}smul{\isacharunderscore}{\kern0pt}left{\isacharcolon}{\kern0pt}\isanewline
\ \ \isakeyword{assumes}\ A{\isacharcolon}{\kern0pt}\ {\isachardoublequoteopen}finite\ A{\isachardoublequoteclose}\ {\isachardoublequoteopen}b\ {\isasymin}\ B{\isachardoublequoteclose}\ {\isachardoublequoteopen}b\ {\isasymin}\ G{\isachardoublequoteclose}\isanewline
\ \ \ \ \isakeyword{and}\ \ \ B{\isacharcolon}{\kern0pt}\ {\isachardoublequoteopen}finite\ B{\isachardoublequoteclose}\ {\isachardoublequoteopen}A\ {\isasymsubseteq}\ G{\isachardoublequoteclose}\isanewline
\ \ \isakeyword{shows}\ {\isachardoublequoteopen}card\ A\ {\isasymle}\ card\ {\isacharparenleft}{\kern0pt}smul\ A\ B{\isacharparenright}{\kern0pt}{\isachardoublequoteclose}\isanewline
%
\isadelimproof
%
\endisadelimproof
%
\isatagproof
\isacommand{proof}\isamarkupfalse%
\ {\isacharminus}{\kern0pt}\isanewline
\ \ \isacommand{have}\isamarkupfalse%
\ {\isachardoublequoteopen}A\ {\isasymsubseteq}\ {\isacharparenleft}{\kern0pt}{\isasymlambda}\ x{\isachardot}{\kern0pt}\ x\ {\isasymcdot}\ {\isacharparenleft}{\kern0pt}inverse\ b{\isacharparenright}{\kern0pt}{\isacharparenright}{\kern0pt}\ {\isacharbackquote}{\kern0pt}\ smul\ A\ B{\isachardoublequoteclose}\isanewline
\ \ \ \ \isacommand{using}\isamarkupfalse%
\ A\ B\isanewline
\ \ \ \ \isacommand{apply}\isamarkupfalse%
\ {\isacharparenleft}{\kern0pt}clarsimp\ simp{\isacharcolon}{\kern0pt}\ smul\ image{\isacharunderscore}{\kern0pt}iff\ associative{\isacharparenright}{\kern0pt}\ \ \ \ \ \isanewline
\ \ \ \ \isacommand{using}\isamarkupfalse%
\ Int{\isacharunderscore}{\kern0pt}absorb{\isadigit{2}}\ Int{\isacharunderscore}{\kern0pt}iff\ invertible\ invertible{\isacharunderscore}{\kern0pt}right{\isacharunderscore}{\kern0pt}inverse\ assms{\isacharparenleft}{\kern0pt}{\isadigit{5}}{\isacharparenright}{\kern0pt}\ \isacommand{by}\isamarkupfalse%
\ {\isacharparenleft}{\kern0pt}metis\ right{\isacharunderscore}{\kern0pt}unit{\isacharparenright}{\kern0pt}\isanewline
\ \ \isacommand{with}\isamarkupfalse%
\ A\ B\ \isacommand{show}\isamarkupfalse%
\ {\isacharquery}{\kern0pt}thesis\isanewline
\ \ \ \ \isacommand{by}\isamarkupfalse%
\ {\isacharparenleft}{\kern0pt}meson\ \ finite{\isacharunderscore}{\kern0pt}smul\ surj{\isacharunderscore}{\kern0pt}card{\isacharunderscore}{\kern0pt}le{\isacharparenright}{\kern0pt}\isanewline
\isacommand{qed}\isamarkupfalse%
%
\endisatagproof
{\isafoldproof}%
%
\isadelimproof
\isanewline
%
\endisadelimproof
\isanewline
\isanewline
\isacommand{lemma}\isamarkupfalse%
\ infinite{\isacharunderscore}{\kern0pt}smul{\isacharunderscore}{\kern0pt}right{\isacharcolon}{\kern0pt}\isanewline
\ \ \isakeyword{assumes}\ {\isachardoublequoteopen}A\ {\isasyminter}\ G\ {\isasymnoteq}\ {\isacharbraceleft}{\kern0pt}{\isacharbraceright}{\kern0pt}{\isachardoublequoteclose}\ \isakeyword{and}\ {\isachardoublequoteopen}infinite\ {\isacharparenleft}{\kern0pt}B\ {\isasyminter}\ G{\isacharparenright}{\kern0pt}{\isachardoublequoteclose}\isanewline
\ \ \isakeyword{shows}\ {\isachardoublequoteopen}infinite\ {\isacharparenleft}{\kern0pt}A\ {\isasymcdots}\ B{\isacharparenright}{\kern0pt}{\isachardoublequoteclose}\ \isanewline
%
\isadelimproof
%
\endisadelimproof
%
\isatagproof
\isacommand{proof}\isamarkupfalse%
\isanewline
\ \ \isacommand{assume}\isamarkupfalse%
\ hfin{\isacharcolon}{\kern0pt}\ {\isachardoublequoteopen}finite\ {\isacharparenleft}{\kern0pt}smul\ A\ B{\isacharparenright}{\kern0pt}{\isachardoublequoteclose}\isanewline
\ \ \isacommand{obtain}\isamarkupfalse%
\ a\ \isakeyword{where}\ ha{\isacharcolon}{\kern0pt}\ {\isachardoublequoteopen}a\ {\isasymin}\ A\ {\isasyminter}\ G{\isachardoublequoteclose}\ \isacommand{using}\isamarkupfalse%
\ assms\ \isacommand{by}\isamarkupfalse%
\ auto\isanewline
\ \ \isacommand{then}\isamarkupfalse%
\ \isacommand{have}\isamarkupfalse%
\ {\isachardoublequoteopen}finite\ {\isacharparenleft}{\kern0pt}smul\ {\isacharbraceleft}{\kern0pt}a{\isacharbraceright}{\kern0pt}\ B{\isacharparenright}{\kern0pt}{\isachardoublequoteclose}\ \isacommand{using}\isamarkupfalse%
\ hfin\ \isacommand{by}\isamarkupfalse%
\ {\isacharparenleft}{\kern0pt}metis\ Int{\isacharunderscore}{\kern0pt}Un{\isacharunderscore}{\kern0pt}eq{\isacharparenleft}{\kern0pt}{\isadigit{1}}{\isacharparenright}{\kern0pt}\ finite{\isacharunderscore}{\kern0pt}subset\ insert{\isacharunderscore}{\kern0pt}is{\isacharunderscore}{\kern0pt}Un\ \isanewline
\ \ \ \ mk{\isacharunderscore}{\kern0pt}disjoint{\isacharunderscore}{\kern0pt}insert\ smul{\isacharunderscore}{\kern0pt}subset{\isacharunderscore}{\kern0pt}Un{\isacharparenleft}{\kern0pt}{\isadigit{2}}{\isacharparenright}{\kern0pt}{\isacharparenright}{\kern0pt}\isanewline
\ \ \isacommand{moreover}\isamarkupfalse%
\ \isacommand{have}\isamarkupfalse%
\ {\isachardoublequoteopen}B\ {\isasyminter}\ G\ {\isasymsubseteq}\ {\isacharparenleft}{\kern0pt}{\isasymlambda}\ x{\isachardot}{\kern0pt}\ inverse\ a\ {\isasymcdot}\ x{\isacharparenright}{\kern0pt}\ {\isacharbackquote}{\kern0pt}\ smul\ {\isacharbraceleft}{\kern0pt}a{\isacharbraceright}{\kern0pt}\ B{\isachardoublequoteclose}\ \isanewline
\ \ \isacommand{proof}\isamarkupfalse%
\isanewline
\ \ \ \ \isacommand{fix}\isamarkupfalse%
\ b\ \isacommand{assume}\isamarkupfalse%
\ hb{\isacharcolon}{\kern0pt}\ {\isachardoublequoteopen}b\ {\isasymin}\ B\ {\isasyminter}\ G{\isachardoublequoteclose}\isanewline
\ \ \ \ \isacommand{then}\isamarkupfalse%
\ \isacommand{have}\isamarkupfalse%
\ {\isachardoublequoteopen}b\ {\isacharequal}{\kern0pt}\ inverse\ a\ {\isasymcdot}\ {\isacharparenleft}{\kern0pt}a\ {\isasymcdot}\ b{\isacharparenright}{\kern0pt}{\isachardoublequoteclose}\ \isacommand{using}\isamarkupfalse%
\ associative\ ha\ \isacommand{by}\isamarkupfalse%
\ {\isacharparenleft}{\kern0pt}simp\ add{\isacharcolon}{\kern0pt}\ invertible{\isacharunderscore}{\kern0pt}left{\isacharunderscore}{\kern0pt}inverse{\isadigit{2}}{\isacharparenright}{\kern0pt}\isanewline
\ \ \ \ \isacommand{then}\isamarkupfalse%
\ \isacommand{show}\isamarkupfalse%
\ {\isachardoublequoteopen}b\ {\isasymin}\ {\isacharparenleft}{\kern0pt}{\isasymlambda}\ x{\isachardot}{\kern0pt}\ inverse\ a\ {\isasymcdot}\ x{\isacharparenright}{\kern0pt}\ {\isacharbackquote}{\kern0pt}\ smul\ {\isacharbraceleft}{\kern0pt}a{\isacharbraceright}{\kern0pt}\ B{\isachardoublequoteclose}\ \isacommand{using}\isamarkupfalse%
\ smul{\isachardot}{\kern0pt}simps\ hb\ ha\ \isacommand{by}\isamarkupfalse%
\ blast\isanewline
\ \ \isacommand{qed}\isamarkupfalse%
\isanewline
\ \ \isacommand{ultimately}\isamarkupfalse%
\ \isacommand{show}\isamarkupfalse%
\ False\ \isacommand{using}\isamarkupfalse%
\ assms\ \isacommand{using}\isamarkupfalse%
\ finite{\isacharunderscore}{\kern0pt}surj\ \isacommand{by}\isamarkupfalse%
\ blast\isanewline
\isacommand{qed}\isamarkupfalse%
%
\endisatagproof
{\isafoldproof}%
%
\isadelimproof
\isanewline
%
\endisadelimproof
\isanewline
\isacommand{lemma}\isamarkupfalse%
\ infinite{\isacharunderscore}{\kern0pt}smul{\isacharunderscore}{\kern0pt}left{\isacharcolon}{\kern0pt}\isanewline
\ \ \isakeyword{assumes}\ {\isachardoublequoteopen}B\ {\isasyminter}\ G\ {\isasymnoteq}\ {\isacharbraceleft}{\kern0pt}{\isacharbraceright}{\kern0pt}{\isachardoublequoteclose}\ \isakeyword{and}\ {\isachardoublequoteopen}infinite\ {\isacharparenleft}{\kern0pt}A\ {\isasyminter}\ G{\isacharparenright}{\kern0pt}{\isachardoublequoteclose}\isanewline
\ \ \isakeyword{shows}\ {\isachardoublequoteopen}infinite\ {\isacharparenleft}{\kern0pt}A\ {\isasymcdots}\ B{\isacharparenright}{\kern0pt}{\isachardoublequoteclose}\isanewline
%
\isadelimproof
%
\endisadelimproof
%
\isatagproof
\isacommand{proof}\isamarkupfalse%
\isanewline
\ \ \isacommand{assume}\isamarkupfalse%
\ hfin{\isacharcolon}{\kern0pt}\ {\isachardoublequoteopen}finite\ {\isacharparenleft}{\kern0pt}smul\ A\ B{\isacharparenright}{\kern0pt}{\isachardoublequoteclose}\isanewline
\ \ \isacommand{obtain}\isamarkupfalse%
\ b\ \isakeyword{where}\ hb{\isacharcolon}{\kern0pt}\ {\isachardoublequoteopen}b\ {\isasymin}\ B\ {\isasyminter}\ G{\isachardoublequoteclose}\ \isacommand{using}\isamarkupfalse%
\ assms\ \isacommand{by}\isamarkupfalse%
\ auto\isanewline
\ \ \isacommand{then}\isamarkupfalse%
\ \isacommand{have}\isamarkupfalse%
\ {\isachardoublequoteopen}finite\ {\isacharparenleft}{\kern0pt}smul\ A\ {\isacharbraceleft}{\kern0pt}b{\isacharbraceright}{\kern0pt}{\isacharparenright}{\kern0pt}{\isachardoublequoteclose}\ \isacommand{using}\isamarkupfalse%
\ hfin\ \isacommand{by}\isamarkupfalse%
\ {\isacharparenleft}{\kern0pt}simp\ add{\isacharcolon}{\kern0pt}\ rev{\isacharunderscore}{\kern0pt}finite{\isacharunderscore}{\kern0pt}subset\ smul{\isacharunderscore}{\kern0pt}mono{\isacharparenright}{\kern0pt}\isanewline
\ \ \isacommand{moreover}\isamarkupfalse%
\ \isacommand{have}\isamarkupfalse%
\ {\isachardoublequoteopen}A\ {\isasyminter}\ G\ {\isasymsubseteq}\ {\isacharparenleft}{\kern0pt}{\isasymlambda}\ x{\isachardot}{\kern0pt}\ x\ {\isasymcdot}\ inverse\ b{\isacharparenright}{\kern0pt}\ {\isacharbackquote}{\kern0pt}\ smul\ A\ {\isacharbraceleft}{\kern0pt}b{\isacharbraceright}{\kern0pt}{\isachardoublequoteclose}\isanewline
\ \ \isacommand{proof}\isamarkupfalse%
\isanewline
\ \ \ \ \isacommand{fix}\isamarkupfalse%
\ a\ \isacommand{assume}\isamarkupfalse%
\ ha{\isacharcolon}{\kern0pt}\ {\isachardoublequoteopen}a\ {\isasymin}\ A\ {\isasyminter}\ G{\isachardoublequoteclose}\isanewline
\ \ \ \ \isacommand{then}\isamarkupfalse%
\ \isacommand{have}\isamarkupfalse%
\ {\isachardoublequoteopen}a\ {\isacharequal}{\kern0pt}\ {\isacharparenleft}{\kern0pt}a\ {\isasymcdot}\ b{\isacharparenright}{\kern0pt}\ {\isasymcdot}\ inverse\ b{\isachardoublequoteclose}\ \isacommand{using}\isamarkupfalse%
\ associative\ hb\isanewline
\ \ \ \ \ \ \isacommand{by}\isamarkupfalse%
\ {\isacharparenleft}{\kern0pt}metis\ IntD{\isadigit{2}}\ invertible\ invertible{\isacharunderscore}{\kern0pt}inverse{\isacharunderscore}{\kern0pt}closed\ invertible{\isacharunderscore}{\kern0pt}right{\isacharunderscore}{\kern0pt}inverse\ right{\isacharunderscore}{\kern0pt}unit{\isacharparenright}{\kern0pt}\isanewline
\ \ \ \ \isacommand{then}\isamarkupfalse%
\ \isacommand{show}\isamarkupfalse%
\ {\isachardoublequoteopen}a\ {\isasymin}\ {\isacharparenleft}{\kern0pt}{\isasymlambda}\ x{\isachardot}{\kern0pt}\ x\ {\isasymcdot}\ inverse\ b{\isacharparenright}{\kern0pt}\ {\isacharbackquote}{\kern0pt}\ smul\ A\ {\isacharbraceleft}{\kern0pt}b{\isacharbraceright}{\kern0pt}{\isachardoublequoteclose}\ \isacommand{using}\isamarkupfalse%
\ smul{\isachardot}{\kern0pt}simps\ hb\ ha\ \isacommand{by}\isamarkupfalse%
\ blast\isanewline
\ \ \isacommand{qed}\isamarkupfalse%
\isanewline
\ \ \isacommand{ultimately}\isamarkupfalse%
\ \isacommand{show}\isamarkupfalse%
\ False\ \isacommand{using}\isamarkupfalse%
\ assms\ \isacommand{using}\isamarkupfalse%
\ finite{\isacharunderscore}{\kern0pt}surj\ \isacommand{by}\isamarkupfalse%
\ blast\isanewline
\isacommand{qed}\isamarkupfalse%
%
\endisatagproof
{\isafoldproof}%
%
\isadelimproof
%
\endisadelimproof
%
\isadelimdocument
%
\endisadelimdocument
%
\isatagdocument
%
\isamarkupsubsection{Pointwise set multiplication in group: auxiliary lemmas%
}
\isamarkuptrue%
%
\endisatagdocument
{\isafolddocument}%
%
\isadelimdocument
%
\endisadelimdocument
\isacommand{lemma}\isamarkupfalse%
\ set{\isacharunderscore}{\kern0pt}inverse{\isacharunderscore}{\kern0pt}composition{\isacharunderscore}{\kern0pt}commute{\isacharcolon}{\kern0pt}\isanewline
\ \ \isakeyword{assumes}\ {\isachardoublequoteopen}X\ {\isasymsubseteq}\ G{\isachardoublequoteclose}\ \isakeyword{and}\ {\isachardoublequoteopen}Y\ {\isasymsubseteq}\ G{\isachardoublequoteclose}\isanewline
\ \ \isakeyword{shows}\ {\isachardoublequoteopen}inverse\ {\isacharbackquote}{\kern0pt}\ {\isacharparenleft}{\kern0pt}X\ {\isasymcdots}\ Y{\isacharparenright}{\kern0pt}\ {\isacharequal}{\kern0pt}\ {\isacharparenleft}{\kern0pt}inverse\ {\isacharbackquote}{\kern0pt}\ Y{\isacharparenright}{\kern0pt}\ {\isasymcdots}\ {\isacharparenleft}{\kern0pt}inverse\ {\isacharbackquote}{\kern0pt}\ X{\isacharparenright}{\kern0pt}{\isachardoublequoteclose}\isanewline
%
\isadelimproof
%
\endisadelimproof
%
\isatagproof
\isacommand{proof}\isamarkupfalse%
\isanewline
\ \ \isacommand{show}\isamarkupfalse%
\ {\isachardoublequoteopen}inverse\ {\isacharbackquote}{\kern0pt}\ {\isacharparenleft}{\kern0pt}X\ {\isasymcdots}\ Y{\isacharparenright}{\kern0pt}\ {\isasymsubseteq}\ {\isacharparenleft}{\kern0pt}inverse\ {\isacharbackquote}{\kern0pt}\ Y{\isacharparenright}{\kern0pt}\ {\isasymcdots}\ {\isacharparenleft}{\kern0pt}inverse\ {\isacharbackquote}{\kern0pt}\ X{\isacharparenright}{\kern0pt}{\isachardoublequoteclose}\isanewline
\ \ \isacommand{proof}\isamarkupfalse%
\isanewline
\ \ \ \ \isacommand{fix}\isamarkupfalse%
\ z\ \isacommand{assume}\isamarkupfalse%
\ {\isachardoublequoteopen}z\ {\isasymin}\ inverse\ {\isacharbackquote}{\kern0pt}\ {\isacharparenleft}{\kern0pt}X\ {\isasymcdots}\ Y{\isacharparenright}{\kern0pt}{\isachardoublequoteclose}\isanewline
\ \ \ \ \isacommand{then}\isamarkupfalse%
\ \isacommand{obtain}\isamarkupfalse%
\ x\ y\ \isakeyword{where}\ {\isachardoublequoteopen}z\ {\isacharequal}{\kern0pt}\ inverse\ {\isacharparenleft}{\kern0pt}x\ {\isasymcdot}\ y{\isacharparenright}{\kern0pt}{\isachardoublequoteclose}\ \isakeyword{and}\ {\isachardoublequoteopen}x\ {\isasymin}\ X{\isachardoublequoteclose}\ \isakeyword{and}\ {\isachardoublequoteopen}y\ {\isasymin}\ Y{\isachardoublequoteclose}\ \isanewline
\ \ \ \ \ \ \isacommand{by}\isamarkupfalse%
\ {\isacharparenleft}{\kern0pt}smt\ {\isacharparenleft}{\kern0pt}verit{\isacharparenright}{\kern0pt}\ image{\isacharunderscore}{\kern0pt}iff\ smul{\isachardot}{\kern0pt}cases{\isacharparenright}{\kern0pt}\isanewline
\ \ \ \ \isacommand{then}\isamarkupfalse%
\ \isacommand{show}\isamarkupfalse%
\ {\isachardoublequoteopen}z\ {\isasymin}\ {\isacharparenleft}{\kern0pt}inverse\ {\isacharbackquote}{\kern0pt}\ Y{\isacharparenright}{\kern0pt}\ {\isasymcdots}\ {\isacharparenleft}{\kern0pt}inverse\ {\isacharbackquote}{\kern0pt}\ X{\isacharparenright}{\kern0pt}{\isachardoublequoteclose}\ \isanewline
\ \ \ \ \ \ \isacommand{using}\isamarkupfalse%
\ inverse{\isacharunderscore}{\kern0pt}composition{\isacharunderscore}{\kern0pt}commute\ assms\ \isanewline
\ \ \ \ \ \ \isacommand{by}\isamarkupfalse%
\ {\isacharparenleft}{\kern0pt}smt\ {\isacharparenleft}{\kern0pt}verit{\isacharparenright}{\kern0pt}\ image{\isacharunderscore}{\kern0pt}eqI\ in{\isacharunderscore}{\kern0pt}mono\ inverse{\isacharunderscore}{\kern0pt}equality\ invertible\ invertibleE\ smul{\isachardot}{\kern0pt}simps{\isacharparenright}{\kern0pt}\isanewline
\ \ \isacommand{qed}\isamarkupfalse%
\isanewline
\ \ \isacommand{show}\isamarkupfalse%
\ {\isachardoublequoteopen}{\isacharparenleft}{\kern0pt}inverse\ {\isacharbackquote}{\kern0pt}\ Y{\isacharparenright}{\kern0pt}\ {\isasymcdots}\ {\isacharparenleft}{\kern0pt}inverse\ {\isacharbackquote}{\kern0pt}\ X{\isacharparenright}{\kern0pt}\ {\isasymsubseteq}\ inverse\ {\isacharbackquote}{\kern0pt}\ {\isacharparenleft}{\kern0pt}X\ {\isasymcdots}\ Y{\isacharparenright}{\kern0pt}{\isachardoublequoteclose}\isanewline
\ \ \isacommand{proof}\isamarkupfalse%
\isanewline
\ \ \ \ \isacommand{fix}\isamarkupfalse%
\ z\ \isacommand{assume}\isamarkupfalse%
\ {\isachardoublequoteopen}z\ {\isasymin}\ {\isacharparenleft}{\kern0pt}inverse\ {\isacharbackquote}{\kern0pt}\ Y{\isacharparenright}{\kern0pt}\ {\isasymcdots}\ {\isacharparenleft}{\kern0pt}inverse\ {\isacharbackquote}{\kern0pt}\ X{\isacharparenright}{\kern0pt}{\isachardoublequoteclose}\isanewline
\ \ \ \ \isacommand{then}\isamarkupfalse%
\ \isacommand{obtain}\isamarkupfalse%
\ x\ y\ \isakeyword{where}\ {\isachardoublequoteopen}x\ {\isasymin}\ X{\isachardoublequoteclose}\ \isakeyword{and}\ {\isachardoublequoteopen}y\ {\isasymin}\ Y{\isachardoublequoteclose}\ \isakeyword{and}\ {\isachardoublequoteopen}z\ {\isacharequal}{\kern0pt}\ inverse\ y\ {\isasymcdot}\ inverse\ x{\isachardoublequoteclose}\ \isanewline
\ \ \ \ \ \ \isacommand{using}\isamarkupfalse%
\ smul{\isachardot}{\kern0pt}cases\ image{\isacharunderscore}{\kern0pt}iff\ \isacommand{by}\isamarkupfalse%
\ blast\isanewline
\ \ \ \ \isacommand{then}\isamarkupfalse%
\ \isacommand{show}\isamarkupfalse%
\ {\isachardoublequoteopen}z\ {\isasymin}\ inverse\ {\isacharbackquote}{\kern0pt}\ {\isacharparenleft}{\kern0pt}X\ {\isasymcdots}\ Y{\isacharparenright}{\kern0pt}{\isachardoublequoteclose}\ \isacommand{using}\isamarkupfalse%
\ inverse{\isacharunderscore}{\kern0pt}composition{\isacharunderscore}{\kern0pt}commute\ assms\ smul{\isachardot}{\kern0pt}simps\isanewline
\ \ \ \ \ \ \isacommand{by}\isamarkupfalse%
\ {\isacharparenleft}{\kern0pt}smt\ {\isacharparenleft}{\kern0pt}verit{\isacharparenright}{\kern0pt}\ image{\isacharunderscore}{\kern0pt}iff\ in{\isacharunderscore}{\kern0pt}mono\ invertible{\isacharparenright}{\kern0pt}\isanewline
\ \ \isacommand{qed}\isamarkupfalse%
\isanewline
\isacommand{qed}\isamarkupfalse%
%
\endisatagproof
{\isafoldproof}%
%
\isadelimproof
\isanewline
%
\endisadelimproof
\isanewline
\isacommand{lemma}\isamarkupfalse%
\ smul{\isacharunderscore}{\kern0pt}singleton{\isacharunderscore}{\kern0pt}eq{\isacharunderscore}{\kern0pt}contains{\isacharunderscore}{\kern0pt}nat{\isacharunderscore}{\kern0pt}powers{\isacharcolon}{\kern0pt}\isanewline
\ \ \isakeyword{fixes}\ n\ {\isacharcolon}{\kern0pt}{\isacharcolon}{\kern0pt}\ nat\isanewline
\ \ \isakeyword{assumes}\ {\isachardoublequoteopen}X\ {\isasymsubseteq}\ G{\isachardoublequoteclose}\ \isakeyword{and}\ {\isachardoublequoteopen}g\ {\isasymin}\ G{\isachardoublequoteclose}\ \isakeyword{and}\ {\isachardoublequoteopen}X\ {\isasymcdots}\ {\isacharbraceleft}{\kern0pt}g{\isacharbraceright}{\kern0pt}\ {\isacharequal}{\kern0pt}\ X{\isachardoublequoteclose}\isanewline
\ \ \isakeyword{shows}\ {\isachardoublequoteopen}X\ {\isasymcdots}\ {\isacharbraceleft}{\kern0pt}g\ {\isacharcircum}{\kern0pt}\ n{\isacharbraceright}{\kern0pt}\ {\isacharequal}{\kern0pt}\ X{\isachardoublequoteclose}\isanewline
%
\isadelimproof
%
\endisadelimproof
%
\isatagproof
\isacommand{proof}\isamarkupfalse%
{\isacharparenleft}{\kern0pt}induction\ n{\isacharparenright}{\kern0pt}\isanewline
\ \ \isacommand{case}\isamarkupfalse%
\ {\isadigit{0}}\isanewline
\ \ \isacommand{then}\isamarkupfalse%
\ \isacommand{show}\isamarkupfalse%
\ {\isacharquery}{\kern0pt}case\ \isacommand{using}\isamarkupfalse%
\ power{\isacharunderscore}{\kern0pt}def\ assms\ \isacommand{by}\isamarkupfalse%
\ auto\isanewline
\isacommand{next}\isamarkupfalse%
\isanewline
\ \ \isacommand{case}\isamarkupfalse%
\ {\isacharparenleft}{\kern0pt}Suc\ n{\isacharparenright}{\kern0pt}\isanewline
\ \ \isacommand{assume}\isamarkupfalse%
\ hXn{\isacharcolon}{\kern0pt}\ {\isachardoublequoteopen}X\ {\isasymcdots}\ {\isacharbraceleft}{\kern0pt}g\ {\isacharcircum}{\kern0pt}\ n{\isacharbraceright}{\kern0pt}\ {\isacharequal}{\kern0pt}\ X{\isachardoublequoteclose}\isanewline
\ \ \isacommand{moreover}\isamarkupfalse%
\ \isacommand{have}\isamarkupfalse%
\ {\isachardoublequoteopen}X\ {\isasymcdots}\ {\isacharbraceleft}{\kern0pt}g\ {\isacharcircum}{\kern0pt}\ Suc\ n{\isacharbraceright}{\kern0pt}\ {\isacharequal}{\kern0pt}\ {\isacharparenleft}{\kern0pt}X\ {\isasymcdots}\ {\isacharbraceleft}{\kern0pt}g\ {\isacharcircum}{\kern0pt}\ n{\isacharbraceright}{\kern0pt}{\isacharparenright}{\kern0pt}\ {\isasymcdots}\ {\isacharbraceleft}{\kern0pt}g{\isacharbraceright}{\kern0pt}{\isachardoublequoteclose}\isanewline
\ \ \isacommand{proof}\isamarkupfalse%
\isanewline
\ \ \ \ \isacommand{show}\isamarkupfalse%
\ {\isachardoublequoteopen}X\ {\isasymcdots}\ {\isacharbraceleft}{\kern0pt}g\ {\isacharcircum}{\kern0pt}\ Suc\ n{\isacharbraceright}{\kern0pt}\ {\isasymsubseteq}\ {\isacharparenleft}{\kern0pt}X\ {\isasymcdots}\ {\isacharbraceleft}{\kern0pt}g\ {\isacharcircum}{\kern0pt}\ n{\isacharbraceright}{\kern0pt}{\isacharparenright}{\kern0pt}\ {\isasymcdots}\ {\isacharbraceleft}{\kern0pt}g{\isacharbraceright}{\kern0pt}{\isachardoublequoteclose}\isanewline
\ \ \ \ \isacommand{proof}\isamarkupfalse%
\isanewline
\ \ \ \ \ \ \isacommand{fix}\isamarkupfalse%
\ z\ \isacommand{assume}\isamarkupfalse%
\ {\isachardoublequoteopen}z\ {\isasymin}\ X\ {\isasymcdots}\ {\isacharbraceleft}{\kern0pt}g\ {\isacharcircum}{\kern0pt}\ Suc\ n{\isacharbraceright}{\kern0pt}{\isachardoublequoteclose}\isanewline
\ \ \ \ \ \ \isacommand{then}\isamarkupfalse%
\ \isacommand{obtain}\isamarkupfalse%
\ x\ \isakeyword{where}\ {\isachardoublequoteopen}z\ {\isacharequal}{\kern0pt}\ x\ {\isasymcdot}\ {\isacharparenleft}{\kern0pt}g\ {\isacharcircum}{\kern0pt}\ Suc\ n{\isacharparenright}{\kern0pt}{\isachardoublequoteclose}\ \isakeyword{and}\ hx{\isacharcolon}{\kern0pt}\ {\isachardoublequoteopen}x\ {\isasymin}\ X{\isachardoublequoteclose}\ \ \isacommand{using}\isamarkupfalse%
\ smul{\isachardot}{\kern0pt}simps\ \isacommand{by}\isamarkupfalse%
\ auto\isanewline
\ \ \ \ \ \ \isacommand{then}\isamarkupfalse%
\ \isacommand{have}\isamarkupfalse%
\ {\isachardoublequoteopen}z\ {\isacharequal}{\kern0pt}\ {\isacharparenleft}{\kern0pt}x\ {\isasymcdot}\ g\ {\isacharcircum}{\kern0pt}\ n{\isacharparenright}{\kern0pt}\ {\isasymcdot}\ g{\isachardoublequoteclose}\ \isacommand{using}\isamarkupfalse%
\ assms\ associative\ \isacommand{by}\isamarkupfalse%
\ {\isacharparenleft}{\kern0pt}simp\ add{\isacharcolon}{\kern0pt}\ in{\isacharunderscore}{\kern0pt}mono\ power{\isacharunderscore}{\kern0pt}mem{\isacharunderscore}{\kern0pt}carrier{\isacharparenright}{\kern0pt}\ \isanewline
\ \ \ \ \ \ \isacommand{then}\isamarkupfalse%
\ \isacommand{show}\isamarkupfalse%
\ {\isachardoublequoteopen}z\ {\isasymin}\ {\isacharparenleft}{\kern0pt}X\ {\isasymcdots}\ {\isacharbraceleft}{\kern0pt}g\ {\isacharcircum}{\kern0pt}\ n{\isacharbraceright}{\kern0pt}{\isacharparenright}{\kern0pt}\ {\isasymcdots}\ {\isacharbraceleft}{\kern0pt}g{\isacharbraceright}{\kern0pt}{\isachardoublequoteclose}\ \isacommand{using}\isamarkupfalse%
\ hx\ assms\ \isanewline
\ \ \ \ \ \ \ \ \isacommand{by}\isamarkupfalse%
\ {\isacharparenleft}{\kern0pt}simp\ add{\isacharcolon}{\kern0pt}\ power{\isacharunderscore}{\kern0pt}mem{\isacharunderscore}{\kern0pt}carrier\ smul{\isachardot}{\kern0pt}smulI\ subsetD{\isacharparenright}{\kern0pt}\isanewline
\ \ \ \ \isacommand{qed}\isamarkupfalse%
\isanewline
\ \ \isacommand{next}\isamarkupfalse%
\isanewline
\ \ \ \ \isacommand{show}\isamarkupfalse%
\ {\isachardoublequoteopen}{\isacharparenleft}{\kern0pt}X\ {\isasymcdots}\ {\isacharbraceleft}{\kern0pt}g\ {\isacharcircum}{\kern0pt}\ n{\isacharbraceright}{\kern0pt}{\isacharparenright}{\kern0pt}\ {\isasymcdots}\ {\isacharbraceleft}{\kern0pt}g{\isacharbraceright}{\kern0pt}\ {\isasymsubseteq}\ X\ {\isasymcdots}\ {\isacharbraceleft}{\kern0pt}g\ {\isacharcircum}{\kern0pt}\ Suc\ n{\isacharbraceright}{\kern0pt}{\isachardoublequoteclose}\isanewline
\ \ \ \ \isacommand{proof}\isamarkupfalse%
\isanewline
\ \ \ \ \ \ \isacommand{fix}\isamarkupfalse%
\ z\ \isacommand{assume}\isamarkupfalse%
\ {\isachardoublequoteopen}z\ {\isasymin}\ {\isacharparenleft}{\kern0pt}X\ {\isasymcdots}\ {\isacharbraceleft}{\kern0pt}g\ {\isacharcircum}{\kern0pt}\ n{\isacharbraceright}{\kern0pt}{\isacharparenright}{\kern0pt}\ {\isasymcdots}\ {\isacharbraceleft}{\kern0pt}g{\isacharbraceright}{\kern0pt}{\isachardoublequoteclose}\isanewline
\ \ \ \ \ \ \isacommand{then}\isamarkupfalse%
\ \isacommand{obtain}\isamarkupfalse%
\ x\ \isakeyword{where}\ {\isachardoublequoteopen}z\ {\isacharequal}{\kern0pt}\ {\isacharparenleft}{\kern0pt}x\ {\isasymcdot}\ g\ {\isacharcircum}{\kern0pt}\ n{\isacharparenright}{\kern0pt}\ {\isasymcdot}\ g{\isachardoublequoteclose}\ \isakeyword{and}\ hx{\isacharcolon}{\kern0pt}\ {\isachardoublequoteopen}x\ {\isasymin}\ X{\isachardoublequoteclose}\ \isacommand{using}\isamarkupfalse%
\ smul{\isachardot}{\kern0pt}simps\ \isacommand{by}\isamarkupfalse%
\ auto\isanewline
\ \ \ \ \ \ \isacommand{then}\isamarkupfalse%
\ \isacommand{have}\isamarkupfalse%
\ {\isachardoublequoteopen}z\ {\isacharequal}{\kern0pt}\ x\ {\isasymcdot}\ g\ {\isacharcircum}{\kern0pt}\ Suc\ n{\isachardoublequoteclose}\ \isanewline
\ \ \ \ \ \ \ \ \isacommand{using}\isamarkupfalse%
\ power{\isacharunderscore}{\kern0pt}def\ associative\ power{\isacharunderscore}{\kern0pt}mem{\isacharunderscore}{\kern0pt}carrier\ assms\ \isacommand{by}\isamarkupfalse%
\ {\isacharparenleft}{\kern0pt}simp\ add{\isacharcolon}{\kern0pt}\ in{\isacharunderscore}{\kern0pt}mono{\isacharparenright}{\kern0pt}\isanewline
\ \ \ \ \ \ \isacommand{then}\isamarkupfalse%
\ \isacommand{show}\isamarkupfalse%
\ {\isachardoublequoteopen}z\ {\isasymin}\ X\ {\isasymcdots}\ {\isacharbraceleft}{\kern0pt}g\ {\isacharcircum}{\kern0pt}\ Suc\ n{\isacharbraceright}{\kern0pt}{\isachardoublequoteclose}\ \isacommand{using}\isamarkupfalse%
\ hx\ assms\ \isanewline
\ \ \ \ \ \ \ \ \isacommand{by}\isamarkupfalse%
\ {\isacharparenleft}{\kern0pt}simp\ add{\isacharcolon}{\kern0pt}\ power{\isacharunderscore}{\kern0pt}mem{\isacharunderscore}{\kern0pt}carrier\ smul{\isachardot}{\kern0pt}smulI\ subsetD{\isacharparenright}{\kern0pt}\isanewline
\ \ \ \ \isacommand{qed}\isamarkupfalse%
\isanewline
\ \ \isacommand{qed}\isamarkupfalse%
\isanewline
\ \ \isacommand{ultimately}\isamarkupfalse%
\ \isacommand{show}\isamarkupfalse%
\ {\isacharquery}{\kern0pt}case\ \isacommand{using}\isamarkupfalse%
\ assms\ \isacommand{by}\isamarkupfalse%
\ simp\isanewline
\isacommand{qed}\isamarkupfalse%
%
\endisatagproof
{\isafoldproof}%
%
\isadelimproof
\isanewline
%
\endisadelimproof
\isanewline
\isacommand{lemma}\isamarkupfalse%
\ smul{\isacharunderscore}{\kern0pt}singleton{\isacharunderscore}{\kern0pt}eq{\isacharunderscore}{\kern0pt}contains{\isacharunderscore}{\kern0pt}inverse{\isacharunderscore}{\kern0pt}nat{\isacharunderscore}{\kern0pt}powers{\isacharcolon}{\kern0pt}\isanewline
\ \ \isakeyword{fixes}\ n\ {\isacharcolon}{\kern0pt}{\isacharcolon}{\kern0pt}\ nat\isanewline
\ \ \isakeyword{assumes}\ {\isachardoublequoteopen}X\ {\isasymsubseteq}\ G{\isachardoublequoteclose}\ \isakeyword{and}\ {\isachardoublequoteopen}g\ {\isasymin}\ G{\isachardoublequoteclose}\ \isakeyword{and}\ {\isachardoublequoteopen}X\ {\isasymcdots}\ {\isacharbraceleft}{\kern0pt}g{\isacharbraceright}{\kern0pt}\ {\isacharequal}{\kern0pt}\ X{\isachardoublequoteclose}\isanewline
\ \ \isakeyword{shows}\ {\isachardoublequoteopen}X\ {\isasymcdots}\ {\isacharbraceleft}{\kern0pt}{\isacharparenleft}{\kern0pt}inverse\ g{\isacharparenright}{\kern0pt}\ {\isacharcircum}{\kern0pt}\ n{\isacharbraceright}{\kern0pt}\ {\isacharequal}{\kern0pt}\ X{\isachardoublequoteclose}\isanewline
%
\isadelimproof
%
\endisadelimproof
%
\isatagproof
\isacommand{proof}\isamarkupfalse%
{\isacharminus}{\kern0pt}\isanewline
\ \ \isacommand{have}\isamarkupfalse%
\ {\isachardoublequoteopen}{\isacharparenleft}{\kern0pt}X\ {\isasymcdots}\ {\isacharbraceleft}{\kern0pt}g{\isacharbraceright}{\kern0pt}{\isacharparenright}{\kern0pt}\ {\isasymcdots}\ {\isacharbraceleft}{\kern0pt}inverse\ g{\isacharbraceright}{\kern0pt}\ {\isacharequal}{\kern0pt}\ X{\isachardoublequoteclose}\isanewline
\ \ \isacommand{proof}\isamarkupfalse%
\isanewline
\ \ \ \ \isacommand{show}\isamarkupfalse%
\ {\isachardoublequoteopen}{\isacharparenleft}{\kern0pt}X\ {\isasymcdots}\ {\isacharbraceleft}{\kern0pt}g{\isacharbraceright}{\kern0pt}{\isacharparenright}{\kern0pt}\ {\isasymcdots}\ {\isacharbraceleft}{\kern0pt}inverse\ g{\isacharbraceright}{\kern0pt}\ {\isasymsubseteq}\ X{\isachardoublequoteclose}\isanewline
\ \ \ \ \isacommand{proof}\isamarkupfalse%
\isanewline
\ \ \ \ \ \ \isacommand{fix}\isamarkupfalse%
\ z\ \isacommand{assume}\isamarkupfalse%
\ {\isachardoublequoteopen}z\ {\isasymin}\ {\isacharparenleft}{\kern0pt}X\ {\isasymcdots}\ {\isacharbraceleft}{\kern0pt}g{\isacharbraceright}{\kern0pt}{\isacharparenright}{\kern0pt}\ {\isasymcdots}\ {\isacharbraceleft}{\kern0pt}inverse\ g{\isacharbraceright}{\kern0pt}{\isachardoublequoteclose}\isanewline
\ \ \ \ \ \ \isacommand{then}\isamarkupfalse%
\ \isacommand{obtain}\isamarkupfalse%
\ y\ x\ \isakeyword{where}\ {\isachardoublequoteopen}y\ {\isasymin}\ X\ {\isasymcdots}\ {\isacharbraceleft}{\kern0pt}g{\isacharbraceright}{\kern0pt}{\isachardoublequoteclose}\ \isakeyword{and}\ {\isachardoublequoteopen}z\ {\isacharequal}{\kern0pt}\ y\ {\isasymcdot}\ inverse\ g{\isachardoublequoteclose}\ \isakeyword{and}\ {\isachardoublequoteopen}x\ {\isasymin}\ X{\isachardoublequoteclose}\ \ \isakeyword{and}\ {\isachardoublequoteopen}y\ {\isacharequal}{\kern0pt}\ x\ {\isasymcdot}\ g{\isachardoublequoteclose}\ \isanewline
\ \ \ \ \ \ \ \ \isacommand{using}\isamarkupfalse%
\ assms\ smul{\isachardot}{\kern0pt}simps\ \isacommand{by}\isamarkupfalse%
\ {\isacharparenleft}{\kern0pt}metis\ empty{\isacharunderscore}{\kern0pt}iff\ insert{\isacharunderscore}{\kern0pt}iff{\isacharparenright}{\kern0pt}\isanewline
\ \ \ \ \ \ \isacommand{then}\isamarkupfalse%
\ \isacommand{show}\isamarkupfalse%
\ {\isachardoublequoteopen}z\ {\isasymin}\ X{\isachardoublequoteclose}\ \isacommand{using}\isamarkupfalse%
\ assms\ \isacommand{by}\isamarkupfalse%
\ {\isacharparenleft}{\kern0pt}simp\ add{\isacharcolon}{\kern0pt}\ associative\ subset{\isacharunderscore}{\kern0pt}eq{\isacharparenright}{\kern0pt}\isanewline
\ \ \ \ \isacommand{qed}\isamarkupfalse%
\isanewline
\ \ \isacommand{next}\isamarkupfalse%
\isanewline
\ \ \ \ \isacommand{show}\isamarkupfalse%
\ {\isachardoublequoteopen}X\ {\isasymsubseteq}\ {\isacharparenleft}{\kern0pt}X\ {\isasymcdots}\ {\isacharbraceleft}{\kern0pt}g{\isacharbraceright}{\kern0pt}{\isacharparenright}{\kern0pt}\ {\isasymcdots}\ {\isacharbraceleft}{\kern0pt}inverse\ g{\isacharbraceright}{\kern0pt}{\isachardoublequoteclose}\ \isanewline
\ \ \ \ \isacommand{proof}\isamarkupfalse%
\isanewline
\ \ \ \ \ \ \isacommand{fix}\isamarkupfalse%
\ x\ \isacommand{assume}\isamarkupfalse%
\ hx{\isacharcolon}{\kern0pt}\ {\isachardoublequoteopen}x\ {\isasymin}\ X{\isachardoublequoteclose}\isanewline
\ \ \ \ \ \ \isacommand{then}\isamarkupfalse%
\ \isacommand{have}\isamarkupfalse%
\ {\isachardoublequoteopen}x\ {\isacharequal}{\kern0pt}\ x\ {\isasymcdot}\ g\ {\isasymcdot}\ inverse\ g{\isachardoublequoteclose}\ \isacommand{using}\isamarkupfalse%
\ assms\ \isacommand{by}\isamarkupfalse%
\ {\isacharparenleft}{\kern0pt}simp\ add{\isacharcolon}{\kern0pt}\ associative\ subset{\isacharunderscore}{\kern0pt}iff{\isacharparenright}{\kern0pt}\isanewline
\ \ \ \ \ \ \isacommand{then}\isamarkupfalse%
\ \isacommand{show}\isamarkupfalse%
\ {\isachardoublequoteopen}x\ {\isasymin}\ {\isacharparenleft}{\kern0pt}X\ {\isasymcdots}\ {\isacharbraceleft}{\kern0pt}g{\isacharbraceright}{\kern0pt}{\isacharparenright}{\kern0pt}\ {\isasymcdots}\ {\isacharbraceleft}{\kern0pt}inverse\ g{\isacharbraceright}{\kern0pt}{\isachardoublequoteclose}\ \isacommand{using}\isamarkupfalse%
\ assms\ smul{\isachardot}{\kern0pt}simps\ hx\ \isacommand{by}\isamarkupfalse%
\ auto\isanewline
\ \ \ \ \isacommand{qed}\isamarkupfalse%
\isanewline
\ \ \isacommand{qed}\isamarkupfalse%
\isanewline
\ \ \isacommand{then}\isamarkupfalse%
\ \isacommand{have}\isamarkupfalse%
\ {\isachardoublequoteopen}X\ {\isasymcdots}\ {\isacharbraceleft}{\kern0pt}inverse\ g{\isacharbraceright}{\kern0pt}\ {\isacharequal}{\kern0pt}\ X{\isachardoublequoteclose}\ \isacommand{using}\isamarkupfalse%
\ assms\ \isacommand{by}\isamarkupfalse%
\ auto\isanewline
\ \ \isacommand{then}\isamarkupfalse%
\ \isacommand{show}\isamarkupfalse%
\ {\isacharquery}{\kern0pt}thesis\ \isacommand{using}\isamarkupfalse%
\ assms\ \isacommand{by}\isamarkupfalse%
\ {\isacharparenleft}{\kern0pt}simp\ add{\isacharcolon}{\kern0pt}\ smul{\isacharunderscore}{\kern0pt}singleton{\isacharunderscore}{\kern0pt}eq{\isacharunderscore}{\kern0pt}contains{\isacharunderscore}{\kern0pt}nat{\isacharunderscore}{\kern0pt}powers{\isacharparenright}{\kern0pt}\isanewline
\isacommand{qed}\isamarkupfalse%
%
\endisatagproof
{\isafoldproof}%
%
\isadelimproof
\isanewline
%
\endisadelimproof
\isanewline
\isacommand{lemma}\isamarkupfalse%
\ smul{\isacharunderscore}{\kern0pt}singleton{\isacharunderscore}{\kern0pt}eq{\isacharunderscore}{\kern0pt}contains{\isacharunderscore}{\kern0pt}powers{\isacharcolon}{\kern0pt}\isanewline
\ \ \isakeyword{fixes}\ n\ {\isacharcolon}{\kern0pt}{\isacharcolon}{\kern0pt}\ nat\isanewline
\ \ \isakeyword{assumes}\ {\isachardoublequoteopen}X\ {\isasymsubseteq}\ G{\isachardoublequoteclose}\ \isakeyword{and}\ {\isachardoublequoteopen}g\ {\isasymin}\ G{\isachardoublequoteclose}\ \isakeyword{and}\ {\isachardoublequoteopen}X\ {\isasymcdots}\ {\isacharbraceleft}{\kern0pt}g{\isacharbraceright}{\kern0pt}\ {\isacharequal}{\kern0pt}\ X{\isachardoublequoteclose}\isanewline
\ \ \isakeyword{shows}\ {\isachardoublequoteopen}X\ {\isasymcdots}\ {\isacharparenleft}{\kern0pt}powers\ g{\isacharparenright}{\kern0pt}\ {\isacharequal}{\kern0pt}\ X{\isachardoublequoteclose}%
\isadelimproof
\ %
\endisadelimproof
%
\isatagproof
\isacommand{using}\isamarkupfalse%
\ powers{\isacharunderscore}{\kern0pt}eq{\isacharunderscore}{\kern0pt}union{\isacharunderscore}{\kern0pt}nat{\isacharunderscore}{\kern0pt}powers\ smul{\isacharunderscore}{\kern0pt}subset{\isacharunderscore}{\kern0pt}Union{\isadigit{2}}\ \isanewline
\ \ \ \ nat{\isacharunderscore}{\kern0pt}powers{\isacharunderscore}{\kern0pt}eq{\isacharunderscore}{\kern0pt}Union\ smul{\isacharunderscore}{\kern0pt}singleton{\isacharunderscore}{\kern0pt}eq{\isacharunderscore}{\kern0pt}contains{\isacharunderscore}{\kern0pt}nat{\isacharunderscore}{\kern0pt}powers\ \isanewline
\ \ \ \ smul{\isacharunderscore}{\kern0pt}singleton{\isacharunderscore}{\kern0pt}eq{\isacharunderscore}{\kern0pt}contains{\isacharunderscore}{\kern0pt}inverse{\isacharunderscore}{\kern0pt}nat{\isacharunderscore}{\kern0pt}powers\ assms\ smul{\isacharunderscore}{\kern0pt}subset{\isacharunderscore}{\kern0pt}Un{\isadigit{2}}\ \isacommand{by}\isamarkupfalse%
\ auto%
\endisatagproof
{\isafoldproof}%
%
\isadelimproof
%
\endisadelimproof
\isanewline
\isanewline
\isacommand{end}\isamarkupfalse%
\isanewline
%
\isadelimtheory
\isanewline
%
\endisadelimtheory
%
\isatagtheory
\isacommand{end}\isamarkupfalse%
%
\endisatagtheory
{\isafoldtheory}%
%
\isadelimtheory
%
\endisadelimtheory
%
\end{isabellebody}%
\endinput
%:%file=~/IsabelleProjects/Generalized_Cauchy_Davenport/Generalized_Cauchy_Davenport_preliminaries.thy%:%
%:%11=1%:%
%:%27=14%:%
%:%28=14%:%
%:%29=15%:%
%:%30=16%:%
%:%31=17%:%
%:%32=18%:%
%:%33=19%:%
%:%47=21%:%
%:%57=23%:%
%:%58=23%:%
%:%59=24%:%
%:%60=25%:%
%:%61=26%:%
%:%62=27%:%
%:%63=28%:%
%:%64=29%:%
%:%71=30%:%
%:%72=30%:%
%:%73=31%:%
%:%74=31%:%
%:%75=32%:%
%:%76=32%:%
%:%77=33%:%
%:%78=33%:%
%:%79=34%:%
%:%80=35%:%
%:%81=35%:%
%:%82=36%:%
%:%83=36%:%
%:%84=37%:%
%:%85=38%:%
%:%86=38%:%
%:%87=39%:%
%:%88=39%:%
%:%89=39%:%
%:%90=40%:%
%:%91=40%:%
%:%92=40%:%
%:%93=41%:%
%:%94=41%:%
%:%95=41%:%
%:%96=42%:%
%:%97=42%:%
%:%98=42%:%
%:%99=43%:%
%:%100=43%:%
%:%101=43%:%
%:%102=44%:%
%:%103=44%:%
%:%104=45%:%
%:%105=45%:%
%:%106=46%:%
%:%107=46%:%
%:%108=47%:%
%:%109=47%:%
%:%110=48%:%
%:%111=48%:%
%:%112=49%:%
%:%113=49%:%
%:%114=49%:%
%:%115=50%:%
%:%116=51%:%
%:%117=51%:%
%:%118=51%:%
%:%119=52%:%
%:%120=52%:%
%:%121=52%:%
%:%122=53%:%
%:%123=53%:%
%:%124=54%:%
%:%125=54%:%
%:%126=55%:%
%:%127=55%:%
%:%128=55%:%
%:%129=56%:%
%:%130=56%:%
%:%131=56%:%
%:%132=57%:%
%:%138=57%:%
%:%141=58%:%
%:%142=59%:%
%:%143=59%:%
%:%144=60%:%
%:%145=61%:%
%:%146=62%:%
%:%147=63%:%
%:%150=64%:%
%:%154=64%:%
%:%155=64%:%
%:%156=65%:%
%:%157=65%:%
%:%162=65%:%
%:%165=66%:%
%:%166=67%:%
%:%167=67%:%
%:%168=68%:%
%:%169=69%:%
%:%176=71%:%
%:%186=73%:%
%:%187=73%:%
%:%188=74%:%
%:%189=75%:%
%:%190=76%:%
%:%191=77%:%
%:%192=77%:%
%:%193=78%:%
%:%194=79%:%
%:%195=79%:%
%:%198=80%:%
%:%202=80%:%
%:%203=80%:%
%:%208=80%:%
%:%211=81%:%
%:%212=82%:%
%:%213=82%:%
%:%216=83%:%
%:%220=83%:%
%:%221=83%:%
%:%226=83%:%
%:%229=84%:%
%:%230=85%:%
%:%231=85%:%
%:%234=86%:%
%:%238=86%:%
%:%239=86%:%
%:%244=86%:%
%:%247=87%:%
%:%248=88%:%
%:%249=88%:%
%:%252=89%:%
%:%256=89%:%
%:%257=89%:%
%:%262=89%:%
%:%265=90%:%
%:%266=91%:%
%:%267=91%:%
%:%270=92%:%
%:%274=92%:%
%:%275=92%:%
%:%280=92%:%
%:%283=93%:%
%:%284=94%:%
%:%285=94%:%
%:%288=95%:%
%:%292=95%:%
%:%293=95%:%
%:%298=95%:%
%:%301=96%:%
%:%302=97%:%
%:%303=97%:%
%:%306=98%:%
%:%310=98%:%
%:%311=98%:%
%:%316=98%:%
%:%319=99%:%
%:%320=100%:%
%:%321=100%:%
%:%324=101%:%
%:%328=101%:%
%:%329=101%:%
%:%334=101%:%
%:%337=102%:%
%:%338=103%:%
%:%339=103%:%
%:%342=104%:%
%:%346=104%:%
%:%347=104%:%
%:%352=104%:%
%:%355=105%:%
%:%356=106%:%
%:%357=106%:%
%:%360=107%:%
%:%364=107%:%
%:%365=107%:%
%:%370=107%:%
%:%373=108%:%
%:%374=109%:%
%:%375=109%:%
%:%378=110%:%
%:%382=110%:%
%:%383=110%:%
%:%388=110%:%
%:%391=111%:%
%:%392=112%:%
%:%393=112%:%
%:%396=113%:%
%:%400=113%:%
%:%401=113%:%
%:%406=113%:%
%:%409=114%:%
%:%410=115%:%
%:%411=115%:%
%:%414=116%:%
%:%418=116%:%
%:%419=116%:%
%:%424=116%:%
%:%427=117%:%
%:%428=118%:%
%:%429=118%:%
%:%432=119%:%
%:%436=119%:%
%:%437=119%:%
%:%442=119%:%
%:%445=120%:%
%:%446=121%:%
%:%447=121%:%
%:%448=122%:%
%:%449=123%:%
%:%452=124%:%
%:%456=124%:%
%:%457=124%:%
%:%458=124%:%
%:%463=124%:%
%:%466=125%:%
%:%467=126%:%
%:%468=126%:%
%:%471=127%:%
%:%475=127%:%
%:%476=127%:%
%:%481=127%:%
%:%484=128%:%
%:%485=129%:%
%:%486=129%:%
%:%489=130%:%
%:%493=130%:%
%:%494=130%:%
%:%499=130%:%
%:%502=131%:%
%:%503=132%:%
%:%504=132%:%
%:%507=133%:%
%:%511=133%:%
%:%512=133%:%
%:%517=133%:%
%:%520=134%:%
%:%521=135%:%
%:%522=135%:%
%:%523=136%:%
%:%526=137%:%
%:%530=137%:%
%:%531=137%:%
%:%536=137%:%
%:%539=138%:%
%:%540=139%:%
%:%541=139%:%
%:%542=140%:%
%:%545=141%:%
%:%549=141%:%
%:%550=141%:%
%:%551=141%:%
%:%556=141%:%
%:%559=142%:%
%:%560=143%:%
%:%561=143%:%
%:%562=144%:%
%:%563=145%:%
%:%566=146%:%
%:%570=146%:%
%:%571=146%:%
%:%572=146%:%
%:%586=148%:%
%:%596=149%:%
%:%597=149%:%
%:%598=150%:%
%:%599=151%:%
%:%600=152%:%
%:%601=153%:%
%:%602=154%:%
%:%603=154%:%
%:%604=155%:%
%:%605=156%:%
%:%607=156%:%
%:%611=156%:%
%:%612=156%:%
%:%613=156%:%
%:%620=156%:%
%:%621=157%:%
%:%622=158%:%
%:%623=158%:%
%:%624=159%:%
%:%625=160%:%
%:%626=161%:%
%:%629=162%:%
%:%633=162%:%
%:%634=162%:%
%:%635=163%:%
%:%641=163%:%
%:%644=164%:%
%:%645=165%:%
%:%646=165%:%
%:%647=166%:%
%:%648=167%:%
%:%655=168%:%
%:%656=168%:%
%:%657=169%:%
%:%658=169%:%
%:%659=170%:%
%:%660=170%:%
%:%661=170%:%
%:%662=170%:%
%:%663=170%:%
%:%664=171%:%
%:%665=171%:%
%:%666=172%:%
%:%667=172%:%
%:%668=173%:%
%:%669=173%:%
%:%670=174%:%
%:%671=174%:%
%:%672=174%:%
%:%673=174%:%
%:%674=174%:%
%:%675=175%:%
%:%676=176%:%
%:%682=176%:%
%:%685=177%:%
%:%686=178%:%
%:%687=178%:%
%:%688=179%:%
%:%689=180%:%
%:%696=181%:%
%:%697=181%:%
%:%698=182%:%
%:%699=182%:%
%:%700=183%:%
%:%701=183%:%
%:%702=183%:%
%:%703=183%:%
%:%704=183%:%
%:%705=184%:%
%:%706=184%:%
%:%707=185%:%
%:%708=185%:%
%:%709=186%:%
%:%710=186%:%
%:%711=187%:%
%:%712=187%:%
%:%713=187%:%
%:%714=188%:%
%:%715=188%:%
%:%716=188%:%
%:%717=189%:%
%:%718=190%:%
%:%719=190%:%
%:%720=190%:%
%:%721=190%:%
%:%722=190%:%
%:%723=191%:%
%:%724=192%:%
%:%730=192%:%
%:%733=193%:%
%:%734=194%:%
%:%735=194%:%
%:%736=195%:%
%:%737=196%:%
%:%739=196%:%
%:%743=196%:%
%:%744=196%:%
%:%745=196%:%
%:%746=197%:%
%:%753=197%:%
%:%754=198%:%
%:%755=199%:%
%:%756=199%:%
%:%757=200%:%
%:%758=201%:%
%:%761=202%:%
%:%765=202%:%
%:%766=202%:%
%:%767=202%:%
%:%772=202%:%
%:%775=203%:%
%:%776=204%:%
%:%777=204%:%
%:%778=205%:%
%:%779=206%:%
%:%780=207%:%
%:%781=207%:%
%:%782=208%:%
%:%783=209%:%
%:%784=209%:%
%:%786=209%:%
%:%790=209%:%
%:%791=209%:%
%:%792=209%:%
%:%799=209%:%
%:%800=210%:%
%:%801=211%:%
%:%802=211%:%
%:%803=212%:%
%:%804=213%:%
%:%805=213%:%
%:%806=214%:%
%:%813=215%:%
%:%814=215%:%
%:%815=216%:%
%:%816=216%:%
%:%817=216%:%
%:%818=217%:%
%:%819=217%:%
%:%820=217%:%
%:%821=217%:%
%:%822=217%:%
%:%823=218%:%
%:%824=218%:%
%:%825=218%:%
%:%826=218%:%
%:%827=219%:%
%:%828=219%:%
%:%829=220%:%
%:%835=220%:%
%:%838=221%:%
%:%839=222%:%
%:%840=222%:%
%:%841=223%:%
%:%848=224%:%
%:%849=224%:%
%:%850=225%:%
%:%851=225%:%
%:%852=225%:%
%:%853=226%:%
%:%854=226%:%
%:%855=226%:%
%:%856=226%:%
%:%857=226%:%
%:%858=227%:%
%:%859=227%:%
%:%860=227%:%
%:%861=228%:%
%:%862=228%:%
%:%863=229%:%
%:%864=229%:%
%:%865=230%:%
%:%866=230%:%
%:%867=230%:%
%:%868=230%:%
%:%869=231%:%
%:%870=231%:%
%:%871=232%:%
%:%872=232%:%
%:%873=233%:%
%:%874=233%:%
%:%875=234%:%
%:%876=234%:%
%:%877=234%:%
%:%878=234%:%
%:%879=235%:%
%:%880=235%:%
%:%881=235%:%
%:%882=235%:%
%:%883=235%:%
%:%884=236%:%
%:%885=236%:%
%:%886=236%:%
%:%887=236%:%
%:%888=236%:%
%:%889=237%:%
%:%890=237%:%
%:%891=238%:%
%:%897=238%:%
%:%900=239%:%
%:%901=240%:%
%:%902=240%:%
%:%903=241%:%
%:%910=242%:%
%:%911=242%:%
%:%912=243%:%
%:%913=243%:%
%:%914=244%:%
%:%915=244%:%
%:%916=244%:%
%:%917=245%:%
%:%918=245%:%
%:%919=246%:%
%:%920=246%:%
%:%921=247%:%
%:%922=247%:%
%:%923=248%:%
%:%929=248%:%
%:%932=249%:%
%:%933=250%:%
%:%934=250%:%
%:%937=251%:%
%:%941=251%:%
%:%942=251%:%
%:%943=251%:%
%:%948=251%:%
%:%951=252%:%
%:%952=253%:%
%:%953=253%:%
%:%954=254%:%
%:%955=255%:%
%:%958=256%:%
%:%962=256%:%
%:%963=256%:%
%:%964=256%:%
%:%969=256%:%
%:%972=257%:%
%:%973=258%:%
%:%974=258%:%
%:%975=259%:%
%:%976=260%:%
%:%979=261%:%
%:%983=261%:%
%:%984=261%:%
%:%985=261%:%
%:%990=261%:%
%:%993=262%:%
%:%994=263%:%
%:%995=263%:%
%:%996=264%:%
%:%997=265%:%
%:%1004=266%:%
%:%1005=266%:%
%:%1006=267%:%
%:%1007=267%:%
%:%1008=267%:%
%:%1009=268%:%
%:%1010=268%:%
%:%1011=268%:%
%:%1012=268%:%
%:%1013=268%:%
%:%1014=269%:%
%:%1015=269%:%
%:%1016=270%:%
%:%1017=270%:%
%:%1018=271%:%
%:%1019=271%:%
%:%1020=272%:%
%:%1021=272%:%
%:%1022=272%:%
%:%1023=272%:%
%:%1024=272%:%
%:%1025=273%:%
%:%1026=273%:%
%:%1027=273%:%
%:%1028=273%:%
%:%1029=274%:%
%:%1030=274%:%
%:%1031=275%:%
%:%1032=275%:%
%:%1033=275%:%
%:%1034=275%:%
%:%1035=275%:%
%:%1036=276%:%
%:%1037=276%:%
%:%1038=277%:%
%:%1039=277%:%
%:%1040=278%:%
%:%1041=278%:%
%:%1042=278%:%
%:%1043=278%:%
%:%1044=279%:%
%:%1045=279%:%
%:%1046=279%:%
%:%1047=279%:%
%:%1048=280%:%
%:%1049=280%:%
%:%1050=281%:%
%:%1051=282%:%
%:%1052=282%:%
%:%1053=282%:%
%:%1054=282%:%
%:%1055=282%:%
%:%1056=283%:%
%:%1057=283%:%
%:%1058=284%:%
%:%1064=284%:%
%:%1067=285%:%
%:%1068=286%:%
%:%1069=286%:%
%:%1070=287%:%
%:%1071=288%:%
%:%1074=289%:%
%:%1078=289%:%
%:%1079=289%:%
%:%1080=290%:%
%:%1085=290%:%
%:%1088=291%:%
%:%1089=292%:%
%:%1090=292%:%
%:%1091=293%:%
%:%1092=294%:%
%:%1095=295%:%
%:%1099=295%:%
%:%1100=295%:%
%:%1101=296%:%
%:%1102=296%:%
%:%1107=296%:%
%:%1110=297%:%
%:%1111=298%:%
%:%1112=298%:%
%:%1113=299%:%
%:%1114=300%:%
%:%1117=301%:%
%:%1121=301%:%
%:%1122=301%:%
%:%1123=302%:%
%:%1124=302%:%
%:%1125=303%:%
%:%1131=303%:%
%:%1134=304%:%
%:%1135=305%:%
%:%1136=305%:%
%:%1137=306%:%
%:%1138=307%:%
%:%1141=308%:%
%:%1145=308%:%
%:%1146=308%:%
%:%1147=308%:%
%:%1148=309%:%
%:%1153=309%:%
%:%1156=310%:%
%:%1157=311%:%
%:%1158=311%:%
%:%1159=312%:%
%:%1160=313%:%
%:%1167=314%:%
%:%1168=314%:%
%:%1169=315%:%
%:%1170=315%:%
%:%1171=315%:%
%:%1172=315%:%
%:%1173=316%:%
%:%1174=316%:%
%:%1175=317%:%
%:%1176=317%:%
%:%1177=318%:%
%:%1178=318%:%
%:%1179=318%:%
%:%1180=319%:%
%:%1181=319%:%
%:%1182=320%:%
%:%1183=320%:%
%:%1184=320%:%
%:%1185=320%:%
%:%1186=321%:%
%:%1192=321%:%
%:%1195=322%:%
%:%1196=323%:%
%:%1197=323%:%
%:%1198=324%:%
%:%1199=325%:%
%:%1202=326%:%
%:%1206=326%:%
%:%1207=326%:%
%:%1208=327%:%
%:%1209=328%:%
%:%1214=328%:%
%:%1217=329%:%
%:%1218=330%:%
%:%1219=330%:%
%:%1220=331%:%
%:%1221=332%:%
%:%1228=333%:%
%:%1229=333%:%
%:%1230=334%:%
%:%1231=334%:%
%:%1232=334%:%
%:%1233=334%:%
%:%1234=335%:%
%:%1235=335%:%
%:%1236=335%:%
%:%1237=335%:%
%:%1238=336%:%
%:%1239=336%:%
%:%1240=337%:%
%:%1241=337%:%
%:%1242=337%:%
%:%1243=337%:%
%:%1244=337%:%
%:%1245=338%:%
%:%1246=339%:%
%:%1252=339%:%
%:%1255=340%:%
%:%1256=341%:%
%:%1257=341%:%
%:%1258=342%:%
%:%1259=343%:%
%:%1262=344%:%
%:%1266=344%:%
%:%1267=344%:%
%:%1268=344%:%
%:%1269=345%:%
%:%1274=345%:%
%:%1277=346%:%
%:%1278=347%:%
%:%1279=347%:%
%:%1280=348%:%
%:%1281=349%:%
%:%1288=350%:%
%:%1289=350%:%
%:%1290=351%:%
%:%1291=351%:%
%:%1292=351%:%
%:%1293=352%:%
%:%1294=352%:%
%:%1295=353%:%
%:%1301=353%:%
%:%1304=354%:%
%:%1305=355%:%
%:%1306=355%:%
%:%1307=356%:%
%:%1308=357%:%
%:%1315=358%:%
%:%1316=358%:%
%:%1317=359%:%
%:%1318=359%:%
%:%1319=359%:%
%:%1320=359%:%
%:%1321=360%:%
%:%1322=360%:%
%:%1323=360%:%
%:%1324=360%:%
%:%1325=360%:%
%:%1326=361%:%
%:%1332=361%:%
%:%1335=362%:%
%:%1336=363%:%
%:%1337=363%:%
%:%1338=364%:%
%:%1339=365%:%
%:%1341=365%:%
%:%1345=365%:%
%:%1346=365%:%
%:%1347=366%:%
%:%1348=366%:%
%:%1362=368%:%
%:%1372=369%:%
%:%1373=369%:%
%:%1374=370%:%
%:%1375=371%:%
%:%1376=372%:%
%:%1377=372%:%
%:%1378=373%:%
%:%1379=374%:%
%:%1387=376%:%
%:%1397=377%:%
%:%1398=377%:%
%:%1399=378%:%
%:%1400=379%:%
%:%1401=380%:%
%:%1402=381%:%
%:%1403=381%:%
%:%1404=382%:%
%:%1411=383%:%
%:%1412=383%:%
%:%1413=384%:%
%:%1414=384%:%
%:%1415=385%:%
%:%1416=385%:%
%:%1417=385%:%
%:%1418=386%:%
%:%1419=386%:%
%:%1420=387%:%
%:%1421=387%:%
%:%1422=387%:%
%:%1423=388%:%
%:%1424=388%:%
%:%1425=389%:%
%:%1426=389%:%
%:%1427=389%:%
%:%1428=390%:%
%:%1429=390%:%
%:%1430=391%:%
%:%1431=391%:%
%:%1436=391%:%
%:%1439=392%:%
%:%1440=393%:%
%:%1441=393%:%
%:%1442=394%:%
%:%1449=395%:%
%:%1450=395%:%
%:%1451=396%:%
%:%1452=396%:%
%:%1453=397%:%
%:%1454=397%:%
%:%1455=397%:%
%:%1456=398%:%
%:%1457=398%:%
%:%1458=399%:%
%:%1459=399%:%
%:%1460=399%:%
%:%1461=400%:%
%:%1462=400%:%
%:%1463=401%:%
%:%1464=401%:%
%:%1465=401%:%
%:%1466=402%:%
%:%1467=402%:%
%:%1468=403%:%
%:%1469=403%:%
%:%1474=403%:%
%:%1477=404%:%
%:%1478=405%:%
%:%1479=405%:%
%:%1480=406%:%
%:%1483=407%:%
%:%1487=407%:%
%:%1488=407%:%
%:%1493=407%:%
%:%1496=408%:%
%:%1497=409%:%
%:%1498=409%:%
%:%1499=410%:%
%:%1502=411%:%
%:%1506=411%:%
%:%1507=411%:%
%:%1512=411%:%
%:%1515=412%:%
%:%1516=413%:%
%:%1517=413%:%
%:%1518=414%:%
%:%1519=415%:%
%:%1520=416%:%
%:%1527=417%:%
%:%1528=417%:%
%:%1529=418%:%
%:%1530=418%:%
%:%1531=419%:%
%:%1532=419%:%
%:%1533=420%:%
%:%1534=420%:%
%:%1535=421%:%
%:%1536=421%:%
%:%1537=421%:%
%:%1538=422%:%
%:%1539=422%:%
%:%1540=422%:%
%:%1541=423%:%
%:%1542=423%:%
%:%1543=424%:%
%:%1549=424%:%
%:%1552=425%:%
%:%1553=426%:%
%:%1554=426%:%
%:%1555=427%:%
%:%1556=428%:%
%:%1557=429%:%
%:%1564=430%:%
%:%1565=430%:%
%:%1566=431%:%
%:%1567=431%:%
%:%1568=432%:%
%:%1569=432%:%
%:%1570=433%:%
%:%1571=433%:%
%:%1572=434%:%
%:%1573=434%:%
%:%1574=434%:%
%:%1575=435%:%
%:%1576=435%:%
%:%1577=435%:%
%:%1578=436%:%
%:%1579=436%:%
%:%1580=437%:%
%:%1586=437%:%
%:%1589=438%:%
%:%1590=439%:%
%:%1591=440%:%
%:%1592=440%:%
%:%1593=441%:%
%:%1594=442%:%
%:%1601=443%:%
%:%1602=443%:%
%:%1603=444%:%
%:%1604=444%:%
%:%1605=445%:%
%:%1606=445%:%
%:%1607=445%:%
%:%1608=445%:%
%:%1609=446%:%
%:%1610=446%:%
%:%1611=446%:%
%:%1612=446%:%
%:%1613=446%:%
%:%1614=447%:%
%:%1615=448%:%
%:%1616=448%:%
%:%1617=448%:%
%:%1618=449%:%
%:%1619=449%:%
%:%1620=450%:%
%:%1621=450%:%
%:%1622=450%:%
%:%1623=451%:%
%:%1624=451%:%
%:%1625=451%:%
%:%1626=451%:%
%:%1627=451%:%
%:%1628=452%:%
%:%1629=452%:%
%:%1630=452%:%
%:%1631=452%:%
%:%1632=452%:%
%:%1633=453%:%
%:%1634=453%:%
%:%1635=454%:%
%:%1636=454%:%
%:%1637=454%:%
%:%1638=454%:%
%:%1639=454%:%
%:%1640=454%:%
%:%1641=455%:%
%:%1647=455%:%
%:%1650=456%:%
%:%1651=457%:%
%:%1652=457%:%
%:%1653=458%:%
%:%1654=459%:%
%:%1661=460%:%
%:%1662=460%:%
%:%1663=461%:%
%:%1664=461%:%
%:%1665=462%:%
%:%1666=462%:%
%:%1667=462%:%
%:%1668=462%:%
%:%1669=463%:%
%:%1670=463%:%
%:%1671=463%:%
%:%1672=463%:%
%:%1673=463%:%
%:%1674=464%:%
%:%1675=464%:%
%:%1676=464%:%
%:%1677=465%:%
%:%1678=465%:%
%:%1679=466%:%
%:%1680=466%:%
%:%1681=466%:%
%:%1682=467%:%
%:%1683=467%:%
%:%1684=467%:%
%:%1685=467%:%
%:%1686=468%:%
%:%1687=468%:%
%:%1688=469%:%
%:%1689=469%:%
%:%1690=469%:%
%:%1691=469%:%
%:%1692=469%:%
%:%1693=470%:%
%:%1694=470%:%
%:%1695=471%:%
%:%1696=471%:%
%:%1697=471%:%
%:%1698=471%:%
%:%1699=471%:%
%:%1700=471%:%
%:%1701=472%:%
%:%1716=474%:%
%:%1726=475%:%
%:%1727=475%:%
%:%1728=476%:%
%:%1729=477%:%
%:%1736=478%:%
%:%1737=478%:%
%:%1738=479%:%
%:%1739=479%:%
%:%1740=480%:%
%:%1741=480%:%
%:%1742=481%:%
%:%1743=481%:%
%:%1744=481%:%
%:%1745=482%:%
%:%1746=482%:%
%:%1747=482%:%
%:%1748=483%:%
%:%1749=483%:%
%:%1750=484%:%
%:%1751=484%:%
%:%1752=484%:%
%:%1753=485%:%
%:%1754=485%:%
%:%1755=486%:%
%:%1756=486%:%
%:%1757=487%:%
%:%1758=487%:%
%:%1759=488%:%
%:%1760=488%:%
%:%1761=489%:%
%:%1762=489%:%
%:%1763=490%:%
%:%1764=490%:%
%:%1765=490%:%
%:%1766=491%:%
%:%1767=491%:%
%:%1768=491%:%
%:%1769=492%:%
%:%1770=492%:%
%:%1771=492%:%
%:%1772=493%:%
%:%1773=493%:%
%:%1774=493%:%
%:%1775=493%:%
%:%1776=494%:%
%:%1777=494%:%
%:%1778=495%:%
%:%1779=495%:%
%:%1780=496%:%
%:%1786=496%:%
%:%1789=497%:%
%:%1790=498%:%
%:%1791=498%:%
%:%1792=499%:%
%:%1793=500%:%
%:%1794=501%:%
%:%1801=502%:%
%:%1802=502%:%
%:%1803=503%:%
%:%1804=503%:%
%:%1805=504%:%
%:%1806=504%:%
%:%1807=504%:%
%:%1808=504%:%
%:%1809=504%:%
%:%1810=505%:%
%:%1811=505%:%
%:%1812=506%:%
%:%1813=506%:%
%:%1814=507%:%
%:%1815=507%:%
%:%1816=508%:%
%:%1817=508%:%
%:%1818=508%:%
%:%1819=509%:%
%:%1820=509%:%
%:%1821=510%:%
%:%1822=510%:%
%:%1823=511%:%
%:%1824=511%:%
%:%1825=512%:%
%:%1826=512%:%
%:%1827=512%:%
%:%1828=513%:%
%:%1829=513%:%
%:%1830=513%:%
%:%1831=513%:%
%:%1832=513%:%
%:%1833=514%:%
%:%1834=514%:%
%:%1835=514%:%
%:%1836=514%:%
%:%1837=514%:%
%:%1838=515%:%
%:%1839=515%:%
%:%1840=515%:%
%:%1841=515%:%
%:%1842=516%:%
%:%1843=516%:%
%:%1844=517%:%
%:%1845=517%:%
%:%1846=518%:%
%:%1847=518%:%
%:%1848=519%:%
%:%1849=519%:%
%:%1850=520%:%
%:%1851=520%:%
%:%1852=521%:%
%:%1853=521%:%
%:%1854=521%:%
%:%1855=522%:%
%:%1856=522%:%
%:%1857=522%:%
%:%1858=522%:%
%:%1859=522%:%
%:%1860=523%:%
%:%1861=523%:%
%:%1862=523%:%
%:%1863=524%:%
%:%1864=524%:%
%:%1865=524%:%
%:%1866=525%:%
%:%1867=525%:%
%:%1868=525%:%
%:%1869=525%:%
%:%1870=526%:%
%:%1871=526%:%
%:%1872=527%:%
%:%1873=527%:%
%:%1874=528%:%
%:%1875=528%:%
%:%1876=529%:%
%:%1877=529%:%
%:%1878=529%:%
%:%1879=529%:%
%:%1880=529%:%
%:%1881=530%:%
%:%1887=530%:%
%:%1890=531%:%
%:%1891=532%:%
%:%1892=532%:%
%:%1893=533%:%
%:%1894=534%:%
%:%1895=535%:%
%:%1902=536%:%
%:%1903=536%:%
%:%1904=537%:%
%:%1905=537%:%
%:%1906=538%:%
%:%1907=538%:%
%:%1908=539%:%
%:%1909=539%:%
%:%1910=540%:%
%:%1911=540%:%
%:%1912=541%:%
%:%1913=541%:%
%:%1914=541%:%
%:%1915=542%:%
%:%1916=542%:%
%:%1917=542%:%
%:%1918=543%:%
%:%1919=543%:%
%:%1920=543%:%
%:%1921=544%:%
%:%1922=544%:%
%:%1923=544%:%
%:%1924=544%:%
%:%1925=544%:%
%:%1926=545%:%
%:%1927=545%:%
%:%1928=546%:%
%:%1929=546%:%
%:%1930=547%:%
%:%1931=547%:%
%:%1932=548%:%
%:%1933=548%:%
%:%1934=549%:%
%:%1935=549%:%
%:%1936=549%:%
%:%1937=550%:%
%:%1938=550%:%
%:%1939=550%:%
%:%1940=550%:%
%:%1941=550%:%
%:%1942=551%:%
%:%1943=551%:%
%:%1944=551%:%
%:%1945=551%:%
%:%1946=551%:%
%:%1947=552%:%
%:%1948=552%:%
%:%1949=553%:%
%:%1950=553%:%
%:%1951=554%:%
%:%1952=554%:%
%:%1953=554%:%
%:%1954=554%:%
%:%1955=554%:%
%:%1956=555%:%
%:%1957=555%:%
%:%1958=555%:%
%:%1959=555%:%
%:%1960=555%:%
%:%1961=556%:%
%:%1967=556%:%
%:%1970=557%:%
%:%1971=558%:%
%:%1972=558%:%
%:%1973=559%:%
%:%1974=560%:%
%:%1975=561%:%
%:%1977=561%:%
%:%1981=561%:%
%:%1982=561%:%
%:%1983=562%:%
%:%1984=563%:%
%:%1985=563%:%
%:%1992=563%:%
%:%1993=564%:%
%:%1994=565%:%
%:%1995=565%:%
%:%1998=566%:%
%:%2003=567%:%

%
\begin{isabellebody}%
\setisabellecontext{Generalized{\isacharunderscore}{\kern0pt}Cauchy{\isacharunderscore}{\kern0pt}Davenport{\isacharunderscore}{\kern0pt}main{\isacharunderscore}{\kern0pt}proof}%
%
\isadelimdocument
%
\endisadelimdocument
%
\isatagdocument
%
\isamarkupsection{Generalized Cauchy--Davenport Theorem: main proof%
}
\isamarkuptrue%
%
\endisatagdocument
{\isafolddocument}%
%
\isadelimdocument
%
\endisadelimdocument
%
\isadelimtheory
%
\endisadelimtheory
%
\isatagtheory
\isacommand{theory}\isamarkupfalse%
\ Generalized{\isacharunderscore}{\kern0pt}Cauchy{\isacharunderscore}{\kern0pt}Davenport{\isacharunderscore}{\kern0pt}main{\isacharunderscore}{\kern0pt}proof\isanewline
\ \ \isakeyword{imports}\isanewline
\ \ Generalized{\isacharunderscore}{\kern0pt}Cauchy{\isacharunderscore}{\kern0pt}Davenport{\isacharunderscore}{\kern0pt}preliminaries\isanewline
\isakeyword{begin}%
\endisatagtheory
{\isafoldtheory}%
%
\isadelimtheory
\isanewline
%
\endisadelimtheory
\isanewline
\isacommand{context}\isamarkupfalse%
\ group\isanewline
\isanewline
\isakeyword{begin}%
\isadelimdocument
%
\endisadelimdocument
%
\isatagdocument
%
\isamarkupsubsection{Definition of the counterexample pair relation in \cite{DeVos2016OnAG}%
}
\isamarkuptrue%
%
\endisatagdocument
{\isafolddocument}%
%
\isadelimdocument
%
\endisadelimdocument
\isacommand{definition}\isamarkupfalse%
\ devos{\isacharunderscore}{\kern0pt}rel\ \isakeyword{where}\ \isanewline
\ \ {\isachardoublequoteopen}devos{\isacharunderscore}{\kern0pt}rel\ {\isacharequal}{\kern0pt}\ {\isacharparenleft}{\kern0pt}{\isasymlambda}\ {\isacharparenleft}{\kern0pt}A{\isacharcomma}{\kern0pt}\ B{\isacharparenright}{\kern0pt}{\isachardot}{\kern0pt}\ card{\isacharparenleft}{\kern0pt}A\ {\isasymcdots}\ B{\isacharparenright}{\kern0pt}{\isacharparenright}{\kern0pt}\ {\isacharless}{\kern0pt}{\isacharasterisk}{\kern0pt}mlex{\isacharasterisk}{\kern0pt}{\isachargreater}{\kern0pt}\ {\isacharparenleft}{\kern0pt}inv{\isacharunderscore}{\kern0pt}image\ {\isacharparenleft}{\kern0pt}{\isacharbraceleft}{\kern0pt}{\isacharparenleft}{\kern0pt}n{\isacharcomma}{\kern0pt}\ m{\isacharparenright}{\kern0pt}{\isachardot}{\kern0pt}\ n\ {\isachargreater}{\kern0pt}\ m{\isacharbraceright}{\kern0pt}\ {\isacharless}{\kern0pt}{\isacharasterisk}{\kern0pt}lex{\isacharasterisk}{\kern0pt}{\isachargreater}{\kern0pt}\ \isanewline
\ \ measure\ {\isacharparenleft}{\kern0pt}{\isasymlambda}\ {\isacharparenleft}{\kern0pt}A{\isacharcomma}{\kern0pt}\ B{\isacharparenright}{\kern0pt}{\isachardot}{\kern0pt}\ card\ A{\isacharparenright}{\kern0pt}{\isacharparenright}{\kern0pt}{\isacharparenright}{\kern0pt}\ {\isacharparenleft}{\kern0pt}{\isasymlambda}\ {\isacharparenleft}{\kern0pt}A{\isacharcomma}{\kern0pt}\ B{\isacharparenright}{\kern0pt}{\isachardot}{\kern0pt}\ {\isacharparenleft}{\kern0pt}card\ A\ {\isacharplus}{\kern0pt}\ card\ B{\isacharcomma}{\kern0pt}\ {\isacharparenleft}{\kern0pt}A{\isacharcomma}{\kern0pt}\ B{\isacharparenright}{\kern0pt}{\isacharparenright}{\kern0pt}{\isacharparenright}{\kern0pt}{\isachardoublequoteclose}%
\isadelimdocument
%
\endisadelimdocument
%
\isatagdocument
%
\isamarkupsubsection{Lemmas about the counterexample pair relation in \cite{DeVos2016OnAG}%
}
\isamarkuptrue%
%
\endisatagdocument
{\isafolddocument}%
%
\isadelimdocument
%
\endisadelimdocument
\isacommand{lemma}\isamarkupfalse%
\ devos{\isacharunderscore}{\kern0pt}rel{\isacharunderscore}{\kern0pt}iff{\isacharcolon}{\kern0pt}\ \isanewline
\ \ {\isachardoublequoteopen}{\isacharparenleft}{\kern0pt}{\isacharparenleft}{\kern0pt}A{\isacharcomma}{\kern0pt}\ B{\isacharparenright}{\kern0pt}{\isacharcomma}{\kern0pt}\ {\isacharparenleft}{\kern0pt}C{\isacharcomma}{\kern0pt}\ D{\isacharparenright}{\kern0pt}{\isacharparenright}{\kern0pt}\ {\isasymin}\ devos{\isacharunderscore}{\kern0pt}rel\ {\isasymlongleftrightarrow}\ card{\isacharparenleft}{\kern0pt}A\ {\isasymcdots}\ B{\isacharparenright}{\kern0pt}\ {\isacharless}{\kern0pt}\ card{\isacharparenleft}{\kern0pt}C\ {\isasymcdots}\ D{\isacharparenright}{\kern0pt}\ {\isasymor}\ \isanewline
\ \ {\isacharparenleft}{\kern0pt}card{\isacharparenleft}{\kern0pt}A\ {\isasymcdots}\ B{\isacharparenright}{\kern0pt}\ {\isacharequal}{\kern0pt}\ card{\isacharparenleft}{\kern0pt}C\ {\isasymcdots}\ D{\isacharparenright}{\kern0pt}\ {\isasymand}\ card\ A\ {\isacharplus}{\kern0pt}\ card\ B\ {\isachargreater}{\kern0pt}\ card\ C\ {\isacharplus}{\kern0pt}\ card\ D{\isacharparenright}{\kern0pt}\ {\isasymor}\isanewline
\ \ {\isacharparenleft}{\kern0pt}card{\isacharparenleft}{\kern0pt}A\ {\isasymcdots}\ B{\isacharparenright}{\kern0pt}\ {\isacharequal}{\kern0pt}\ card{\isacharparenleft}{\kern0pt}C\ {\isasymcdots}\ D{\isacharparenright}{\kern0pt}\ {\isasymand}\ card\ A\ {\isacharplus}{\kern0pt}\ card\ B\ {\isacharequal}{\kern0pt}\ card\ C\ {\isacharplus}{\kern0pt}\ card\ D\ {\isasymand}\ card\ A\ {\isacharless}{\kern0pt}\ card\ C{\isacharparenright}{\kern0pt}{\isachardoublequoteclose}\isanewline
%
\isadelimproof
\ \ %
\endisadelimproof
%
\isatagproof
\isacommand{using}\isamarkupfalse%
\ devos{\isacharunderscore}{\kern0pt}rel{\isacharunderscore}{\kern0pt}def\ mlex{\isacharunderscore}{\kern0pt}iff{\isacharbrackleft}{\kern0pt}of\ {\isacharunderscore}{\kern0pt}\ {\isacharunderscore}{\kern0pt}\ {\isachardoublequoteopen}{\isasymlambda}\ {\isacharparenleft}{\kern0pt}A{\isacharcomma}{\kern0pt}\ B{\isacharparenright}{\kern0pt}{\isachardot}{\kern0pt}\ card{\isacharparenleft}{\kern0pt}A\ {\isasymcdots}\ B{\isacharparenright}{\kern0pt}{\isachardoublequoteclose}{\isacharbrackright}{\kern0pt}\ \isacommand{by}\isamarkupfalse%
\ fastforce%
\endisatagproof
{\isafoldproof}%
%
\isadelimproof
\isanewline
%
\endisadelimproof
\isanewline
\isacommand{lemma}\isamarkupfalse%
\ devos{\isacharunderscore}{\kern0pt}rel{\isacharunderscore}{\kern0pt}le{\isacharunderscore}{\kern0pt}smul{\isacharcolon}{\kern0pt}\isanewline
\ \ {\isachardoublequoteopen}{\isacharparenleft}{\kern0pt}{\isacharparenleft}{\kern0pt}A{\isacharcomma}{\kern0pt}\ B{\isacharparenright}{\kern0pt}{\isacharcomma}{\kern0pt}\ {\isacharparenleft}{\kern0pt}C{\isacharcomma}{\kern0pt}\ D{\isacharparenright}{\kern0pt}{\isacharparenright}{\kern0pt}\ {\isasymin}\ devos{\isacharunderscore}{\kern0pt}rel\ {\isasymLongrightarrow}\ card{\isacharparenleft}{\kern0pt}A\ {\isasymcdots}\ B{\isacharparenright}{\kern0pt}\ {\isasymle}\ card{\isacharparenleft}{\kern0pt}C\ {\isasymcdots}\ D{\isacharparenright}{\kern0pt}{\isachardoublequoteclose}\isanewline
%
\isadelimproof
\ \ %
\endisadelimproof
%
\isatagproof
\isacommand{using}\isamarkupfalse%
\ devos{\isacharunderscore}{\kern0pt}rel{\isacharunderscore}{\kern0pt}iff\ \isacommand{by}\isamarkupfalse%
\ fastforce%
\endisatagproof
{\isafoldproof}%
%
\isadelimproof
%
\endisadelimproof
%
\begin{isamarkuptext}%
Lemma stating that the above relation due to DeVos is well-founded%
\end{isamarkuptext}\isamarkuptrue%
\isacommand{lemma}\isamarkupfalse%
\ devos{\isacharunderscore}{\kern0pt}rel{\isacharunderscore}{\kern0pt}wf\ {\isacharcolon}{\kern0pt}\ {\isachardoublequoteopen}wf\ {\isacharparenleft}{\kern0pt}Restr\ devos{\isacharunderscore}{\kern0pt}rel\ \isanewline
\ \ {\isacharbraceleft}{\kern0pt}{\isacharparenleft}{\kern0pt}A{\isacharcomma}{\kern0pt}\ B{\isacharparenright}{\kern0pt}{\isachardot}{\kern0pt}\ finite\ A\ {\isasymand}\ A\ {\isasymnoteq}\ {\isacharbraceleft}{\kern0pt}{\isacharbraceright}{\kern0pt}\ {\isasymand}\ A\ {\isasymsubseteq}\ G\ {\isasymand}\ finite\ B\ {\isasymand}\ B\ {\isasymnoteq}\ {\isacharbraceleft}{\kern0pt}{\isacharbraceright}{\kern0pt}\ {\isasymand}\ B\ {\isasymsubseteq}\ G{\isacharbraceright}{\kern0pt}{\isacharparenright}{\kern0pt}{\isachardoublequoteclose}\ {\isacharparenleft}{\kern0pt}\isakeyword{is}\ {\isachardoublequoteopen}wf\ {\isacharparenleft}{\kern0pt}Restr\ devos{\isacharunderscore}{\kern0pt}rel\ {\isacharquery}{\kern0pt}fin{\isacharparenright}{\kern0pt}{\isachardoublequoteclose}{\isacharparenright}{\kern0pt}\isanewline
%
\isadelimproof
%
\endisadelimproof
%
\isatagproof
\isacommand{proof}\isamarkupfalse%
{\isacharminus}{\kern0pt}\isanewline
\ \ \isacommand{define}\isamarkupfalse%
\ f\ \isakeyword{where}\ {\isachardoublequoteopen}f\ {\isacharequal}{\kern0pt}\ {\isacharparenleft}{\kern0pt}{\isasymlambda}\ {\isacharparenleft}{\kern0pt}A{\isacharcomma}{\kern0pt}\ B{\isacharparenright}{\kern0pt}{\isachardot}{\kern0pt}\ card{\isacharparenleft}{\kern0pt}A{\isasymcdots}B{\isacharparenright}{\kern0pt}{\isacharparenright}{\kern0pt}{\isachardoublequoteclose}\isanewline
\ \ \isacommand{define}\isamarkupfalse%
\ g\ \isakeyword{where}\ {\isachardoublequoteopen}g\ {\isacharequal}{\kern0pt}\ {\isacharparenleft}{\kern0pt}{\isasymlambda}\ {\isacharparenleft}{\kern0pt}A\ {\isacharcolon}{\kern0pt}{\isacharcolon}{\kern0pt}\ {\isacharprime}{\kern0pt}a\ set{\isacharcomma}{\kern0pt}\ B\ {\isacharcolon}{\kern0pt}{\isacharcolon}{\kern0pt}\ {\isacharprime}{\kern0pt}a\ set{\isacharparenright}{\kern0pt}{\isachardot}{\kern0pt}\ {\isacharparenleft}{\kern0pt}card\ A\ {\isacharplus}{\kern0pt}\ card\ B{\isacharcomma}{\kern0pt}\ {\isacharparenleft}{\kern0pt}A{\isacharcomma}{\kern0pt}\ B{\isacharparenright}{\kern0pt}{\isacharparenright}{\kern0pt}{\isacharparenright}{\kern0pt}{\isachardoublequoteclose}\isanewline
\ \ \isacommand{define}\isamarkupfalse%
\ h\ \isakeyword{where}\ {\isachardoublequoteopen}h\ {\isacharequal}{\kern0pt}\ {\isacharparenleft}{\kern0pt}{\isasymlambda}\ {\isacharparenleft}{\kern0pt}A\ {\isacharcolon}{\kern0pt}{\isacharcolon}{\kern0pt}\ {\isacharprime}{\kern0pt}a\ set{\isacharcomma}{\kern0pt}\ B\ {\isacharcolon}{\kern0pt}{\isacharcolon}{\kern0pt}\ {\isacharprime}{\kern0pt}a\ set{\isacharparenright}{\kern0pt}{\isachardot}{\kern0pt}\ card\ A\ {\isacharplus}{\kern0pt}\ card\ B{\isacharparenright}{\kern0pt}{\isachardoublequoteclose}\isanewline
\ \ \isacommand{define}\isamarkupfalse%
\ s\ \isakeyword{where}\ {\isachardoublequoteopen}s\ {\isacharequal}{\kern0pt}\ {\isacharparenleft}{\kern0pt}{\isacharbraceleft}{\kern0pt}{\isacharparenleft}{\kern0pt}n\ {\isacharcolon}{\kern0pt}{\isacharcolon}{\kern0pt}\ nat{\isacharcomma}{\kern0pt}\ m\ {\isacharcolon}{\kern0pt}{\isacharcolon}{\kern0pt}\ nat{\isacharparenright}{\kern0pt}{\isachardot}{\kern0pt}\ n\ {\isachargreater}{\kern0pt}\ m{\isacharbraceright}{\kern0pt}\ {\isacharless}{\kern0pt}{\isacharasterisk}{\kern0pt}lex{\isacharasterisk}{\kern0pt}{\isachargreater}{\kern0pt}\ measure\ {\isacharparenleft}{\kern0pt}{\isasymlambda}\ {\isacharparenleft}{\kern0pt}A\ {\isacharcolon}{\kern0pt}{\isacharcolon}{\kern0pt}\ {\isacharprime}{\kern0pt}a\ set{\isacharcomma}{\kern0pt}\ B\ {\isacharcolon}{\kern0pt}{\isacharcolon}{\kern0pt}\ {\isacharprime}{\kern0pt}a\ set{\isacharparenright}{\kern0pt}{\isachardot}{\kern0pt}\ card\ A{\isacharparenright}{\kern0pt}{\isacharparenright}{\kern0pt}{\isachardoublequoteclose}\isanewline
\ \ \isacommand{have}\isamarkupfalse%
\ hle{\isadigit{2}}f{\isacharcolon}{\kern0pt}\ {\isachardoublequoteopen}{\isasymAnd}\ x{\isachardot}{\kern0pt}\ x\ {\isasymin}\ {\isacharquery}{\kern0pt}fin\ {\isasymLongrightarrow}\ h\ x\ {\isasymle}\ {\isadigit{2}}\ {\isacharasterisk}{\kern0pt}\ f\ x{\isachardoublequoteclose}\isanewline
\ \ \isacommand{proof}\isamarkupfalse%
{\isacharminus}{\kern0pt}\isanewline
\ \ \ \ \isacommand{fix}\isamarkupfalse%
\ x\ \isacommand{assume}\isamarkupfalse%
\ hx{\isacharcolon}{\kern0pt}\ {\isachardoublequoteopen}x\ {\isasymin}\ {\isacharquery}{\kern0pt}fin{\isachardoublequoteclose}\isanewline
\ \ \ \ \isacommand{then}\isamarkupfalse%
\ \isacommand{obtain}\isamarkupfalse%
\ A\ B\ \isakeyword{where}\ hxAB{\isacharcolon}{\kern0pt}\ {\isachardoublequoteopen}x\ {\isacharequal}{\kern0pt}\ {\isacharparenleft}{\kern0pt}A{\isacharcomma}{\kern0pt}\ B{\isacharparenright}{\kern0pt}{\isachardoublequoteclose}\ \isacommand{by}\isamarkupfalse%
\ blast\isanewline
\ \ \ \ \isacommand{then}\isamarkupfalse%
\ \isacommand{have}\isamarkupfalse%
\ {\isachardoublequoteopen}card\ A\ {\isasymle}\ card\ {\isacharparenleft}{\kern0pt}A\ {\isasymcdots}\ B{\isacharparenright}{\kern0pt}{\isachardoublequoteclose}\ \isakeyword{and}\ {\isachardoublequoteopen}card\ B\ {\isasymle}\ card\ {\isacharparenleft}{\kern0pt}A\ {\isasymcdots}\ B{\isacharparenright}{\kern0pt}{\isachardoublequoteclose}\ \isanewline
\ \ \ \ \ \ \isacommand{using}\isamarkupfalse%
\ card{\isacharunderscore}{\kern0pt}le{\isacharunderscore}{\kern0pt}smul{\isacharunderscore}{\kern0pt}left\ card{\isacharunderscore}{\kern0pt}le{\isacharunderscore}{\kern0pt}smul{\isacharunderscore}{\kern0pt}right\ hx\ \isacommand{by}\isamarkupfalse%
\ auto\isanewline
\ \ \ \ \isacommand{then}\isamarkupfalse%
\ \isacommand{show}\isamarkupfalse%
\ {\isachardoublequoteopen}h\ x\ {\isasymle}\ {\isadigit{2}}\ {\isacharasterisk}{\kern0pt}\ f\ x{\isachardoublequoteclose}\ \isacommand{using}\isamarkupfalse%
\ hxAB\ h{\isacharunderscore}{\kern0pt}def\ f{\isacharunderscore}{\kern0pt}def\ \isacommand{by}\isamarkupfalse%
\ force\isanewline
\ \ \isacommand{qed}\isamarkupfalse%
\isanewline
\ \ \isacommand{have}\isamarkupfalse%
\ {\isachardoublequoteopen}wf\ {\isacharparenleft}{\kern0pt}Restr\ {\isacharparenleft}{\kern0pt}measure\ f{\isacharparenright}{\kern0pt}\ {\isacharquery}{\kern0pt}fin{\isacharparenright}{\kern0pt}{\isachardoublequoteclose}\ \isacommand{by}\isamarkupfalse%
\ {\isacharparenleft}{\kern0pt}simp\ add{\isacharcolon}{\kern0pt}\ wf{\isacharunderscore}{\kern0pt}Int{\isadigit{1}}{\isacharparenright}{\kern0pt}\isanewline
\ \ \isacommand{moreover}\isamarkupfalse%
\ \isacommand{have}\isamarkupfalse%
\ {\isachardoublequoteopen}{\isasymAnd}\ a{\isachardot}{\kern0pt}\ a\ {\isasymin}\ range\ f\ {\isasymLongrightarrow}\ wf\ {\isacharparenleft}{\kern0pt}Restr\ {\isacharparenleft}{\kern0pt}Restr\ {\isacharparenleft}{\kern0pt}inv{\isacharunderscore}{\kern0pt}image\ s\ g{\isacharparenright}{\kern0pt}\ {\isacharbraceleft}{\kern0pt}x{\isachardot}{\kern0pt}\ f\ x\ {\isacharequal}{\kern0pt}\ a{\isacharbraceright}{\kern0pt}{\isacharparenright}{\kern0pt}\ {\isacharquery}{\kern0pt}fin{\isacharparenright}{\kern0pt}{\isachardoublequoteclose}\isanewline
\ \ \isacommand{proof}\isamarkupfalse%
{\isacharminus}{\kern0pt}\isanewline
\ \ \ \ \isacommand{fix}\isamarkupfalse%
\ y\ \isacommand{assume}\isamarkupfalse%
\ {\isachardoublequoteopen}y\ {\isasymin}\ \ range\ f{\isachardoublequoteclose}\isanewline
\ \ \ \ \isacommand{then}\isamarkupfalse%
\ \isacommand{show}\isamarkupfalse%
\ {\isachardoublequoteopen}wf\ {\isacharparenleft}{\kern0pt}Restr\ {\isacharparenleft}{\kern0pt}Restr\ {\isacharparenleft}{\kern0pt}inv{\isacharunderscore}{\kern0pt}image\ s\ g{\isacharparenright}{\kern0pt}\ {\isacharbraceleft}{\kern0pt}x{\isachardot}{\kern0pt}\ f\ x\ {\isacharequal}{\kern0pt}\ y{\isacharbraceright}{\kern0pt}{\isacharparenright}{\kern0pt}\ {\isacharquery}{\kern0pt}fin{\isacharparenright}{\kern0pt}{\isachardoublequoteclose}\isanewline
\ \ \ \ \isacommand{proof}\isamarkupfalse%
{\isacharminus}{\kern0pt}\isanewline
\ \ \ \ \ \ \isacommand{have}\isamarkupfalse%
\ {\isachardoublequoteopen}Restr\ {\isacharparenleft}{\kern0pt}{\isacharbraceleft}{\kern0pt}x{\isachardot}{\kern0pt}\ f\ x\ {\isacharequal}{\kern0pt}\ y{\isacharbraceright}{\kern0pt}\ {\isasymtimes}\ {\isacharbraceleft}{\kern0pt}x{\isachardot}{\kern0pt}\ f\ x\ {\isacharequal}{\kern0pt}\ y{\isacharbraceright}{\kern0pt}\ {\isasyminter}\ {\isacharparenleft}{\kern0pt}inv{\isacharunderscore}{\kern0pt}image\ s\ g{\isacharparenright}{\kern0pt}{\isacharparenright}{\kern0pt}\ {\isacharquery}{\kern0pt}fin\ {\isasymsubseteq}\ \isanewline
\ \ \ \ \ \ \ \ Restr\ {\isacharparenleft}{\kern0pt}{\isacharparenleft}{\kern0pt}{\isacharparenleft}{\kern0pt}{\isasymlambda}\ x{\isachardot}{\kern0pt}\ {\isadigit{2}}\ {\isacharasterisk}{\kern0pt}\ f\ x\ {\isacharminus}{\kern0pt}\ h\ x{\isacharparenright}{\kern0pt}\ {\isacharless}{\kern0pt}{\isacharasterisk}{\kern0pt}mlex{\isacharasterisk}{\kern0pt}{\isachargreater}{\kern0pt}\ measure\ {\isacharparenleft}{\kern0pt}{\isasymlambda}\ {\isacharparenleft}{\kern0pt}A\ {\isacharcolon}{\kern0pt}{\isacharcolon}{\kern0pt}\ {\isacharprime}{\kern0pt}a\ set{\isacharcomma}{\kern0pt}\ B\ {\isacharcolon}{\kern0pt}{\isacharcolon}{\kern0pt}\ {\isacharprime}{\kern0pt}a\ set{\isacharparenright}{\kern0pt}{\isachardot}{\kern0pt}\ card\ A{\isacharparenright}{\kern0pt}{\isacharparenright}{\kern0pt}\ {\isasyminter}\ \isanewline
\ \ \ \ \ \ \ \ {\isacharbraceleft}{\kern0pt}x{\isachardot}{\kern0pt}\ f\ x\ {\isacharequal}{\kern0pt}\ y{\isacharbraceright}{\kern0pt}\ {\isasymtimes}\ {\isacharbraceleft}{\kern0pt}x{\isachardot}{\kern0pt}\ f\ x\ {\isacharequal}{\kern0pt}\ y{\isacharbraceright}{\kern0pt}{\isacharparenright}{\kern0pt}\ {\isacharquery}{\kern0pt}fin{\isachardoublequoteclose}\isanewline
\ \ \ \ \ \ \isacommand{proof}\isamarkupfalse%
\isanewline
\ \ \ \ \ \ \ \ \isacommand{fix}\isamarkupfalse%
\ z\ \isacommand{assume}\isamarkupfalse%
\ hz{\isacharcolon}{\kern0pt}\ {\isachardoublequoteopen}z\ {\isasymin}\ Restr\ {\isacharparenleft}{\kern0pt}{\isacharbraceleft}{\kern0pt}x{\isachardot}{\kern0pt}\ f\ x\ {\isacharequal}{\kern0pt}\ y{\isacharbraceright}{\kern0pt}\ {\isasymtimes}\ {\isacharbraceleft}{\kern0pt}x{\isachardot}{\kern0pt}\ f\ x\ {\isacharequal}{\kern0pt}\ y{\isacharbraceright}{\kern0pt}\ {\isasyminter}\ {\isacharparenleft}{\kern0pt}inv{\isacharunderscore}{\kern0pt}image\ s\ g{\isacharparenright}{\kern0pt}{\isacharparenright}{\kern0pt}\ {\isacharquery}{\kern0pt}fin{\isachardoublequoteclose}\isanewline
\ \ \ \ \ \ \ \ \isacommand{then}\isamarkupfalse%
\ \isacommand{obtain}\isamarkupfalse%
\ a\ b\ \isakeyword{where}\ hzab{\isacharcolon}{\kern0pt}\ {\isachardoublequoteopen}z\ {\isacharequal}{\kern0pt}\ {\isacharparenleft}{\kern0pt}a{\isacharcomma}{\kern0pt}\ b{\isacharparenright}{\kern0pt}{\isachardoublequoteclose}\ \isakeyword{and}\ {\isachardoublequoteopen}f\ a\ {\isacharequal}{\kern0pt}\ y{\isachardoublequoteclose}\ \isakeyword{and}\ {\isachardoublequoteopen}f\ b\ {\isacharequal}{\kern0pt}\ y{\isachardoublequoteclose}\ \isakeyword{and}\ \isanewline
\ \ \ \ \ \ \ \ \ \ {\isachardoublequoteopen}h\ a\ {\isachargreater}{\kern0pt}\ h\ b\ {\isasymor}\ h\ a\ {\isacharequal}{\kern0pt}\ h\ b\ {\isasymand}\ {\isacharparenleft}{\kern0pt}a{\isacharcomma}{\kern0pt}\ b{\isacharparenright}{\kern0pt}\ {\isasymin}\ measure\ {\isacharparenleft}{\kern0pt}{\isasymlambda}\ {\isacharparenleft}{\kern0pt}A{\isacharcomma}{\kern0pt}\ B{\isacharparenright}{\kern0pt}{\isachardot}{\kern0pt}\ card\ A{\isacharparenright}{\kern0pt}{\isachardoublequoteclose}\ \isakeyword{and}\ \isanewline
\ \ \ \ \ \ \ \ \ \ {\isachardoublequoteopen}a\ {\isasymin}\ {\isacharquery}{\kern0pt}fin{\isachardoublequoteclose}\ \isakeyword{and}\ {\isachardoublequoteopen}b\ {\isasymin}\ {\isacharquery}{\kern0pt}fin{\isachardoublequoteclose}\isanewline
\ \ \ \ \ \ \ \ \ \ \isacommand{using}\isamarkupfalse%
\ s{\isacharunderscore}{\kern0pt}def\ g{\isacharunderscore}{\kern0pt}def\ h{\isacharunderscore}{\kern0pt}def\ \isacommand{by}\isamarkupfalse%
\ force\isanewline
\ \ \ \ \ \ \ \ \isacommand{then}\isamarkupfalse%
\ \isacommand{obtain}\isamarkupfalse%
\ {\isachardoublequoteopen}{\isadigit{2}}\ {\isacharasterisk}{\kern0pt}\ f\ a\ {\isacharminus}{\kern0pt}\ h\ a\ {\isacharless}{\kern0pt}\ {\isadigit{2}}\ {\isacharasterisk}{\kern0pt}\ f\ b\ {\isacharminus}{\kern0pt}\ h\ b\ {\isasymor}\ \isanewline
\ \ \ \ \ \ \ \ \ \ {\isadigit{2}}\ {\isacharasterisk}{\kern0pt}\ f\ a\ {\isacharminus}{\kern0pt}\ h\ a\ {\isacharequal}{\kern0pt}\ {\isadigit{2}}\ {\isacharasterisk}{\kern0pt}\ f\ b\ {\isacharminus}{\kern0pt}\ h\ b\ {\isasymand}\ {\isacharparenleft}{\kern0pt}a{\isacharcomma}{\kern0pt}\ b{\isacharparenright}{\kern0pt}\ {\isasymin}\ measure\ {\isacharparenleft}{\kern0pt}{\isasymlambda}\ {\isacharparenleft}{\kern0pt}A{\isacharcomma}{\kern0pt}\ B{\isacharparenright}{\kern0pt}{\isachardot}{\kern0pt}\ card\ A{\isacharparenright}{\kern0pt}{\isachardoublequoteclose}\ \isanewline
\ \ \ \ \ \ \ \ \ \ \isacommand{using}\isamarkupfalse%
\ hle{\isadigit{2}}f\ \isacommand{by}\isamarkupfalse%
\ {\isacharparenleft}{\kern0pt}smt\ {\isacharparenleft}{\kern0pt}verit{\isacharparenright}{\kern0pt}\ diff{\isacharunderscore}{\kern0pt}less{\isacharunderscore}{\kern0pt}mono{\isadigit{2}}\ le{\isacharunderscore}{\kern0pt}antisym\ nat{\isacharunderscore}{\kern0pt}less{\isacharunderscore}{\kern0pt}le{\isacharparenright}{\kern0pt}\isanewline
\ \ \ \ \ \ \ \ \isacommand{then}\isamarkupfalse%
\ \isacommand{show}\isamarkupfalse%
\ {\isachardoublequoteopen}z\ {\isasymin}\ Restr\ {\isacharparenleft}{\kern0pt}{\isacharparenleft}{\kern0pt}{\isacharparenleft}{\kern0pt}{\isasymlambda}\ x{\isachardot}{\kern0pt}\ {\isadigit{2}}\ {\isacharasterisk}{\kern0pt}\ f\ x\ {\isacharminus}{\kern0pt}\ h\ x{\isacharparenright}{\kern0pt}\ {\isacharless}{\kern0pt}{\isacharasterisk}{\kern0pt}mlex{\isacharasterisk}{\kern0pt}{\isachargreater}{\kern0pt}\ measure\ {\isacharparenleft}{\kern0pt}{\isasymlambda}\ {\isacharparenleft}{\kern0pt}A{\isacharcomma}{\kern0pt}\ B{\isacharparenright}{\kern0pt}{\isachardot}{\kern0pt}\ card\ A{\isacharparenright}{\kern0pt}{\isacharparenright}{\kern0pt}\ {\isasyminter}\ \isanewline
\ \ \ \ \ \ \ \ {\isacharbraceleft}{\kern0pt}x{\isachardot}{\kern0pt}\ f\ x\ {\isacharequal}{\kern0pt}\ y{\isacharbraceright}{\kern0pt}\ {\isasymtimes}\ {\isacharbraceleft}{\kern0pt}x{\isachardot}{\kern0pt}\ f\ x\ {\isacharequal}{\kern0pt}\ y{\isacharbraceright}{\kern0pt}{\isacharparenright}{\kern0pt}\ {\isacharquery}{\kern0pt}fin{\isachardoublequoteclose}\ \isacommand{using}\isamarkupfalse%
\ hzab\ hz\ \isacommand{by}\isamarkupfalse%
\ {\isacharparenleft}{\kern0pt}simp\ add{\isacharcolon}{\kern0pt}\ mlex{\isacharunderscore}{\kern0pt}iff{\isacharparenright}{\kern0pt}\isanewline
\ \ \ \ \ \ \isacommand{qed}\isamarkupfalse%
\isanewline
\ \ \ \ \ \ \isacommand{moreover}\isamarkupfalse%
\ \isacommand{have}\isamarkupfalse%
\ {\isachardoublequoteopen}wf\ {\isacharparenleft}{\kern0pt}{\isacharparenleft}{\kern0pt}{\isasymlambda}\ x{\isachardot}{\kern0pt}\ {\isadigit{2}}\ {\isacharasterisk}{\kern0pt}\ f\ x\ {\isacharminus}{\kern0pt}\ h\ x{\isacharparenright}{\kern0pt}\ {\isacharless}{\kern0pt}{\isacharasterisk}{\kern0pt}mlex{\isacharasterisk}{\kern0pt}{\isachargreater}{\kern0pt}\ measure\ {\isacharparenleft}{\kern0pt}{\isasymlambda}\ {\isacharparenleft}{\kern0pt}A{\isacharcomma}{\kern0pt}\ B{\isacharparenright}{\kern0pt}{\isachardot}{\kern0pt}\ card\ A{\isacharparenright}{\kern0pt}{\isacharparenright}{\kern0pt}{\isachardoublequoteclose}\isanewline
\ \ \ \ \ \ \ \ \isacommand{by}\isamarkupfalse%
\ {\isacharparenleft}{\kern0pt}simp\ add{\isacharcolon}{\kern0pt}\ wf{\isacharunderscore}{\kern0pt}mlex{\isacharparenright}{\kern0pt}\isanewline
\ \ \ \ \ \ \isacommand{ultimately}\isamarkupfalse%
\ \isacommand{show}\isamarkupfalse%
\ {\isacharquery}{\kern0pt}thesis\ \isacommand{by}\isamarkupfalse%
\ {\isacharparenleft}{\kern0pt}simp\ add{\isacharcolon}{\kern0pt}\ Int{\isacharunderscore}{\kern0pt}commute\ wf{\isacharunderscore}{\kern0pt}Int{\isadigit{1}}\ wf{\isacharunderscore}{\kern0pt}subset{\isacharparenright}{\kern0pt}\isanewline
\ \ \ \ \isacommand{qed}\isamarkupfalse%
\isanewline
\ \ \isacommand{qed}\isamarkupfalse%
\isanewline
\ \ \isacommand{moreover}\isamarkupfalse%
\ \isacommand{have}\isamarkupfalse%
\ {\isachardoublequoteopen}trans\ {\isacharparenleft}{\kern0pt}{\isacharquery}{\kern0pt}fin\ {\isasymtimes}\ {\isacharquery}{\kern0pt}fin{\isacharparenright}{\kern0pt}{\isachardoublequoteclose}\ \isacommand{using}\isamarkupfalse%
\ trans{\isacharunderscore}{\kern0pt}def\ \isacommand{by}\isamarkupfalse%
\ fast\isanewline
\ \ \isacommand{ultimately}\isamarkupfalse%
\ \isacommand{have}\isamarkupfalse%
\ {\isachardoublequoteopen}wf\ {\isacharparenleft}{\kern0pt}Restr\ {\isacharparenleft}{\kern0pt}inv{\isacharunderscore}{\kern0pt}image\ {\isacharparenleft}{\kern0pt}less{\isacharunderscore}{\kern0pt}than\ {\isacharless}{\kern0pt}{\isacharasterisk}{\kern0pt}lex{\isacharasterisk}{\kern0pt}{\isachargreater}{\kern0pt}\ s{\isacharparenright}{\kern0pt}\ {\isacharparenleft}{\kern0pt}{\isasymlambda}\ c{\isachardot}{\kern0pt}\ {\isacharparenleft}{\kern0pt}f\ c{\isacharcomma}{\kern0pt}\ g\ c{\isacharparenright}{\kern0pt}{\isacharparenright}{\kern0pt}{\isacharparenright}{\kern0pt}\ {\isacharquery}{\kern0pt}fin{\isacharparenright}{\kern0pt}{\isachardoublequoteclose}\ \isanewline
\ \ \ \ \isacommand{using}\isamarkupfalse%
\ wf{\isacharunderscore}{\kern0pt}prod{\isacharunderscore}{\kern0pt}lex{\isacharunderscore}{\kern0pt}fibers{\isacharunderscore}{\kern0pt}inter{\isacharbrackleft}{\kern0pt}of\ {\isachardoublequoteopen}less{\isacharunderscore}{\kern0pt}than{\isachardoublequoteclose}\ {\isachardoublequoteopen}f{\isachardoublequoteclose}\ {\isachardoublequoteopen}{\isacharquery}{\kern0pt}fin\ {\isasymtimes}\ {\isacharquery}{\kern0pt}fin{\isachardoublequoteclose}\ {\isachardoublequoteopen}s{\isachardoublequoteclose}\ {\isachardoublequoteopen}g{\isachardoublequoteclose}{\isacharbrackright}{\kern0pt}\ measure{\isacharunderscore}{\kern0pt}def\isanewline
\ \ \ \ \isacommand{by}\isamarkupfalse%
\ {\isacharparenleft}{\kern0pt}metis\ {\isacharparenleft}{\kern0pt}no{\isacharunderscore}{\kern0pt}types{\isacharcomma}{\kern0pt}\ lifting{\isacharparenright}{\kern0pt}\ inf{\isacharunderscore}{\kern0pt}sup{\isacharunderscore}{\kern0pt}aci{\isacharparenleft}{\kern0pt}{\isadigit{1}}{\isacharparenright}{\kern0pt}{\isacharparenright}{\kern0pt}\isanewline
\ \ \isacommand{moreover}\isamarkupfalse%
\ \isacommand{have}\isamarkupfalse%
\ {\isachardoublequoteopen}{\isacharparenleft}{\kern0pt}inv{\isacharunderscore}{\kern0pt}image\ {\isacharparenleft}{\kern0pt}less{\isacharunderscore}{\kern0pt}than\ {\isacharless}{\kern0pt}{\isacharasterisk}{\kern0pt}lex{\isacharasterisk}{\kern0pt}{\isachargreater}{\kern0pt}\ s{\isacharparenright}{\kern0pt}\ {\isacharparenleft}{\kern0pt}{\isasymlambda}\ c{\isachardot}{\kern0pt}\ {\isacharparenleft}{\kern0pt}f\ c{\isacharcomma}{\kern0pt}\ g\ c{\isacharparenright}{\kern0pt}{\isacharparenright}{\kern0pt}{\isacharparenright}{\kern0pt}\ {\isacharequal}{\kern0pt}\ devos{\isacharunderscore}{\kern0pt}rel{\isachardoublequoteclose}\isanewline
\ \ \ \ \isacommand{using}\isamarkupfalse%
\ s{\isacharunderscore}{\kern0pt}def\ f{\isacharunderscore}{\kern0pt}def\ g{\isacharunderscore}{\kern0pt}def\ devos{\isacharunderscore}{\kern0pt}rel{\isacharunderscore}{\kern0pt}def\ mlex{\isacharunderscore}{\kern0pt}prod{\isacharunderscore}{\kern0pt}def\ \isacommand{by}\isamarkupfalse%
\ fastforce\isanewline
\ \ \isacommand{ultimately}\isamarkupfalse%
\ \isacommand{show}\isamarkupfalse%
\ {\isacharquery}{\kern0pt}thesis\ \isacommand{by}\isamarkupfalse%
\ simp\isanewline
\isacommand{qed}\isamarkupfalse%
%
\endisatagproof
{\isafoldproof}%
%
\isadelimproof
%
\endisadelimproof
%
\isadelimdocument
%
\endisadelimdocument
%
\isatagdocument
%
\isamarkupsubsection{Definition of $p(G)$ in \cite{DeVos2016OnAG} with associated lemmas%
}
\isamarkuptrue%
%
\endisatagdocument
{\isafolddocument}%
%
\isadelimdocument
%
\endisadelimdocument
\isacommand{definition}\isamarkupfalse%
\ p\ \isakeyword{where}\ {\isachardoublequoteopen}p\ {\isacharequal}{\kern0pt}\ Inf\ {\isacharparenleft}{\kern0pt}card\ {\isacharbackquote}{\kern0pt}\ {\isacharbraceleft}{\kern0pt}H{\isachardot}{\kern0pt}\ subgroup\ H\ G\ {\isacharparenleft}{\kern0pt}{\isasymcdot}{\isacharparenright}{\kern0pt}\ {\isasymone}\ {\isasymand}\ finite\ H\ {\isasymand}\ H\ {\isasymnoteq}\ {\isacharbraceleft}{\kern0pt}{\isasymone}{\isacharbraceright}{\kern0pt}{\isacharbraceright}{\kern0pt}{\isacharparenright}{\kern0pt}{\isachardoublequoteclose}\isanewline
\isanewline
\isacommand{lemma}\isamarkupfalse%
\ powers{\isacharunderscore}{\kern0pt}subgroup{\isacharcolon}{\kern0pt}\isanewline
\ \ \isakeyword{assumes}\ {\isachardoublequoteopen}g\ {\isasymin}\ G{\isachardoublequoteclose}\isanewline
\ \ \isakeyword{shows}\ {\isachardoublequoteopen}subgroup\ {\isacharparenleft}{\kern0pt}powers\ g{\isacharparenright}{\kern0pt}\ G\ {\isacharparenleft}{\kern0pt}{\isasymcdot}{\isacharparenright}{\kern0pt}\ {\isasymone}{\isachardoublequoteclose}\ \isanewline
%
\isadelimproof
\ \ %
\endisadelimproof
%
\isatagproof
\isacommand{by}\isamarkupfalse%
\ {\isacharparenleft}{\kern0pt}simp\ add{\isacharcolon}{\kern0pt}\ assms\ powers{\isacharunderscore}{\kern0pt}group\ powers{\isacharunderscore}{\kern0pt}submonoid\ subgroup{\isachardot}{\kern0pt}intro{\isacharparenright}{\kern0pt}%
\endisatagproof
{\isafoldproof}%
%
\isadelimproof
\isanewline
%
\endisadelimproof
\isanewline
\isacommand{lemma}\isamarkupfalse%
\ subgroup{\isacharunderscore}{\kern0pt}finite{\isacharunderscore}{\kern0pt}ge{\isacharcolon}{\kern0pt}\isanewline
\ \ \isakeyword{assumes}\ {\isachardoublequoteopen}subgroup\ H\ G\ {\isacharparenleft}{\kern0pt}{\isasymcdot}{\isacharparenright}{\kern0pt}\ {\isasymone}{\isachardoublequoteclose}\ \isakeyword{and}\ {\isachardoublequoteopen}H\ {\isasymnoteq}\ {\isacharbraceleft}{\kern0pt}{\isasymone}{\isacharbraceright}{\kern0pt}{\isachardoublequoteclose}\ \isakeyword{and}\ {\isachardoublequoteopen}finite\ H{\isachardoublequoteclose}\isanewline
\ \ \isakeyword{shows}\ {\isachardoublequoteopen}card\ H\ {\isasymge}\ p{\isachardoublequoteclose}\isanewline
%
\isadelimproof
\ \ %
\endisadelimproof
%
\isatagproof
\isacommand{using}\isamarkupfalse%
\ p{\isacharunderscore}{\kern0pt}def\ assms\ \isacommand{by}\isamarkupfalse%
\ {\isacharparenleft}{\kern0pt}simp\ add{\isacharcolon}{\kern0pt}\ wellorder{\isacharunderscore}{\kern0pt}Inf{\isacharunderscore}{\kern0pt}le{\isadigit{1}}{\isacharparenright}{\kern0pt}%
\endisatagproof
{\isafoldproof}%
%
\isadelimproof
\isanewline
%
\endisadelimproof
\isanewline
\isacommand{lemma}\isamarkupfalse%
\ subgroup{\isacharunderscore}{\kern0pt}infinite{\isacharunderscore}{\kern0pt}or{\isacharunderscore}{\kern0pt}ge{\isacharcolon}{\kern0pt}\isanewline
\ \ \isakeyword{assumes}\ {\isachardoublequoteopen}subgroup\ H\ G\ {\isacharparenleft}{\kern0pt}{\isasymcdot}{\isacharparenright}{\kern0pt}\ {\isasymone}{\isachardoublequoteclose}\ \isakeyword{and}\ {\isachardoublequoteopen}H\ {\isasymnoteq}\ {\isacharbraceleft}{\kern0pt}{\isasymone}{\isacharbraceright}{\kern0pt}{\isachardoublequoteclose}\isanewline
\ \ \isakeyword{shows}\ {\isachardoublequoteopen}infinite\ H\ {\isasymor}\ card\ H\ {\isasymge}\ p{\isachardoublequoteclose}%
\isadelimproof
\ %
\endisadelimproof
%
\isatagproof
\isacommand{using}\isamarkupfalse%
\ subgroup{\isacharunderscore}{\kern0pt}finite{\isacharunderscore}{\kern0pt}ge\ assms\ \isacommand{by}\isamarkupfalse%
\ auto%
\endisatagproof
{\isafoldproof}%
%
\isadelimproof
%
\endisadelimproof
\isanewline
\isanewline
\isacommand{end}\isamarkupfalse%
%
\isadelimdocument
%
\endisadelimdocument
%
\isatagdocument
%
\isamarkupsubsection{Proof of the Generalized Cauchy-Davenport for (non-abelian) groups%
}
\isamarkuptrue%
%
\endisatagdocument
{\isafolddocument}%
%
\isadelimdocument
%
\endisadelimdocument
%
\begin{isamarkuptext}%
Generalized Cauchy-Davenport theorem for (non-abelian) groups due to Matt DeVos \cite{DeVos2016OnAG}%
\end{isamarkuptext}\isamarkuptrue%
\isacommand{theorem}\isamarkupfalse%
\ {\isacharparenleft}{\kern0pt}\isakeyword{in}\ group{\isacharparenright}{\kern0pt}\ Generalized{\isacharunderscore}{\kern0pt}Cauchy{\isacharunderscore}{\kern0pt}Davenport{\isacharcolon}{\kern0pt}\isanewline
\ \ \isakeyword{assumes}\ hAne{\isacharcolon}{\kern0pt}\ {\isachardoublequoteopen}A\ {\isasymnoteq}\ {\isacharbraceleft}{\kern0pt}{\isacharbraceright}{\kern0pt}{\isachardoublequoteclose}\ \isakeyword{and}\ hBne{\isacharcolon}{\kern0pt}\ {\isachardoublequoteopen}B\ {\isasymnoteq}\ {\isacharbraceleft}{\kern0pt}{\isacharbraceright}{\kern0pt}{\isachardoublequoteclose}\ \isakeyword{and}\ hAG{\isacharcolon}{\kern0pt}\ {\isachardoublequoteopen}A\ {\isasymsubseteq}\ G{\isachardoublequoteclose}\ \isakeyword{and}\ hBG{\isacharcolon}{\kern0pt}\ {\isachardoublequoteopen}B\ {\isasymsubseteq}\ G{\isachardoublequoteclose}\ \isakeyword{and}\isanewline
\ \ hAfin{\isacharcolon}{\kern0pt}\ {\isachardoublequoteopen}finite\ A{\isachardoublequoteclose}\ \isakeyword{and}\ hBfin{\isacharcolon}{\kern0pt}\ {\isachardoublequoteopen}finite\ B{\isachardoublequoteclose}\ \isakeyword{and}\isanewline
\ \ hsub{\isacharcolon}{\kern0pt}\ {\isachardoublequoteopen}{\isacharbraceleft}{\kern0pt}H{\isachardot}{\kern0pt}\ subgroup{\isacharunderscore}{\kern0pt}of{\isacharunderscore}{\kern0pt}group\ H\ G\ {\isacharparenleft}{\kern0pt}{\isasymcdot}{\isacharparenright}{\kern0pt}\ {\isasymone}\ {\isasymand}\ finite\ H\ {\isasymand}\ H\ {\isasymnoteq}\ {\isacharbraceleft}{\kern0pt}{\isasymone}{\isacharbraceright}{\kern0pt}{\isacharbraceright}{\kern0pt}\ {\isasymnoteq}\ {\isacharbraceleft}{\kern0pt}{\isacharbraceright}{\kern0pt}{\isachardoublequoteclose}\isanewline
\ \ \isakeyword{shows}\ {\isachardoublequoteopen}card\ {\isacharparenleft}{\kern0pt}A\ {\isasymcdots}\ B{\isacharparenright}{\kern0pt}\ {\isasymge}\ min\ p\ {\isacharparenleft}{\kern0pt}card\ A\ {\isacharplus}{\kern0pt}\ card\ B\ {\isacharminus}{\kern0pt}\ {\isadigit{1}}{\isacharparenright}{\kern0pt}{\isachardoublequoteclose}\isanewline
%
\isadelimproof
%
\endisadelimproof
%
\isatagproof
\isacommand{proof}\isamarkupfalse%
{\isacharparenleft}{\kern0pt}rule\ ccontr{\isacharparenright}{\kern0pt}\isanewline
\ \ \isacommand{assume}\isamarkupfalse%
\ hcontr{\isacharcolon}{\kern0pt}\ {\isachardoublequoteopen}{\isasymnot}\ min\ p\ {\isacharparenleft}{\kern0pt}card\ A\ {\isacharplus}{\kern0pt}\ card\ B\ {\isacharminus}{\kern0pt}\ {\isadigit{1}}{\isacharparenright}{\kern0pt}\ {\isasymle}\ card\ {\isacharparenleft}{\kern0pt}A\ {\isasymcdots}\ B{\isacharparenright}{\kern0pt}{\isachardoublequoteclose}\isanewline
\ \ \isacommand{let}\isamarkupfalse%
\ {\isacharquery}{\kern0pt}fin\ {\isacharequal}{\kern0pt}\ {\isachardoublequoteopen}{\isacharbraceleft}{\kern0pt}{\isacharparenleft}{\kern0pt}A{\isacharcomma}{\kern0pt}\ B{\isacharparenright}{\kern0pt}{\isachardot}{\kern0pt}\ finite\ A\ {\isasymand}\ A\ {\isasymnoteq}\ {\isacharbraceleft}{\kern0pt}{\isacharbraceright}{\kern0pt}\ {\isasymand}\ A\ {\isasymsubseteq}\ G\ {\isasymand}\ finite\ B\ {\isasymand}\ B\ {\isasymnoteq}\ {\isacharbraceleft}{\kern0pt}{\isacharbraceright}{\kern0pt}\ {\isasymand}\ B\ {\isasymsubseteq}\ G{\isacharbraceright}{\kern0pt}{\isachardoublequoteclose}\isanewline
\ \ \isacommand{define}\isamarkupfalse%
\ M\ \isakeyword{where}\ {\isachardoublequoteopen}M\ {\isacharequal}{\kern0pt}\ {\isacharbraceleft}{\kern0pt}{\isacharparenleft}{\kern0pt}A{\isacharcomma}{\kern0pt}\ B{\isacharparenright}{\kern0pt}{\isachardot}{\kern0pt}\ card\ {\isacharparenleft}{\kern0pt}A\ {\isasymcdots}\ B{\isacharparenright}{\kern0pt}\ {\isacharless}{\kern0pt}\ min\ p\ {\isacharparenleft}{\kern0pt}card\ A\ {\isacharplus}{\kern0pt}\ card\ B\ {\isacharminus}{\kern0pt}\ {\isadigit{1}}{\isacharparenright}{\kern0pt}{\isacharbraceright}{\kern0pt}\ {\isasyminter}\ {\isacharquery}{\kern0pt}fin{\isachardoublequoteclose}\isanewline
\ \ \isacommand{have}\isamarkupfalse%
\ hmemM{\isacharcolon}{\kern0pt}\ {\isachardoublequoteopen}{\isacharparenleft}{\kern0pt}A{\isacharcomma}{\kern0pt}\ B{\isacharparenright}{\kern0pt}\ {\isasymin}\ M{\isachardoublequoteclose}\ \isacommand{using}\isamarkupfalse%
\ assms\ hcontr\ M{\isacharunderscore}{\kern0pt}def\ \isacommand{by}\isamarkupfalse%
\ auto\isanewline
\ \ \isacommand{then}\isamarkupfalse%
\ \isacommand{obtain}\isamarkupfalse%
\ X\ Y\ \isakeyword{where}\ hXYM{\isacharcolon}{\kern0pt}\ {\isachardoublequoteopen}{\isacharparenleft}{\kern0pt}X{\isacharcomma}{\kern0pt}\ Y{\isacharparenright}{\kern0pt}\ {\isasymin}\ M{\isachardoublequoteclose}\ \isakeyword{and}\ hmin{\isacharcolon}{\kern0pt}\ {\isachardoublequoteopen}{\isasymAnd}Z{\isachardot}{\kern0pt}\ Z\ {\isasymin}\ M\ {\isasymLongrightarrow}\ {\isacharparenleft}{\kern0pt}Z{\isacharcomma}{\kern0pt}\ {\isacharparenleft}{\kern0pt}X{\isacharcomma}{\kern0pt}\ Y{\isacharparenright}{\kern0pt}{\isacharparenright}{\kern0pt}\ {\isasymnotin}\ Restr\ devos{\isacharunderscore}{\kern0pt}rel\ {\isacharquery}{\kern0pt}fin{\isachardoublequoteclose}\ \isanewline
\ \ \ \ \isacommand{using}\isamarkupfalse%
\ devos{\isacharunderscore}{\kern0pt}rel{\isacharunderscore}{\kern0pt}wf\ wfE{\isacharunderscore}{\kern0pt}min\ \isacommand{by}\isamarkupfalse%
\ {\isacharparenleft}{\kern0pt}smt\ {\isacharparenleft}{\kern0pt}verit{\isacharcomma}{\kern0pt}\ del{\isacharunderscore}{\kern0pt}insts{\isacharparenright}{\kern0pt}\ old{\isachardot}{\kern0pt}prod{\isachardot}{\kern0pt}exhaust{\isacharparenright}{\kern0pt}\isanewline
\ \ \isacommand{have}\isamarkupfalse%
\ hXG{\isacharcolon}{\kern0pt}\ {\isachardoublequoteopen}X\ {\isasymsubseteq}\ G{\isachardoublequoteclose}\ \isakeyword{and}\ hYG{\isacharcolon}{\kern0pt}\ {\isachardoublequoteopen}Y\ {\isasymsubseteq}\ G{\isachardoublequoteclose}\ \isakeyword{and}\ hXfin{\isacharcolon}{\kern0pt}\ {\isachardoublequoteopen}finite\ X{\isachardoublequoteclose}\ \isakeyword{and}\ hYfin{\isacharcolon}{\kern0pt}\ {\isachardoublequoteopen}finite\ Y{\isachardoublequoteclose}\ \isakeyword{and}\ \isanewline
\ \ \ \ hXYlt{\isacharcolon}{\kern0pt}\ {\isachardoublequoteopen}card\ {\isacharparenleft}{\kern0pt}X\ {\isasymcdots}\ Y{\isacharparenright}{\kern0pt}\ {\isacharless}{\kern0pt}\ min\ p\ {\isacharparenleft}{\kern0pt}card\ X\ {\isacharplus}{\kern0pt}\ card\ Y\ {\isacharminus}{\kern0pt}\ {\isadigit{1}}{\isacharparenright}{\kern0pt}{\isachardoublequoteclose}\ \isacommand{using}\isamarkupfalse%
\ hXYM\ M{\isacharunderscore}{\kern0pt}def\ \isacommand{by}\isamarkupfalse%
\ auto\isanewline
\ \ \isacommand{have}\isamarkupfalse%
\ hXY{\isacharcolon}{\kern0pt}\ {\isachardoublequoteopen}card\ X\ {\isasymle}\ card\ Y{\isachardoublequoteclose}\isanewline
\ \ \isacommand{proof}\isamarkupfalse%
{\isacharparenleft}{\kern0pt}rule\ ccontr{\isacharparenright}{\kern0pt}\isanewline
\ \ \ \ \isacommand{assume}\isamarkupfalse%
\ hcontr{\isacharcolon}{\kern0pt}\ {\isachardoublequoteopen}{\isasymnot}\ card\ X\ {\isasymle}\ card\ Y{\isachardoublequoteclose}\isanewline
\ \ \ \ \isacommand{have}\isamarkupfalse%
\ hinvinj{\isacharcolon}{\kern0pt}\ {\isachardoublequoteopen}inj{\isacharunderscore}{\kern0pt}on\ inverse\ G{\isachardoublequoteclose}\ \isacommand{using}\isamarkupfalse%
\ inj{\isacharunderscore}{\kern0pt}onI\ invertible\ invertible{\isacharunderscore}{\kern0pt}inverse{\isacharunderscore}{\kern0pt}inverse\ \isacommand{by}\isamarkupfalse%
\ metis\isanewline
\ \ \ \ \isacommand{let}\isamarkupfalse%
\ {\isacharquery}{\kern0pt}M\ {\isacharequal}{\kern0pt}\ {\isachardoublequoteopen}inverse\ {\isacharbackquote}{\kern0pt}\ X{\isachardoublequoteclose}\isanewline
\ \ \ \ \isacommand{let}\isamarkupfalse%
\ {\isacharquery}{\kern0pt}N\ {\isacharequal}{\kern0pt}\ {\isachardoublequoteopen}inverse\ {\isacharbackquote}{\kern0pt}\ Y{\isachardoublequoteclose}\isanewline
\ \ \ \ \isacommand{have}\isamarkupfalse%
\ {\isachardoublequoteopen}{\isacharquery}{\kern0pt}N\ {\isasymcdots}\ {\isacharquery}{\kern0pt}M\ {\isacharequal}{\kern0pt}\ inverse\ {\isacharbackquote}{\kern0pt}\ {\isacharparenleft}{\kern0pt}X\ {\isasymcdots}\ Y{\isacharparenright}{\kern0pt}{\isachardoublequoteclose}\ \isacommand{using}\isamarkupfalse%
\ set{\isacharunderscore}{\kern0pt}inverse{\isacharunderscore}{\kern0pt}composition{\isacharunderscore}{\kern0pt}commute\ hXYM\ M{\isacharunderscore}{\kern0pt}def\ \isacommand{by}\isamarkupfalse%
\ auto\isanewline
\ \ \ \ \isacommand{then}\isamarkupfalse%
\ \isacommand{have}\isamarkupfalse%
\ hNM{\isacharcolon}{\kern0pt}\ {\isachardoublequoteopen}card\ {\isacharparenleft}{\kern0pt}{\isacharquery}{\kern0pt}N\ {\isasymcdots}\ {\isacharquery}{\kern0pt}M{\isacharparenright}{\kern0pt}\ {\isacharequal}{\kern0pt}\ card\ {\isacharparenleft}{\kern0pt}X\ {\isasymcdots}\ Y{\isacharparenright}{\kern0pt}{\isachardoublequoteclose}\ \isanewline
\ \ \ \ \ \ \isacommand{using}\isamarkupfalse%
\ hinvinj\ card{\isacharunderscore}{\kern0pt}image\ subset{\isacharunderscore}{\kern0pt}inj{\isacharunderscore}{\kern0pt}on\ smul{\isacharunderscore}{\kern0pt}subset{\isacharunderscore}{\kern0pt}carrier\ \isacommand{by}\isamarkupfalse%
\ metis\isanewline
\ \ \ \ \isacommand{moreover}\isamarkupfalse%
\ \isacommand{have}\isamarkupfalse%
\ hM{\isacharcolon}{\kern0pt}\ {\isachardoublequoteopen}card\ {\isacharquery}{\kern0pt}M\ {\isacharequal}{\kern0pt}\ card\ X{\isachardoublequoteclose}\isanewline
\ \ \ \ \ \ \isacommand{using}\isamarkupfalse%
\ hinvinj\ hXG\ hYG\ card{\isacharunderscore}{\kern0pt}image\ subset{\isacharunderscore}{\kern0pt}inj{\isacharunderscore}{\kern0pt}on\ \isacommand{by}\isamarkupfalse%
\ metis\isanewline
\ \ \ \ \isacommand{moreover}\isamarkupfalse%
\ \isacommand{have}\isamarkupfalse%
\ hN{\isacharcolon}{\kern0pt}\ {\isachardoublequoteopen}card\ {\isacharquery}{\kern0pt}N\ {\isacharequal}{\kern0pt}\ card\ Y{\isachardoublequoteclose}\ \isanewline
\ \ \ \ \ \ \isacommand{using}\isamarkupfalse%
\ hinvinj\ hYG\ card{\isacharunderscore}{\kern0pt}image\ subset{\isacharunderscore}{\kern0pt}inj{\isacharunderscore}{\kern0pt}on\ \isacommand{by}\isamarkupfalse%
\ metis\isanewline
\ \ \ \ \isacommand{moreover}\isamarkupfalse%
\ \isacommand{have}\isamarkupfalse%
\ hNplusM{\isacharcolon}{\kern0pt}\ {\isachardoublequoteopen}card\ {\isacharquery}{\kern0pt}N\ {\isacharplus}{\kern0pt}\ card\ {\isacharquery}{\kern0pt}M\ {\isacharequal}{\kern0pt}\ card\ X\ {\isacharplus}{\kern0pt}\ card\ Y{\isachardoublequoteclose}\ \isacommand{using}\isamarkupfalse%
\ hM\ hN\ \isacommand{by}\isamarkupfalse%
\ auto\isanewline
\ \ \ \ \isacommand{ultimately}\isamarkupfalse%
\ \isacommand{have}\isamarkupfalse%
\ {\isachardoublequoteopen}card\ {\isacharparenleft}{\kern0pt}{\isacharquery}{\kern0pt}N\ {\isasymcdots}\ {\isacharquery}{\kern0pt}M{\isacharparenright}{\kern0pt}\ {\isacharless}{\kern0pt}\ min\ p\ {\isacharparenleft}{\kern0pt}card\ {\isacharquery}{\kern0pt}N\ {\isacharplus}{\kern0pt}\ card\ {\isacharquery}{\kern0pt}M\ {\isacharminus}{\kern0pt}\ {\isadigit{1}}{\isacharparenright}{\kern0pt}{\isachardoublequoteclose}\ \isanewline
\ \ \ \ \ \ \isacommand{using}\isamarkupfalse%
\ hXYM\ M{\isacharunderscore}{\kern0pt}def\ \isacommand{by}\isamarkupfalse%
\ auto\isanewline
\ \ \ \ \isacommand{then}\isamarkupfalse%
\ \isacommand{have}\isamarkupfalse%
\ {\isachardoublequoteopen}{\isacharparenleft}{\kern0pt}{\isacharquery}{\kern0pt}N{\isacharcomma}{\kern0pt}\ {\isacharquery}{\kern0pt}M{\isacharparenright}{\kern0pt}\ {\isasymin}\ M{\isachardoublequoteclose}\ \isacommand{using}\isamarkupfalse%
\ M{\isacharunderscore}{\kern0pt}def\ hXYM\ \isacommand{by}\isamarkupfalse%
\ blast\isanewline
\ \ \ \ \isacommand{then}\isamarkupfalse%
\ \isacommand{have}\isamarkupfalse%
\ {\isachardoublequoteopen}{\isacharparenleft}{\kern0pt}{\isacharparenleft}{\kern0pt}{\isacharquery}{\kern0pt}N{\isacharcomma}{\kern0pt}\ {\isacharquery}{\kern0pt}M{\isacharparenright}{\kern0pt}{\isacharcomma}{\kern0pt}\ {\isacharparenleft}{\kern0pt}X{\isacharcomma}{\kern0pt}\ Y{\isacharparenright}{\kern0pt}{\isacharparenright}{\kern0pt}\ {\isasymnotin}\ devos{\isacharunderscore}{\kern0pt}rel{\isachardoublequoteclose}\ \isacommand{using}\isamarkupfalse%
\ hmin\ hXYM\ M{\isacharunderscore}{\kern0pt}def\ \isacommand{by}\isamarkupfalse%
\ blast\isanewline
\ \ \ \ \isacommand{then}\isamarkupfalse%
\ \isacommand{have}\isamarkupfalse%
\ {\isachardoublequoteopen}{\isasymnot}\ card\ Y\ {\isacharless}{\kern0pt}\ card\ X{\isachardoublequoteclose}\ \isacommand{using}\isamarkupfalse%
\ hN\ \ hNM\ hNplusM\ devos{\isacharunderscore}{\kern0pt}rel{\isacharunderscore}{\kern0pt}iff\ \isacommand{by}\isamarkupfalse%
\ simp\isanewline
\ \ \ \ \isacommand{then}\isamarkupfalse%
\ \isacommand{show}\isamarkupfalse%
\ False\ \isacommand{using}\isamarkupfalse%
\ hcontr\ \isacommand{by}\isamarkupfalse%
\ linarith\isanewline
\ \ \isacommand{qed}\isamarkupfalse%
\isanewline
\ \ \isacommand{have}\isamarkupfalse%
\ hX{\isadigit{2}}{\isacharcolon}{\kern0pt}\ {\isachardoublequoteopen}{\isadigit{2}}\ {\isasymle}\ card\ X{\isachardoublequoteclose}\isanewline
\ \ \isacommand{proof}\isamarkupfalse%
{\isacharparenleft}{\kern0pt}rule\ ccontr{\isacharparenright}{\kern0pt}\isanewline
\ \ \ \ \isacommand{assume}\isamarkupfalse%
\ {\isachardoublequoteopen}\ {\isasymnot}\ {\isadigit{2}}\ {\isasymle}\ card\ X{\isachardoublequoteclose}\isanewline
\ \ \ \ \isacommand{moreover}\isamarkupfalse%
\ \isacommand{have}\isamarkupfalse%
\ {\isachardoublequoteopen}card\ X\ {\isachargreater}{\kern0pt}\ {\isadigit{0}}{\isachardoublequoteclose}\ \isacommand{using}\isamarkupfalse%
\ hXYM\ M{\isacharunderscore}{\kern0pt}def\ card{\isacharunderscore}{\kern0pt}gt{\isacharunderscore}{\kern0pt}{\isadigit{0}}{\isacharunderscore}{\kern0pt}iff\ \isacommand{by}\isamarkupfalse%
\ blast\isanewline
\ \ \ \ \isacommand{ultimately}\isamarkupfalse%
\ \isacommand{have}\isamarkupfalse%
\ hX{\isadigit{1}}{\isacharcolon}{\kern0pt}\ {\isachardoublequoteopen}card\ X\ {\isacharequal}{\kern0pt}\ {\isadigit{1}}{\isachardoublequoteclose}\ \isacommand{by}\isamarkupfalse%
\ auto\isanewline
\ \ \ \ \isacommand{then}\isamarkupfalse%
\ \isacommand{obtain}\isamarkupfalse%
\ x\ \isakeyword{where}\ {\isachardoublequoteopen}X\ {\isacharequal}{\kern0pt}\ {\isacharbraceleft}{\kern0pt}x{\isacharbraceright}{\kern0pt}{\isachardoublequoteclose}\ \isakeyword{and}\ {\isachardoublequoteopen}x\ {\isasymin}\ G{\isachardoublequoteclose}\ \isacommand{using}\isamarkupfalse%
\ hXG\ \isacommand{by}\isamarkupfalse%
\ {\isacharparenleft}{\kern0pt}metis\ card{\isacharunderscore}{\kern0pt}{\isadigit{1}}{\isacharunderscore}{\kern0pt}singletonE\ insert{\isacharunderscore}{\kern0pt}subset{\isacharparenright}{\kern0pt}\isanewline
\ \ \ \ \isacommand{then}\isamarkupfalse%
\ \isacommand{have}\isamarkupfalse%
\ {\isachardoublequoteopen}card\ {\isacharparenleft}{\kern0pt}X\ {\isasymcdots}\ Y{\isacharparenright}{\kern0pt}\ {\isacharequal}{\kern0pt}\ card\ X\ {\isacharplus}{\kern0pt}\ card\ Y\ {\isacharminus}{\kern0pt}\ {\isadigit{1}}{\isachardoublequoteclose}\ \isacommand{using}\isamarkupfalse%
\ card{\isacharunderscore}{\kern0pt}smul{\isacharunderscore}{\kern0pt}singleton{\isacharunderscore}{\kern0pt}left{\isacharunderscore}{\kern0pt}eq\ hYG\ hXYM\ M{\isacharunderscore}{\kern0pt}def\isanewline
\ \ \ \ \ \ \isacommand{by}\isamarkupfalse%
\ {\isacharparenleft}{\kern0pt}simp\ add{\isacharcolon}{\kern0pt}\ Int{\isacharunderscore}{\kern0pt}absorb{\isadigit{2}}{\isacharparenright}{\kern0pt}\isanewline
\ \ \ \ \isacommand{then}\isamarkupfalse%
\ \isacommand{show}\isamarkupfalse%
\ False\ \isacommand{using}\isamarkupfalse%
\ hXYlt\ \isacommand{by}\isamarkupfalse%
\ linarith\isanewline
\ \ \isacommand{qed}\isamarkupfalse%
\isanewline
\ \ \isacommand{then}\isamarkupfalse%
\ \isacommand{obtain}\isamarkupfalse%
\ a\ b\ \isakeyword{where}\ habX{\isacharcolon}{\kern0pt}\ {\isachardoublequoteopen}{\isacharbraceleft}{\kern0pt}a{\isacharcomma}{\kern0pt}\ b{\isacharbraceright}{\kern0pt}\ {\isasymsubseteq}\ X{\isachardoublequoteclose}\ \isakeyword{and}\ habne{\isacharcolon}{\kern0pt}\ {\isachardoublequoteopen}a\ {\isasymnoteq}\ b{\isachardoublequoteclose}\ \isacommand{by}\isamarkupfalse%
\ {\isacharparenleft}{\kern0pt}metis\ card{\isacharunderscore}{\kern0pt}{\isadigit{2}}{\isacharunderscore}{\kern0pt}iff\ obtain{\isacharunderscore}{\kern0pt}subset{\isacharunderscore}{\kern0pt}with{\isacharunderscore}{\kern0pt}card{\isacharunderscore}{\kern0pt}n{\isacharparenright}{\kern0pt}\isanewline
\ \ \isacommand{moreover}\isamarkupfalse%
\ \isacommand{have}\isamarkupfalse%
\ {\isachardoublequoteopen}b\ {\isasymin}\ X\ {\isasymcdots}\ {\isacharbraceleft}{\kern0pt}inverse\ a\ {\isasymcdot}\ b{\isacharbraceright}{\kern0pt}{\isachardoublequoteclose}\ \isacommand{by}\isamarkupfalse%
\ {\isacharparenleft}{\kern0pt}smt\ {\isacharparenleft}{\kern0pt}verit{\isacharparenright}{\kern0pt}\ habX\ composition{\isacharunderscore}{\kern0pt}closed\ hXG\ insert{\isacharunderscore}{\kern0pt}subset\ \isanewline
\ \ \ \ invertible\ invertible{\isacharunderscore}{\kern0pt}inverse{\isacharunderscore}{\kern0pt}closed\ invertible{\isacharunderscore}{\kern0pt}right{\isacharunderscore}{\kern0pt}inverse{\isadigit{2}}\ singletonI\ smul{\isachardot}{\kern0pt}smulI\ subsetD{\isacharparenright}{\kern0pt}\isanewline
\ \ \isacommand{then}\isamarkupfalse%
\ \isacommand{obtain}\isamarkupfalse%
\ g\ \isakeyword{where}\ hgG{\isacharcolon}{\kern0pt}\ {\isachardoublequoteopen}g\ {\isasymin}\ G{\isachardoublequoteclose}\ \isakeyword{and}\ hg{\isadigit{1}}{\isacharcolon}{\kern0pt}\ {\isachardoublequoteopen}g\ {\isasymnoteq}\ {\isasymone}{\isachardoublequoteclose}\ \isakeyword{and}\ hXgne{\isacharcolon}{\kern0pt}\ {\isachardoublequoteopen}{\isacharparenleft}{\kern0pt}X\ {\isasymcdots}\ {\isacharbraceleft}{\kern0pt}g{\isacharbraceright}{\kern0pt}{\isacharparenright}{\kern0pt}\ {\isasyminter}\ X\ {\isasymnoteq}\ {\isacharbraceleft}{\kern0pt}{\isacharbraceright}{\kern0pt}{\isachardoublequoteclose}\ \isanewline
\ \ \ \ \isacommand{using}\isamarkupfalse%
\ habne\ habX\ hXG\ \isacommand{by}\isamarkupfalse%
\ {\isacharparenleft}{\kern0pt}metis\ composition{\isacharunderscore}{\kern0pt}closed\ insert{\isacharunderscore}{\kern0pt}disjoint{\isacharparenleft}{\kern0pt}{\isadigit{2}}{\isacharparenright}{\kern0pt}\ insert{\isacharunderscore}{\kern0pt}subset\ invertible\ \isanewline
\ \ \ \ \ \ invertible{\isacharunderscore}{\kern0pt}inverse{\isacharunderscore}{\kern0pt}closed\ invertible{\isacharunderscore}{\kern0pt}right{\isacharunderscore}{\kern0pt}inverse{\isadigit{2}}\ mk{\isacharunderscore}{\kern0pt}disjoint{\isacharunderscore}{\kern0pt}insert\ right{\isacharunderscore}{\kern0pt}unit{\isacharparenright}{\kern0pt}\isanewline
\ \ \isacommand{have}\isamarkupfalse%
\ hpsubX{\isacharcolon}{\kern0pt}\ {\isachardoublequoteopen}{\isacharparenleft}{\kern0pt}X\ {\isasymcdots}\ {\isacharbraceleft}{\kern0pt}g{\isacharbraceright}{\kern0pt}{\isacharparenright}{\kern0pt}\ {\isasyminter}\ X\ {\isasymsubset}\ X{\isachardoublequoteclose}\isanewline
\ \ \isacommand{proof}\isamarkupfalse%
{\isacharparenleft}{\kern0pt}rule\ ccontr{\isacharparenright}{\kern0pt}\isanewline
\ \ \ \ \isacommand{assume}\isamarkupfalse%
\ {\isachardoublequoteopen}{\isasymnot}\ {\isacharparenleft}{\kern0pt}X\ {\isasymcdots}\ {\isacharbraceleft}{\kern0pt}g{\isacharbraceright}{\kern0pt}{\isacharparenright}{\kern0pt}\ {\isasyminter}\ X\ {\isasymsubset}\ X{\isachardoublequoteclose}\isanewline
\ \ \ \ \isacommand{then}\isamarkupfalse%
\ \isacommand{have}\isamarkupfalse%
\ hXsub{\isacharcolon}{\kern0pt}\ {\isachardoublequoteopen}X\ {\isasymsubseteq}\ X\ {\isasymcdots}\ {\isacharbraceleft}{\kern0pt}g{\isacharbraceright}{\kern0pt}{\isachardoublequoteclose}\ \isacommand{by}\isamarkupfalse%
\ auto\isanewline
\ \ \ \ \isacommand{then}\isamarkupfalse%
\ \isacommand{have}\isamarkupfalse%
\ {\isachardoublequoteopen}card\ X\ {\isasymcdots}\ {\isacharbraceleft}{\kern0pt}g{\isacharbraceright}{\kern0pt}\ {\isacharequal}{\kern0pt}\ card\ X{\isachardoublequoteclose}\ \isacommand{using}\isamarkupfalse%
\ card{\isacharunderscore}{\kern0pt}smul{\isacharunderscore}{\kern0pt}sing{\isacharunderscore}{\kern0pt}right{\isacharunderscore}{\kern0pt}le\ hXYM\ M{\isacharunderscore}{\kern0pt}def\isanewline
\ \ \ \ \ \ \isacommand{by}\isamarkupfalse%
\ {\isacharparenleft}{\kern0pt}metis\ Int{\isacharunderscore}{\kern0pt}absorb{\isadigit{2}}\ {\isacartoucheopen}g\ {\isasymin}\ G{\isacartoucheclose}\ card{\isachardot}{\kern0pt}infinite\ card{\isacharunderscore}{\kern0pt}smul{\isacharunderscore}{\kern0pt}singleton{\isacharunderscore}{\kern0pt}right{\isacharunderscore}{\kern0pt}eq\ finite{\isacharunderscore}{\kern0pt}Int\ hXG{\isacharparenright}{\kern0pt}\isanewline
\ \ \ \ \isacommand{moreover}\isamarkupfalse%
\ \isacommand{have}\isamarkupfalse%
\ hXfin{\isacharcolon}{\kern0pt}\ {\isachardoublequoteopen}finite\ X{\isachardoublequoteclose}\ \isacommand{using}\isamarkupfalse%
\ hXYM\ M{\isacharunderscore}{\kern0pt}def\ \isacommand{by}\isamarkupfalse%
\ auto\isanewline
\ \ \ \ \isacommand{ultimately}\isamarkupfalse%
\ \isacommand{have}\isamarkupfalse%
\ {\isachardoublequoteopen}X\ {\isasymcdots}\ {\isacharbraceleft}{\kern0pt}g{\isacharbraceright}{\kern0pt}\ {\isacharequal}{\kern0pt}\ X{\isachardoublequoteclose}\ \isacommand{using}\isamarkupfalse%
\ hXsub\ \isanewline
\ \ \ \ \ \ \isacommand{by}\isamarkupfalse%
\ {\isacharparenleft}{\kern0pt}metis\ card{\isacharunderscore}{\kern0pt}subset{\isacharunderscore}{\kern0pt}eq\ finite{\isachardot}{\kern0pt}emptyI\ finite{\isachardot}{\kern0pt}insertI\ finite{\isacharunderscore}{\kern0pt}smul{\isacharparenright}{\kern0pt}\isanewline
\ \ \ \ \isacommand{then}\isamarkupfalse%
\ \isacommand{have}\isamarkupfalse%
\ hXpow{\isacharcolon}{\kern0pt}\ {\isachardoublequoteopen}X\ {\isasymcdots}\ {\isacharparenleft}{\kern0pt}powers\ g{\isacharparenright}{\kern0pt}\ {\isacharequal}{\kern0pt}\ X{\isachardoublequoteclose}\ \isacommand{by}\isamarkupfalse%
\ {\isacharparenleft}{\kern0pt}simp\ add{\isacharcolon}{\kern0pt}\ hXG\ hgG\ smul{\isacharunderscore}{\kern0pt}singleton{\isacharunderscore}{\kern0pt}eq{\isacharunderscore}{\kern0pt}contains{\isacharunderscore}{\kern0pt}powers{\isacharparenright}{\kern0pt}\isanewline
\ \ \ \ \isacommand{moreover}\isamarkupfalse%
\ \isacommand{have}\isamarkupfalse%
\ hfinpowers{\isacharcolon}{\kern0pt}\ {\isachardoublequoteopen}finite\ {\isacharparenleft}{\kern0pt}powers\ g{\isacharparenright}{\kern0pt}{\isachardoublequoteclose}\isanewline
\ \ \ \ \isacommand{proof}\isamarkupfalse%
{\isacharparenleft}{\kern0pt}rule\ ccontr{\isacharparenright}{\kern0pt}\isanewline
\ \ \ \ \ \ \isacommand{assume}\isamarkupfalse%
\ {\isachardoublequoteopen}infinite\ {\isacharparenleft}{\kern0pt}powers\ g{\isacharparenright}{\kern0pt}{\isachardoublequoteclose}\isanewline
\ \ \ \ \ \ \isacommand{then}\isamarkupfalse%
\ \isacommand{have}\isamarkupfalse%
\ {\isachardoublequoteopen}infinite\ X{\isachardoublequoteclose}\ \isacommand{using}\isamarkupfalse%
\ hXG\ hX{\isadigit{2}}\ hXpow\ \isacommand{by}\isamarkupfalse%
\ {\isacharparenleft}{\kern0pt}metis\ Int{\isacharunderscore}{\kern0pt}absorb{\isadigit{1}}\ hXgne\ hXsub\ hgG\ \isanewline
\ \ \ \ \ \ \ \ infinite{\isacharunderscore}{\kern0pt}smul{\isacharunderscore}{\kern0pt}right\ invertible\ le{\isacharunderscore}{\kern0pt}iff{\isacharunderscore}{\kern0pt}inf\ powers{\isacharunderscore}{\kern0pt}submonoid\ submonoid{\isachardot}{\kern0pt}subset{\isacharparenright}{\kern0pt}\isanewline
\ \ \ \ \ \ \isacommand{then}\isamarkupfalse%
\ \isacommand{show}\isamarkupfalse%
\ False\ \isacommand{using}\isamarkupfalse%
\ hXYM\ M{\isacharunderscore}{\kern0pt}def\ \isacommand{by}\isamarkupfalse%
\ auto\isanewline
\ \ \ \ \isacommand{qed}\isamarkupfalse%
\isanewline
\ \ \ \ \isacommand{ultimately}\isamarkupfalse%
\ \isacommand{have}\isamarkupfalse%
\ {\isachardoublequoteopen}card\ {\isacharparenleft}{\kern0pt}powers\ g{\isacharparenright}{\kern0pt}\ {\isasymle}\ card\ X{\isachardoublequoteclose}\ \isacommand{using}\isamarkupfalse%
\ card{\isacharunderscore}{\kern0pt}le{\isacharunderscore}{\kern0pt}smul{\isacharunderscore}{\kern0pt}right\ \isanewline
\ \ \ \ \ \ powers{\isacharunderscore}{\kern0pt}submonoid\ submonoid{\isachardot}{\kern0pt}subset\ hXYM\ M{\isacharunderscore}{\kern0pt}def\ habX\ hXG\ \isanewline
\ \ \ \ \ \ \isacommand{by}\isamarkupfalse%
\ {\isacharparenleft}{\kern0pt}metis\ {\isacharparenleft}{\kern0pt}no{\isacharunderscore}{\kern0pt}types{\isacharcomma}{\kern0pt}\ lifting{\isacharparenright}{\kern0pt}\ hXfin\ hgG\ insert{\isacharunderscore}{\kern0pt}subset\ invertible\ subsetD{\isacharparenright}{\kern0pt}\isanewline
\ \ \ \ \isacommand{then}\isamarkupfalse%
\ \isacommand{have}\isamarkupfalse%
\ {\isachardoublequoteopen}p\ {\isasymle}\ card\ X{\isachardoublequoteclose}\ \isanewline
\ \ \ \ \ \ \isacommand{by}\isamarkupfalse%
\ {\isacharparenleft}{\kern0pt}meson\ hfinpowers\ hg{\isadigit{1}}\ hgG\ le{\isacharunderscore}{\kern0pt}trans\ powers{\isacharunderscore}{\kern0pt}ne{\isacharunderscore}{\kern0pt}one\ powers{\isacharunderscore}{\kern0pt}subgroup\ subgroup{\isacharunderscore}{\kern0pt}infinite{\isacharunderscore}{\kern0pt}or{\isacharunderscore}{\kern0pt}ge{\isacharparenright}{\kern0pt}\isanewline
\ \ \ \ \isacommand{then}\isamarkupfalse%
\ \isacommand{have}\isamarkupfalse%
\ {\isachardoublequoteopen}p\ {\isasymle}\ card\ {\isacharparenleft}{\kern0pt}X\ {\isasymcdots}\ Y{\isacharparenright}{\kern0pt}{\isachardoublequoteclose}\ \isacommand{using}\isamarkupfalse%
\ card{\isacharunderscore}{\kern0pt}le{\isacharunderscore}{\kern0pt}smul{\isacharunderscore}{\kern0pt}left\ hXYM\ M{\isacharunderscore}{\kern0pt}def\ \isanewline
\ \ \ \ \ \ \isacommand{by}\isamarkupfalse%
\ {\isacharparenleft}{\kern0pt}metis\ {\isacharparenleft}{\kern0pt}full{\isacharunderscore}{\kern0pt}types{\isacharparenright}{\kern0pt}\ {\isacartoucheopen}b\ {\isasymin}\ smul\ X\ {\isacharbraceleft}{\kern0pt}inverse\ a\ {\isasymcdot}\ b{\isacharbraceright}{\kern0pt}{\isacartoucheclose}\ bot{\isacharunderscore}{\kern0pt}nat{\isacharunderscore}{\kern0pt}{\isadigit{0}}{\isachardot}{\kern0pt}extremum{\isacharunderscore}{\kern0pt}uniqueI\ card{\isachardot}{\kern0pt}infinite\ \isanewline
\ \ \ \ \ \ \ \ \ \ card{\isacharunderscore}{\kern0pt}{\isadigit{0}}{\isacharunderscore}{\kern0pt}eq\ card{\isacharunderscore}{\kern0pt}le{\isacharunderscore}{\kern0pt}smul{\isacharunderscore}{\kern0pt}right\ empty{\isacharunderscore}{\kern0pt}iff\ hXY\ hXfin\ hYG\ le{\isacharunderscore}{\kern0pt}trans\ smul{\isachardot}{\kern0pt}cases{\isacharparenright}{\kern0pt}\isanewline
\ \ \ \ \isacommand{then}\isamarkupfalse%
\ \isacommand{show}\isamarkupfalse%
\ False\ \isacommand{using}\isamarkupfalse%
\ hXYlt\ \isacommand{by}\isamarkupfalse%
\ auto\isanewline
\ \ \isacommand{qed}\isamarkupfalse%
\isanewline
\ \ \isacommand{let}\isamarkupfalse%
\ {\isacharquery}{\kern0pt}X{\isadigit{1}}\ {\isacharequal}{\kern0pt}\ {\isachardoublequoteopen}{\isacharparenleft}{\kern0pt}X\ {\isasymcdots}\ {\isacharbraceleft}{\kern0pt}g{\isacharbraceright}{\kern0pt}{\isacharparenright}{\kern0pt}\ {\isasyminter}\ X{\isachardoublequoteclose}\isanewline
\ \ \isacommand{let}\isamarkupfalse%
\ {\isacharquery}{\kern0pt}X{\isadigit{2}}\ {\isacharequal}{\kern0pt}\ {\isachardoublequoteopen}{\isacharparenleft}{\kern0pt}X\ {\isasymcdots}\ {\isacharbraceleft}{\kern0pt}g{\isacharbraceright}{\kern0pt}{\isacharparenright}{\kern0pt}\ {\isasymunion}\ X{\isachardoublequoteclose}\isanewline
\ \ \isacommand{let}\isamarkupfalse%
\ {\isacharquery}{\kern0pt}Y{\isadigit{1}}\ {\isacharequal}{\kern0pt}\ {\isachardoublequoteopen}{\isacharparenleft}{\kern0pt}{\isacharbraceleft}{\kern0pt}inverse\ g{\isacharbraceright}{\kern0pt}\ {\isasymcdots}\ Y{\isacharparenright}{\kern0pt}\ {\isasymunion}\ Y{\isachardoublequoteclose}\isanewline
\ \ \isacommand{let}\isamarkupfalse%
\ {\isacharquery}{\kern0pt}Y{\isadigit{2}}\ {\isacharequal}{\kern0pt}\ {\isachardoublequoteopen}{\isacharparenleft}{\kern0pt}{\isacharbraceleft}{\kern0pt}inverse\ g{\isacharbraceright}{\kern0pt}\ {\isasymcdots}\ Y{\isacharparenright}{\kern0pt}\ {\isasyminter}\ Y{\isachardoublequoteclose}\isanewline
\ \ \isacommand{have}\isamarkupfalse%
\ hY{\isadigit{1}}G{\isacharcolon}{\kern0pt}\ {\isachardoublequoteopen}{\isacharquery}{\kern0pt}Y{\isadigit{1}}\ {\isasymsubseteq}\ G{\isachardoublequoteclose}\ \isakeyword{and}\ hY{\isadigit{1}}fin{\isacharcolon}{\kern0pt}\ {\isachardoublequoteopen}finite\ {\isacharquery}{\kern0pt}Y{\isadigit{1}}{\isachardoublequoteclose}\ \isakeyword{and}\ hX{\isadigit{2}}G{\isacharcolon}{\kern0pt}\ {\isachardoublequoteopen}{\isacharquery}{\kern0pt}X{\isadigit{2}}\ {\isasymsubseteq}\ G{\isachardoublequoteclose}\ \isakeyword{and}\ hX{\isadigit{2}}fin{\isacharcolon}{\kern0pt}\ {\isachardoublequoteopen}finite\ {\isacharquery}{\kern0pt}X{\isadigit{2}}{\isachardoublequoteclose}\ \isanewline
\ \ \ \ \isacommand{using}\isamarkupfalse%
\ hYfin\ hYG\ hXG\ finite{\isacharunderscore}{\kern0pt}smul\ hXfin\ smul{\isacharunderscore}{\kern0pt}subset{\isacharunderscore}{\kern0pt}carrier\ \isacommand{by}\isamarkupfalse%
\ auto\isanewline
\ \ \isacommand{have}\isamarkupfalse%
\ hXY{\isadigit{1}}{\isacharcolon}{\kern0pt}\ {\isachardoublequoteopen}{\isacharquery}{\kern0pt}X{\isadigit{1}}\ {\isasymcdots}\ {\isacharquery}{\kern0pt}Y{\isadigit{1}}\ {\isasymsubseteq}\ X\ {\isasymcdots}\ Y{\isachardoublequoteclose}\isanewline
\ \ \isacommand{proof}\isamarkupfalse%
\isanewline
\ \ \ \ \isacommand{fix}\isamarkupfalse%
\ z\ \isacommand{assume}\isamarkupfalse%
\ {\isachardoublequoteopen}z\ {\isasymin}\ {\isacharquery}{\kern0pt}X{\isadigit{1}}\ {\isasymcdots}\ {\isacharquery}{\kern0pt}Y{\isadigit{1}}{\isachardoublequoteclose}\isanewline
\ \ \ \ \isacommand{then}\isamarkupfalse%
\ \isacommand{obtain}\isamarkupfalse%
\ x\ y\ \isakeyword{where}\ hz{\isacharcolon}{\kern0pt}\ {\isachardoublequoteopen}z\ {\isacharequal}{\kern0pt}\ x\ {\isasymcdot}\ y{\isachardoublequoteclose}\ \isakeyword{and}\ hx{\isacharcolon}{\kern0pt}\ {\isachardoublequoteopen}x\ {\isasymin}\ {\isacharquery}{\kern0pt}X{\isadigit{1}}{\isachardoublequoteclose}\ \isakeyword{and}\ hy{\isacharcolon}{\kern0pt}\ {\isachardoublequoteopen}y\ {\isasymin}\ {\isacharquery}{\kern0pt}Y{\isadigit{1}}{\isachardoublequoteclose}\ \isacommand{by}\isamarkupfalse%
\ {\isacharparenleft}{\kern0pt}meson\ smul{\isachardot}{\kern0pt}cases{\isacharparenright}{\kern0pt}\isanewline
\ \ \ \ \isacommand{show}\isamarkupfalse%
\ {\isachardoublequoteopen}z\ {\isasymin}\ X\ {\isasymcdots}\ Y{\isachardoublequoteclose}\isanewline
\ \ \ \ \isacommand{proof}\isamarkupfalse%
{\isacharparenleft}{\kern0pt}cases\ {\isachardoublequoteopen}y\ {\isasymin}\ Y{\isachardoublequoteclose}{\isacharparenright}{\kern0pt}\isanewline
\ \ \ \ \ \ \isacommand{case}\isamarkupfalse%
\ True\isanewline
\ \ \ \ \ \ \isacommand{then}\isamarkupfalse%
\ \isacommand{show}\isamarkupfalse%
\ {\isacharquery}{\kern0pt}thesis\ \isacommand{using}\isamarkupfalse%
\ hz\ hx\ smulI\ hXG\ hYG\ \isacommand{by}\isamarkupfalse%
\ auto\isanewline
\ \ \ \ \isacommand{next}\isamarkupfalse%
\isanewline
\ \ \ \ \ \ \isacommand{case}\isamarkupfalse%
\ False\isanewline
\ \ \ \ \ \ \isacommand{then}\isamarkupfalse%
\ \isacommand{obtain}\isamarkupfalse%
\ w\ \isakeyword{where}\ {\isachardoublequoteopen}y\ {\isacharequal}{\kern0pt}\ inverse\ g\ {\isasymcdot}\ \ w{\isachardoublequoteclose}\ \isakeyword{and}\ {\isachardoublequoteopen}w\ {\isasymin}\ Y{\isachardoublequoteclose}\ \isacommand{using}\isamarkupfalse%
\ hy\ smul{\isachardot}{\kern0pt}cases\ \isacommand{by}\isamarkupfalse%
\ {\isacharparenleft}{\kern0pt}metis\ UnE\ singletonD{\isacharparenright}{\kern0pt}\isanewline
\ \ \ \ \ \ \isacommand{moreover}\isamarkupfalse%
\ \isacommand{obtain}\isamarkupfalse%
\ v\ \isakeyword{where}\ {\isachardoublequoteopen}x\ {\isacharequal}{\kern0pt}\ v\ {\isasymcdot}\ g{\isachardoublequoteclose}\ \isakeyword{and}\ {\isachardoublequoteopen}v\ {\isasymin}\ X{\isachardoublequoteclose}\ \isacommand{using}\isamarkupfalse%
\ hx\ smul{\isachardot}{\kern0pt}cases\ \isacommand{by}\isamarkupfalse%
\ blast\isanewline
\ \ \ \ \ \ \isacommand{ultimately}\isamarkupfalse%
\ \isacommand{show}\isamarkupfalse%
\ {\isacharquery}{\kern0pt}thesis\ \isacommand{using}\isamarkupfalse%
\ hz\ hXG\ hYG\ hgG\ associative\ invertible{\isacharunderscore}{\kern0pt}right{\isacharunderscore}{\kern0pt}inverse{\isadigit{2}}\isanewline
\ \ \ \ \ \ \ \ \isacommand{by}\isamarkupfalse%
\ {\isacharparenleft}{\kern0pt}simp\ add{\isacharcolon}{\kern0pt}\ smul{\isachardot}{\kern0pt}smulI\ subsetD{\isacharparenright}{\kern0pt}\isanewline
\ \ \ \ \isacommand{qed}\isamarkupfalse%
\isanewline
\ \ \isacommand{qed}\isamarkupfalse%
\isanewline
\ \ \isacommand{have}\isamarkupfalse%
\ hXY{\isadigit{2}}{\isacharcolon}{\kern0pt}\ {\isachardoublequoteopen}{\isacharquery}{\kern0pt}X{\isadigit{2}}\ {\isasymcdots}\ {\isacharquery}{\kern0pt}Y{\isadigit{2}}\ {\isasymsubseteq}\ X\ {\isasymcdots}\ Y{\isachardoublequoteclose}\isanewline
\ \ \isacommand{proof}\isamarkupfalse%
\isanewline
\ \ \ \ \isacommand{fix}\isamarkupfalse%
\ z\ \isacommand{assume}\isamarkupfalse%
\ {\isachardoublequoteopen}z\ {\isasymin}\ {\isacharquery}{\kern0pt}X{\isadigit{2}}\ {\isasymcdots}\ {\isacharquery}{\kern0pt}Y{\isadigit{2}}{\isachardoublequoteclose}\isanewline
\ \ \ \ \isacommand{then}\isamarkupfalse%
\ \isacommand{obtain}\isamarkupfalse%
\ x\ y\ \isakeyword{where}\ hz{\isacharcolon}{\kern0pt}\ {\isachardoublequoteopen}z\ {\isacharequal}{\kern0pt}\ x\ {\isasymcdot}\ y{\isachardoublequoteclose}\ \isakeyword{and}\ hx{\isacharcolon}{\kern0pt}\ {\isachardoublequoteopen}x\ {\isasymin}\ {\isacharquery}{\kern0pt}X{\isadigit{2}}{\isachardoublequoteclose}\ \isakeyword{and}\ hy{\isacharcolon}{\kern0pt}\ {\isachardoublequoteopen}y\ {\isasymin}\ {\isacharquery}{\kern0pt}Y{\isadigit{2}}{\isachardoublequoteclose}\ \isacommand{by}\isamarkupfalse%
\ {\isacharparenleft}{\kern0pt}meson\ smul{\isachardot}{\kern0pt}cases{\isacharparenright}{\kern0pt}\isanewline
\ \ \ \ \isacommand{show}\isamarkupfalse%
\ {\isachardoublequoteopen}z\ {\isasymin}\ X\ {\isasymcdots}\ Y{\isachardoublequoteclose}\isanewline
\ \ \ \ \isacommand{proof}\isamarkupfalse%
{\isacharparenleft}{\kern0pt}cases\ {\isachardoublequoteopen}x\ {\isasymin}\ X{\isachardoublequoteclose}{\isacharparenright}{\kern0pt}\isanewline
\ \ \ \ \ \ \isacommand{case}\isamarkupfalse%
\ True\isanewline
\ \ \ \ \ \ \isacommand{then}\isamarkupfalse%
\ \isacommand{show}\isamarkupfalse%
\ {\isacharquery}{\kern0pt}thesis\ \isacommand{using}\isamarkupfalse%
\ hz\ hy\ smulI\ hXG\ hYG\ \isacommand{by}\isamarkupfalse%
\ auto\isanewline
\ \ \ \ \isacommand{next}\isamarkupfalse%
\isanewline
\ \ \ \ \ \ \isacommand{case}\isamarkupfalse%
\ False\isanewline
\ \ \ \ \ \ \isacommand{then}\isamarkupfalse%
\ \isacommand{obtain}\isamarkupfalse%
\ v\ \isakeyword{where}\ {\isachardoublequoteopen}x\ {\isacharequal}{\kern0pt}\ v\ {\isasymcdot}\ g{\isachardoublequoteclose}\ \isakeyword{and}\ {\isachardoublequoteopen}v\ {\isasymin}\ X{\isachardoublequoteclose}\ \isacommand{using}\isamarkupfalse%
\ hx\ smul{\isachardot}{\kern0pt}cases\ \isacommand{by}\isamarkupfalse%
\ {\isacharparenleft}{\kern0pt}metis\ UnE\ singletonD{\isacharparenright}{\kern0pt}\isanewline
\ \ \ \ \ \ \isacommand{moreover}\isamarkupfalse%
\ \isacommand{obtain}\isamarkupfalse%
\ w\ \isakeyword{where}\ {\isachardoublequoteopen}y\ {\isacharequal}{\kern0pt}\ inverse\ g\ {\isasymcdot}\ w{\isachardoublequoteclose}\ \isakeyword{and}\ {\isachardoublequoteopen}w\ {\isasymin}\ Y{\isachardoublequoteclose}\ \isacommand{using}\isamarkupfalse%
\ hy\ smul{\isachardot}{\kern0pt}cases\ \isacommand{by}\isamarkupfalse%
\ blast\isanewline
\ \ \ \ \ \ \isacommand{ultimately}\isamarkupfalse%
\ \isacommand{show}\isamarkupfalse%
\ {\isacharquery}{\kern0pt}thesis\ \isacommand{using}\isamarkupfalse%
\ hz\ hXG\ hYG\ hgG\ associative\ invertible{\isacharunderscore}{\kern0pt}right{\isacharunderscore}{\kern0pt}inverse{\isadigit{2}}\isanewline
\ \ \ \ \ \ \ \ \isacommand{by}\isamarkupfalse%
\ {\isacharparenleft}{\kern0pt}simp\ add{\isacharcolon}{\kern0pt}\ smul{\isachardot}{\kern0pt}smulI\ subsetD{\isacharparenright}{\kern0pt}\isanewline
\ \ \ \ \isacommand{qed}\isamarkupfalse%
\isanewline
\ \ \isacommand{qed}\isamarkupfalse%
\isanewline
\ \ \isacommand{have}\isamarkupfalse%
\ hY{\isadigit{2}}ne{\isacharcolon}{\kern0pt}\ {\isachardoublequoteopen}{\isacharquery}{\kern0pt}Y{\isadigit{2}}\ {\isasymnoteq}\ {\isacharbraceleft}{\kern0pt}{\isacharbraceright}{\kern0pt}{\isachardoublequoteclose}\isanewline
\ \ \isacommand{proof}\isamarkupfalse%
\isanewline
\ \ \ \ \isacommand{assume}\isamarkupfalse%
\ hY{\isadigit{2}}{\isacharcolon}{\kern0pt}\ {\isachardoublequoteopen}{\isacharquery}{\kern0pt}Y{\isadigit{2}}\ {\isacharequal}{\kern0pt}\ {\isacharbraceleft}{\kern0pt}{\isacharbraceright}{\kern0pt}{\isachardoublequoteclose}\isanewline
\ \ \ \ \isacommand{have}\isamarkupfalse%
\ {\isachardoublequoteopen}card\ X\ {\isacharplus}{\kern0pt}\ card\ Y\ {\isasymle}\ card\ Y\ {\isacharplus}{\kern0pt}\ card\ Y{\isachardoublequoteclose}\ \isacommand{by}\isamarkupfalse%
\ {\isacharparenleft}{\kern0pt}simp\ add{\isacharcolon}{\kern0pt}\ hXY{\isacharparenright}{\kern0pt}\isanewline
\ \ \ \ \isacommand{also}\isamarkupfalse%
\ \isacommand{have}\isamarkupfalse%
\ {\isachardoublequoteopen}{\isachardot}{\kern0pt}{\isachardot}{\kern0pt}{\isachardot}{\kern0pt}\ {\isacharequal}{\kern0pt}\ card\ {\isacharquery}{\kern0pt}Y{\isadigit{1}}{\isachardoublequoteclose}\ \isacommand{using}\isamarkupfalse%
\ card{\isacharunderscore}{\kern0pt}Un{\isacharunderscore}{\kern0pt}disjoint\ hYfin\ hYG\ hgG\ finite{\isacharunderscore}{\kern0pt}smul\ inf{\isachardot}{\kern0pt}orderE\ invertible\isanewline
\ \ \ \ \ \ \isacommand{by}\isamarkupfalse%
\ {\isacharparenleft}{\kern0pt}metis\ hY{\isadigit{2}}\ card{\isacharunderscore}{\kern0pt}smul{\isacharunderscore}{\kern0pt}singleton{\isacharunderscore}{\kern0pt}left{\isacharunderscore}{\kern0pt}eq\ finite{\isachardot}{\kern0pt}emptyI\ finite{\isachardot}{\kern0pt}insertI\ invertible{\isacharunderscore}{\kern0pt}inverse{\isacharunderscore}{\kern0pt}closed{\isacharparenright}{\kern0pt}\isanewline
\ \ \ \ \isacommand{also}\isamarkupfalse%
\ \isacommand{have}\isamarkupfalse%
\ {\isachardoublequoteopen}{\isachardot}{\kern0pt}{\isachardot}{\kern0pt}{\isachardot}{\kern0pt}\ {\isasymle}\ card\ {\isacharparenleft}{\kern0pt}{\isacharquery}{\kern0pt}X{\isadigit{1}}\ {\isasymcdots}\ {\isacharquery}{\kern0pt}Y{\isadigit{1}}{\isacharparenright}{\kern0pt}{\isachardoublequoteclose}\ \isacommand{using}\isamarkupfalse%
\ card{\isacharunderscore}{\kern0pt}le{\isacharunderscore}{\kern0pt}smul{\isacharunderscore}{\kern0pt}right{\isacharbrackleft}{\kern0pt}OF\ {\isacharunderscore}{\kern0pt}\ {\isacharunderscore}{\kern0pt}\ {\isacharunderscore}{\kern0pt}\ hY{\isadigit{1}}fin\ hY{\isadigit{1}}G{\isacharbrackright}{\kern0pt}\ \isanewline
\ \ \ \ \ \ \ \ hXgne\ hXG\ Int{\isacharunderscore}{\kern0pt}assoc\ Int{\isacharunderscore}{\kern0pt}commute\ ex{\isacharunderscore}{\kern0pt}in{\isacharunderscore}{\kern0pt}conv\ finite{\isacharunderscore}{\kern0pt}Int\ hXfin\ smul{\isachardot}{\kern0pt}simps\ smul{\isacharunderscore}{\kern0pt}D{\isacharparenleft}{\kern0pt}{\isadigit{2}}{\isacharparenright}{\kern0pt}\ \isanewline
\ \ \ \ \ \ \ \ smul{\isacharunderscore}{\kern0pt}Int{\isacharunderscore}{\kern0pt}carrier\ unit{\isacharunderscore}{\kern0pt}closed\ \isacommand{by}\isamarkupfalse%
\ auto\isanewline
\ \ \ \ \isacommand{also}\isamarkupfalse%
\ \isacommand{have}\isamarkupfalse%
\ {\isachardoublequoteopen}{\isachardot}{\kern0pt}{\isachardot}{\kern0pt}{\isachardot}{\kern0pt}\ {\isasymle}\ card\ {\isacharparenleft}{\kern0pt}X\ {\isasymcdots}\ Y{\isacharparenright}{\kern0pt}{\isachardoublequoteclose}\ \isacommand{using}\isamarkupfalse%
\ hXY{\isadigit{1}}\ finite{\isacharunderscore}{\kern0pt}smul\ card{\isacharunderscore}{\kern0pt}mono\ \isacommand{by}\isamarkupfalse%
\ {\isacharparenleft}{\kern0pt}metis\ hXfin\ hYfin{\isacharparenright}{\kern0pt}\isanewline
\ \ \ \ \isacommand{finally}\isamarkupfalse%
\ \isacommand{show}\isamarkupfalse%
\ False\ \isacommand{using}\isamarkupfalse%
\ hXYlt\ \isacommand{by}\isamarkupfalse%
\ linarith\isanewline
\ \ \isacommand{qed}\isamarkupfalse%
\isanewline
\ \ \isacommand{have}\isamarkupfalse%
\ hXadd{\isacharcolon}{\kern0pt}\ {\isachardoublequoteopen}card\ {\isacharquery}{\kern0pt}X{\isadigit{1}}\ {\isacharplus}{\kern0pt}\ card\ {\isacharquery}{\kern0pt}X{\isadigit{2}}\ {\isacharequal}{\kern0pt}\ {\isadigit{2}}\ {\isacharasterisk}{\kern0pt}\ card\ X{\isachardoublequoteclose}\ \isanewline
\ \ \ \ \isacommand{using}\isamarkupfalse%
\ card{\isacharunderscore}{\kern0pt}smul{\isacharunderscore}{\kern0pt}singleton{\isacharunderscore}{\kern0pt}right{\isacharunderscore}{\kern0pt}eq\ hgG\ hXfin\ hXG\ card{\isacharunderscore}{\kern0pt}Un{\isacharunderscore}{\kern0pt}Int\isanewline
\ \ \ \ \isacommand{by}\isamarkupfalse%
\ {\isacharparenleft}{\kern0pt}metis\ Un{\isacharunderscore}{\kern0pt}Int{\isacharunderscore}{\kern0pt}eq{\isacharparenleft}{\kern0pt}{\isadigit{3}}{\isacharparenright}{\kern0pt}\ add{\isachardot}{\kern0pt}commute\ finite{\isachardot}{\kern0pt}emptyI\ finite{\isachardot}{\kern0pt}insertI\ finite{\isacharunderscore}{\kern0pt}smul\ mult{\isacharunderscore}{\kern0pt}{\isadigit{2}}\ subset{\isacharunderscore}{\kern0pt}Un{\isacharunderscore}{\kern0pt}eq{\isacharparenright}{\kern0pt}\isanewline
\ \ \isacommand{have}\isamarkupfalse%
\ hYadd{\isacharcolon}{\kern0pt}\ {\isachardoublequoteopen}card\ {\isacharquery}{\kern0pt}Y{\isadigit{1}}\ {\isacharplus}{\kern0pt}\ card\ {\isacharquery}{\kern0pt}Y{\isadigit{2}}\ {\isacharequal}{\kern0pt}\ {\isadigit{2}}\ {\isacharasterisk}{\kern0pt}\ card\ Y{\isachardoublequoteclose}\isanewline
\ \ \ \ \isacommand{using}\isamarkupfalse%
\ card{\isacharunderscore}{\kern0pt}smul{\isacharunderscore}{\kern0pt}singleton{\isacharunderscore}{\kern0pt}left{\isacharunderscore}{\kern0pt}eq\ hgG\ hYfin\ hYG\ card{\isacharunderscore}{\kern0pt}Un{\isacharunderscore}{\kern0pt}Int\ finite{\isacharunderscore}{\kern0pt}smul\isanewline
\ \ \ \ \isacommand{by}\isamarkupfalse%
\ {\isacharparenleft}{\kern0pt}metis\ Int{\isacharunderscore}{\kern0pt}lower{\isadigit{1}}\ Un{\isacharunderscore}{\kern0pt}Int{\isacharunderscore}{\kern0pt}eq{\isacharparenleft}{\kern0pt}{\isadigit{3}}{\isacharparenright}{\kern0pt}\ card{\isacharunderscore}{\kern0pt}{\isadigit{0}}{\isacharunderscore}{\kern0pt}eq\ card{\isacharunderscore}{\kern0pt}Un{\isacharunderscore}{\kern0pt}le\ card{\isacharunderscore}{\kern0pt}seteq\ finite{\isachardot}{\kern0pt}emptyI\ finite{\isachardot}{\kern0pt}insertI\ \ \isanewline
\ \ \ \ \ \ hY{\isadigit{2}}ne\ le{\isacharunderscore}{\kern0pt}add{\isacharunderscore}{\kern0pt}same{\isacharunderscore}{\kern0pt}cancel{\isadigit{1}}\ mult{\isacharunderscore}{\kern0pt}{\isadigit{2}}\ subset{\isacharunderscore}{\kern0pt}Un{\isacharunderscore}{\kern0pt}eq{\isacharparenright}{\kern0pt}\isanewline
\ \ \isacommand{show}\isamarkupfalse%
\ False\isanewline
\ \ \isacommand{proof}\isamarkupfalse%
\ {\isacharparenleft}{\kern0pt}cases\ {\isachardoublequoteopen}card\ {\isacharquery}{\kern0pt}X{\isadigit{2}}\ {\isacharplus}{\kern0pt}\ card\ {\isacharquery}{\kern0pt}Y{\isadigit{2}}\ {\isachargreater}{\kern0pt}\ card\ X\ {\isacharplus}{\kern0pt}\ card\ Y{\isachardoublequoteclose}{\isacharparenright}{\kern0pt}\isanewline
\ \ \ \ \isacommand{case}\isamarkupfalse%
\ h{\isacharcolon}{\kern0pt}\ True\isanewline
\ \ \ \ \isacommand{have}\isamarkupfalse%
\ hXY{\isadigit{2}}le{\isacharcolon}{\kern0pt}\ {\isachardoublequoteopen}card\ {\isacharparenleft}{\kern0pt}{\isacharquery}{\kern0pt}X{\isadigit{2}}\ {\isasymcdots}\ {\isacharquery}{\kern0pt}Y{\isadigit{2}}{\isacharparenright}{\kern0pt}\ {\isasymle}\ card\ {\isacharparenleft}{\kern0pt}X\ {\isasymcdots}\ Y{\isacharparenright}{\kern0pt}{\isachardoublequoteclose}\ \isacommand{using}\isamarkupfalse%
\ hXY{\isadigit{2}}\ finite{\isacharunderscore}{\kern0pt}smul\ card{\isacharunderscore}{\kern0pt}mono\ \isacommand{by}\isamarkupfalse%
\ {\isacharparenleft}{\kern0pt}metis\ hXfin\ hYfin{\isacharparenright}{\kern0pt}\isanewline
\ \ \ \ \isacommand{also}\isamarkupfalse%
\ \isacommand{have}\isamarkupfalse%
\ {\isachardoublequoteopen}{\isachardot}{\kern0pt}{\isachardot}{\kern0pt}{\isachardot}{\kern0pt}\ {\isacharless}{\kern0pt}\ min\ p\ {\isacharparenleft}{\kern0pt}card\ X\ {\isacharplus}{\kern0pt}\ card\ Y\ {\isacharminus}{\kern0pt}\ {\isadigit{1}}{\isacharparenright}{\kern0pt}{\isachardoublequoteclose}\ \isacommand{using}\isamarkupfalse%
\ hXYlt\ \isacommand{by}\isamarkupfalse%
\ auto\isanewline
\ \ \ \ \isacommand{also}\isamarkupfalse%
\ \isacommand{have}\isamarkupfalse%
\ {\isachardoublequoteopen}{\isachardot}{\kern0pt}{\isachardot}{\kern0pt}{\isachardot}{\kern0pt}\ {\isasymle}\ min\ p\ {\isacharparenleft}{\kern0pt}card\ {\isacharquery}{\kern0pt}X{\isadigit{2}}\ {\isacharplus}{\kern0pt}\ card\ {\isacharquery}{\kern0pt}Y{\isadigit{2}}\ {\isacharminus}{\kern0pt}\ {\isadigit{1}}{\isacharparenright}{\kern0pt}{\isachardoublequoteclose}\ \isacommand{using}\isamarkupfalse%
\ h\ \isacommand{by}\isamarkupfalse%
\ simp\isanewline
\ \ \ \ \isacommand{finally}\isamarkupfalse%
\ \isacommand{have}\isamarkupfalse%
\ hXY{\isadigit{1}}M{\isacharcolon}{\kern0pt}\ {\isachardoublequoteopen}{\isacharparenleft}{\kern0pt}{\isacharquery}{\kern0pt}X{\isadigit{2}}{\isacharcomma}{\kern0pt}\ {\isacharquery}{\kern0pt}Y{\isadigit{2}}{\isacharparenright}{\kern0pt}\ {\isasymin}\ M{\isachardoublequoteclose}\ \isacommand{using}\isamarkupfalse%
\ M{\isacharunderscore}{\kern0pt}def\ hY{\isadigit{2}}ne\ hX{\isadigit{2}}fin\ hX{\isadigit{2}}G\ hXYM\ \isacommand{by}\isamarkupfalse%
\ auto\isanewline
\ \ \ \ \isacommand{moreover}\isamarkupfalse%
\ \isacommand{have}\isamarkupfalse%
\ {\isachardoublequoteopen}{\isacharparenleft}{\kern0pt}{\isacharparenleft}{\kern0pt}{\isacharquery}{\kern0pt}X{\isadigit{2}}{\isacharcomma}{\kern0pt}\ {\isacharquery}{\kern0pt}Y{\isadigit{2}}{\isacharparenright}{\kern0pt}{\isacharcomma}{\kern0pt}\ {\isacharparenleft}{\kern0pt}X{\isacharcomma}{\kern0pt}\ Y{\isacharparenright}{\kern0pt}{\isacharparenright}{\kern0pt}\ {\isasymin}\ \ Restr\ devos{\isacharunderscore}{\kern0pt}rel\ {\isacharquery}{\kern0pt}fin{\isachardoublequoteclose}\ \isacommand{using}\isamarkupfalse%
\ hXYM\ M{\isacharunderscore}{\kern0pt}def\ hXY{\isadigit{1}}M\ h\ hXY{\isadigit{2}}le\ \isanewline
\ \ \ \ \ \ \ \ devos{\isacharunderscore}{\kern0pt}rel{\isacharunderscore}{\kern0pt}iff\ \isacommand{by}\isamarkupfalse%
\ auto\isanewline
\ \ \ \ \isacommand{ultimately}\isamarkupfalse%
\ \isacommand{show}\isamarkupfalse%
\ False\ \isacommand{using}\isamarkupfalse%
\ hmin\ \isacommand{by}\isamarkupfalse%
\ blast\ \isanewline
\ \ \isacommand{next}\isamarkupfalse%
\isanewline
\ \ \ \ \isacommand{case}\isamarkupfalse%
\ False\isanewline
\ \ \ \ \isacommand{then}\isamarkupfalse%
\ \isacommand{have}\isamarkupfalse%
\ h{\isacharcolon}{\kern0pt}\ {\isachardoublequoteopen}card\ {\isacharquery}{\kern0pt}X{\isadigit{1}}\ {\isacharplus}{\kern0pt}\ card\ {\isacharquery}{\kern0pt}Y{\isadigit{1}}\ {\isasymge}\ card\ X\ {\isacharplus}{\kern0pt}\ card\ Y{\isachardoublequoteclose}\ \isacommand{using}\isamarkupfalse%
\ hXadd\ hYadd\ \isacommand{by}\isamarkupfalse%
\ linarith\isanewline
\ \ \ \ \isacommand{have}\isamarkupfalse%
\ hX{\isadigit{1}}lt{\isacharcolon}{\kern0pt}\ {\isachardoublequoteopen}card\ {\isacharquery}{\kern0pt}X{\isadigit{1}}\ {\isacharless}{\kern0pt}\ card\ X{\isachardoublequoteclose}\ \isacommand{using}\isamarkupfalse%
\ hXfin\ \isacommand{by}\isamarkupfalse%
\ {\isacharparenleft}{\kern0pt}simp\ add{\isacharcolon}{\kern0pt}\ hpsubX\ psubset{\isacharunderscore}{\kern0pt}card{\isacharunderscore}{\kern0pt}mono{\isacharparenright}{\kern0pt}\isanewline
\ \ \ \ \isacommand{have}\isamarkupfalse%
\ hXY{\isadigit{1}}le{\isacharcolon}{\kern0pt}\ {\isachardoublequoteopen}card\ {\isacharparenleft}{\kern0pt}{\isacharquery}{\kern0pt}X{\isadigit{1}}\ {\isasymcdots}\ {\isacharquery}{\kern0pt}Y{\isadigit{1}}{\isacharparenright}{\kern0pt}\ {\isasymle}\ card\ {\isacharparenleft}{\kern0pt}X\ {\isasymcdots}\ Y{\isacharparenright}{\kern0pt}{\isachardoublequoteclose}\ \isacommand{using}\isamarkupfalse%
\ hXY{\isadigit{1}}\ finite{\isacharunderscore}{\kern0pt}smul\ card{\isacharunderscore}{\kern0pt}mono\ hYfin\ hXfin\ \isacommand{by}\isamarkupfalse%
\ metis\isanewline
\ \ \ \ \isacommand{also}\isamarkupfalse%
\ \isacommand{have}\isamarkupfalse%
\ {\isachardoublequoteopen}{\isachardot}{\kern0pt}{\isachardot}{\kern0pt}{\isachardot}{\kern0pt}\ {\isacharless}{\kern0pt}\ min\ p\ {\isacharparenleft}{\kern0pt}card\ X\ {\isacharplus}{\kern0pt}\ card\ Y\ {\isacharminus}{\kern0pt}\ {\isadigit{1}}{\isacharparenright}{\kern0pt}{\isachardoublequoteclose}\ \isacommand{using}\isamarkupfalse%
\ hXYlt\ \isacommand{by}\isamarkupfalse%
\ auto\isanewline
\ \ \ \ \isacommand{also}\isamarkupfalse%
\ \isacommand{have}\isamarkupfalse%
\ {\isachardoublequoteopen}{\isachardot}{\kern0pt}{\isachardot}{\kern0pt}{\isachardot}{\kern0pt}\ {\isasymle}\ min\ p\ {\isacharparenleft}{\kern0pt}card\ {\isacharquery}{\kern0pt}X{\isadigit{1}}\ {\isacharplus}{\kern0pt}\ card\ {\isacharquery}{\kern0pt}Y{\isadigit{1}}\ {\isacharminus}{\kern0pt}\ {\isadigit{1}}{\isacharparenright}{\kern0pt}{\isachardoublequoteclose}\ \isacommand{using}\isamarkupfalse%
\ h\ \isacommand{by}\isamarkupfalse%
\ simp\isanewline
\ \ \ \ \isacommand{finally}\isamarkupfalse%
\ \isacommand{have}\isamarkupfalse%
\ hXY{\isadigit{1}}M{\isacharcolon}{\kern0pt}\ {\isachardoublequoteopen}{\isacharparenleft}{\kern0pt}{\isacharquery}{\kern0pt}X{\isadigit{1}}{\isacharcomma}{\kern0pt}\ {\isacharquery}{\kern0pt}Y{\isadigit{1}}{\isacharparenright}{\kern0pt}\ {\isasymin}\ M{\isachardoublequoteclose}\ \isacommand{using}\isamarkupfalse%
\ M{\isacharunderscore}{\kern0pt}def\ hXgne\ hY{\isadigit{1}}fin\ hY{\isadigit{1}}G\ hXYM\ \isacommand{by}\isamarkupfalse%
\ auto\isanewline
\ \ \ \ \isacommand{moreover}\isamarkupfalse%
\ \isacommand{have}\isamarkupfalse%
\ {\isachardoublequoteopen}{\isacharparenleft}{\kern0pt}{\isacharparenleft}{\kern0pt}{\isacharquery}{\kern0pt}X{\isadigit{1}}{\isacharcomma}{\kern0pt}\ {\isacharquery}{\kern0pt}Y{\isadigit{1}}{\isacharparenright}{\kern0pt}{\isacharcomma}{\kern0pt}\ {\isacharparenleft}{\kern0pt}X{\isacharcomma}{\kern0pt}\ Y{\isacharparenright}{\kern0pt}{\isacharparenright}{\kern0pt}\ {\isasymin}\ \ Restr\ devos{\isacharunderscore}{\kern0pt}rel\ {\isacharquery}{\kern0pt}fin{\isachardoublequoteclose}\ \isacommand{using}\isamarkupfalse%
\ hXYM\ M{\isacharunderscore}{\kern0pt}def\ hXY{\isadigit{1}}M\ h\ hXY{\isadigit{1}}le\ \isanewline
\ \ \ \ \ \ \ \ devos{\isacharunderscore}{\kern0pt}rel{\isacharunderscore}{\kern0pt}iff\ hX{\isadigit{1}}lt\ hXY{\isadigit{1}}le\ h\ \isacommand{by}\isamarkupfalse%
\ force\isanewline
\ \ \ \ \isacommand{ultimately}\isamarkupfalse%
\ \isacommand{show}\isamarkupfalse%
\ {\isacharquery}{\kern0pt}thesis\ \isacommand{using}\isamarkupfalse%
\ hmin\ \isacommand{by}\isamarkupfalse%
\ blast\isanewline
\ \ \isacommand{qed}\isamarkupfalse%
\isanewline
\isacommand{qed}\isamarkupfalse%
%
\endisatagproof
{\isafoldproof}%
%
\isadelimproof
\isanewline
%
\endisadelimproof
\isanewline
%
\isadelimtheory
\isanewline
%
\endisadelimtheory
%
\isatagtheory
\isacommand{end}\isamarkupfalse%
%
\endisatagtheory
{\isafoldtheory}%
%
\isadelimtheory
%
\endisadelimtheory
%
\end{isabellebody}%
\endinput
%:%file=~/IsabelleProjects/Generalized_Cauchy_Davenport/Generalized_Cauchy_Davenport_main_proof.thy%:%
%:%11=1%:%
%:%27=14%:%
%:%28=14%:%
%:%29=15%:%
%:%30=16%:%
%:%31=17%:%
%:%36=17%:%
%:%39=18%:%
%:%40=19%:%
%:%41=19%:%
%:%42=20%:%
%:%43=21%:%
%:%50=23%:%
%:%60=24%:%
%:%61=24%:%
%:%62=25%:%
%:%70=28%:%
%:%80=29%:%
%:%81=29%:%
%:%82=30%:%
%:%84=32%:%
%:%87=33%:%
%:%91=33%:%
%:%92=33%:%
%:%93=33%:%
%:%98=33%:%
%:%101=34%:%
%:%102=35%:%
%:%103=35%:%
%:%104=36%:%
%:%107=37%:%
%:%111=37%:%
%:%112=37%:%
%:%113=37%:%
%:%122=39%:%
%:%124=40%:%
%:%125=40%:%
%:%126=41%:%
%:%133=42%:%
%:%134=42%:%
%:%135=43%:%
%:%136=43%:%
%:%137=44%:%
%:%138=44%:%
%:%139=45%:%
%:%140=45%:%
%:%141=46%:%
%:%142=46%:%
%:%143=47%:%
%:%144=47%:%
%:%145=48%:%
%:%146=48%:%
%:%147=49%:%
%:%148=49%:%
%:%149=49%:%
%:%150=50%:%
%:%151=50%:%
%:%152=50%:%
%:%153=50%:%
%:%154=51%:%
%:%155=51%:%
%:%156=51%:%
%:%157=52%:%
%:%158=52%:%
%:%159=52%:%
%:%160=53%:%
%:%161=53%:%
%:%162=53%:%
%:%163=53%:%
%:%164=53%:%
%:%165=54%:%
%:%166=54%:%
%:%167=55%:%
%:%168=55%:%
%:%169=55%:%
%:%170=56%:%
%:%171=56%:%
%:%172=56%:%
%:%173=57%:%
%:%174=57%:%
%:%175=58%:%
%:%176=58%:%
%:%177=58%:%
%:%178=59%:%
%:%179=59%:%
%:%180=59%:%
%:%181=60%:%
%:%182=60%:%
%:%183=61%:%
%:%184=61%:%
%:%186=63%:%
%:%187=64%:%
%:%188=64%:%
%:%189=65%:%
%:%190=65%:%
%:%191=65%:%
%:%192=66%:%
%:%193=66%:%
%:%194=66%:%
%:%195=67%:%
%:%196=68%:%
%:%197=69%:%
%:%198=69%:%
%:%199=69%:%
%:%200=70%:%
%:%201=70%:%
%:%202=70%:%
%:%203=71%:%
%:%204=72%:%
%:%205=72%:%
%:%206=72%:%
%:%207=73%:%
%:%208=73%:%
%:%209=73%:%
%:%210=74%:%
%:%211=74%:%
%:%212=74%:%
%:%213=75%:%
%:%214=75%:%
%:%215=76%:%
%:%216=76%:%
%:%217=76%:%
%:%218=77%:%
%:%219=77%:%
%:%220=78%:%
%:%221=78%:%
%:%222=78%:%
%:%223=78%:%
%:%224=79%:%
%:%225=79%:%
%:%226=80%:%
%:%227=80%:%
%:%228=81%:%
%:%229=81%:%
%:%230=81%:%
%:%231=81%:%
%:%232=81%:%
%:%233=82%:%
%:%234=82%:%
%:%235=82%:%
%:%236=83%:%
%:%237=83%:%
%:%238=84%:%
%:%239=84%:%
%:%240=85%:%
%:%241=85%:%
%:%242=85%:%
%:%243=86%:%
%:%244=86%:%
%:%245=86%:%
%:%246=87%:%
%:%247=87%:%
%:%248=87%:%
%:%249=87%:%
%:%250=88%:%
%:%265=90%:%
%:%275=91%:%
%:%276=91%:%
%:%277=92%:%
%:%278=93%:%
%:%279=93%:%
%:%280=94%:%
%:%281=95%:%
%:%284=96%:%
%:%288=96%:%
%:%289=96%:%
%:%294=96%:%
%:%297=97%:%
%:%298=98%:%
%:%299=98%:%
%:%300=99%:%
%:%301=100%:%
%:%304=101%:%
%:%308=101%:%
%:%309=101%:%
%:%310=101%:%
%:%315=101%:%
%:%318=102%:%
%:%319=103%:%
%:%320=103%:%
%:%321=104%:%
%:%322=105%:%
%:%324=105%:%
%:%328=105%:%
%:%329=105%:%
%:%330=105%:%
%:%337=105%:%
%:%338=106%:%
%:%339=107%:%
%:%347=109%:%
%:%359=111%:%
%:%361=112%:%
%:%362=112%:%
%:%363=113%:%
%:%364=114%:%
%:%365=115%:%
%:%366=116%:%
%:%373=117%:%
%:%374=117%:%
%:%375=118%:%
%:%376=118%:%
%:%377=119%:%
%:%378=119%:%
%:%379=120%:%
%:%380=120%:%
%:%381=121%:%
%:%382=121%:%
%:%383=121%:%
%:%384=121%:%
%:%385=122%:%
%:%386=122%:%
%:%387=122%:%
%:%388=123%:%
%:%389=123%:%
%:%390=123%:%
%:%391=124%:%
%:%392=124%:%
%:%393=125%:%
%:%394=125%:%
%:%395=125%:%
%:%396=126%:%
%:%397=126%:%
%:%398=127%:%
%:%399=127%:%
%:%400=128%:%
%:%401=128%:%
%:%402=129%:%
%:%403=129%:%
%:%404=129%:%
%:%405=129%:%
%:%406=130%:%
%:%407=130%:%
%:%408=131%:%
%:%409=131%:%
%:%410=132%:%
%:%411=132%:%
%:%412=132%:%
%:%413=132%:%
%:%414=133%:%
%:%415=133%:%
%:%416=133%:%
%:%417=134%:%
%:%418=134%:%
%:%419=134%:%
%:%420=135%:%
%:%421=135%:%
%:%422=135%:%
%:%423=136%:%
%:%424=136%:%
%:%425=136%:%
%:%426=137%:%
%:%427=137%:%
%:%428=137%:%
%:%429=138%:%
%:%430=138%:%
%:%431=138%:%
%:%432=139%:%
%:%433=139%:%
%:%434=139%:%
%:%435=139%:%
%:%436=139%:%
%:%437=140%:%
%:%438=140%:%
%:%439=140%:%
%:%440=141%:%
%:%441=141%:%
%:%442=141%:%
%:%443=142%:%
%:%444=142%:%
%:%445=142%:%
%:%446=142%:%
%:%447=142%:%
%:%448=143%:%
%:%449=143%:%
%:%450=143%:%
%:%451=143%:%
%:%452=143%:%
%:%453=144%:%
%:%454=144%:%
%:%455=144%:%
%:%456=144%:%
%:%457=144%:%
%:%458=145%:%
%:%459=145%:%
%:%460=145%:%
%:%461=145%:%
%:%462=145%:%
%:%463=146%:%
%:%464=146%:%
%:%465=147%:%
%:%466=147%:%
%:%467=148%:%
%:%468=148%:%
%:%469=149%:%
%:%470=149%:%
%:%471=150%:%
%:%472=150%:%
%:%473=150%:%
%:%474=150%:%
%:%475=150%:%
%:%476=151%:%
%:%477=151%:%
%:%478=151%:%
%:%479=151%:%
%:%480=152%:%
%:%481=152%:%
%:%482=152%:%
%:%483=152%:%
%:%484=152%:%
%:%485=153%:%
%:%486=153%:%
%:%487=153%:%
%:%488=153%:%
%:%489=154%:%
%:%490=154%:%
%:%491=155%:%
%:%492=155%:%
%:%493=155%:%
%:%494=155%:%
%:%495=155%:%
%:%496=156%:%
%:%497=156%:%
%:%498=157%:%
%:%499=157%:%
%:%500=157%:%
%:%501=157%:%
%:%502=158%:%
%:%503=158%:%
%:%504=158%:%
%:%505=158%:%
%:%506=159%:%
%:%507=160%:%
%:%508=160%:%
%:%509=160%:%
%:%510=161%:%
%:%511=161%:%
%:%512=161%:%
%:%513=162%:%
%:%514=163%:%
%:%515=163%:%
%:%516=164%:%
%:%517=164%:%
%:%518=165%:%
%:%519=165%:%
%:%520=166%:%
%:%521=166%:%
%:%522=166%:%
%:%523=166%:%
%:%524=167%:%
%:%525=167%:%
%:%526=167%:%
%:%527=167%:%
%:%528=168%:%
%:%529=168%:%
%:%530=169%:%
%:%531=169%:%
%:%532=169%:%
%:%533=169%:%
%:%534=169%:%
%:%535=170%:%
%:%536=170%:%
%:%537=170%:%
%:%538=170%:%
%:%539=171%:%
%:%540=171%:%
%:%541=172%:%
%:%542=172%:%
%:%543=172%:%
%:%544=172%:%
%:%545=173%:%
%:%546=173%:%
%:%547=173%:%
%:%548=174%:%
%:%549=174%:%
%:%550=175%:%
%:%551=175%:%
%:%552=176%:%
%:%553=176%:%
%:%554=176%:%
%:%555=176%:%
%:%556=176%:%
%:%557=177%:%
%:%558=178%:%
%:%559=178%:%
%:%560=178%:%
%:%561=178%:%
%:%562=178%:%
%:%563=179%:%
%:%564=179%:%
%:%565=180%:%
%:%566=180%:%
%:%567=180%:%
%:%568=180%:%
%:%569=181%:%
%:%570=182%:%
%:%571=182%:%
%:%572=183%:%
%:%573=183%:%
%:%574=183%:%
%:%575=184%:%
%:%576=184%:%
%:%577=185%:%
%:%578=185%:%
%:%579=185%:%
%:%580=185%:%
%:%581=186%:%
%:%582=186%:%
%:%583=187%:%
%:%584=188%:%
%:%585=188%:%
%:%586=188%:%
%:%587=188%:%
%:%588=188%:%
%:%589=189%:%
%:%590=189%:%
%:%591=190%:%
%:%592=190%:%
%:%593=191%:%
%:%594=191%:%
%:%595=192%:%
%:%596=192%:%
%:%597=193%:%
%:%598=193%:%
%:%599=194%:%
%:%600=194%:%
%:%601=195%:%
%:%602=195%:%
%:%603=195%:%
%:%604=196%:%
%:%605=196%:%
%:%606=197%:%
%:%607=197%:%
%:%608=198%:%
%:%609=198%:%
%:%610=198%:%
%:%611=199%:%
%:%612=199%:%
%:%613=199%:%
%:%614=199%:%
%:%615=200%:%
%:%616=200%:%
%:%617=201%:%
%:%618=201%:%
%:%619=202%:%
%:%620=202%:%
%:%621=203%:%
%:%622=203%:%
%:%623=203%:%
%:%624=203%:%
%:%625=203%:%
%:%626=204%:%
%:%627=204%:%
%:%628=205%:%
%:%629=205%:%
%:%630=206%:%
%:%631=206%:%
%:%632=206%:%
%:%633=206%:%
%:%634=206%:%
%:%635=207%:%
%:%636=207%:%
%:%637=207%:%
%:%638=207%:%
%:%639=207%:%
%:%640=208%:%
%:%641=208%:%
%:%642=208%:%
%:%643=208%:%
%:%644=209%:%
%:%645=209%:%
%:%646=210%:%
%:%647=210%:%
%:%648=211%:%
%:%649=211%:%
%:%650=212%:%
%:%651=212%:%
%:%652=213%:%
%:%653=213%:%
%:%654=214%:%
%:%655=214%:%
%:%656=214%:%
%:%657=215%:%
%:%658=215%:%
%:%659=215%:%
%:%660=215%:%
%:%661=216%:%
%:%662=216%:%
%:%663=217%:%
%:%664=217%:%
%:%665=218%:%
%:%666=218%:%
%:%667=219%:%
%:%668=219%:%
%:%669=219%:%
%:%670=219%:%
%:%671=219%:%
%:%672=220%:%
%:%673=220%:%
%:%674=221%:%
%:%675=221%:%
%:%676=222%:%
%:%677=222%:%
%:%678=222%:%
%:%679=222%:%
%:%680=222%:%
%:%681=223%:%
%:%682=223%:%
%:%683=223%:%
%:%684=223%:%
%:%685=223%:%
%:%686=224%:%
%:%687=224%:%
%:%688=224%:%
%:%689=224%:%
%:%690=225%:%
%:%691=225%:%
%:%692=226%:%
%:%693=226%:%
%:%694=227%:%
%:%695=227%:%
%:%696=228%:%
%:%697=228%:%
%:%698=229%:%
%:%699=229%:%
%:%700=230%:%
%:%701=230%:%
%:%702=231%:%
%:%703=231%:%
%:%704=231%:%
%:%705=232%:%
%:%706=232%:%
%:%707=232%:%
%:%708=232%:%
%:%709=233%:%
%:%710=233%:%
%:%711=234%:%
%:%712=234%:%
%:%713=234%:%
%:%714=234%:%
%:%715=235%:%
%:%716=236%:%
%:%717=236%:%
%:%718=237%:%
%:%719=237%:%
%:%720=237%:%
%:%721=237%:%
%:%722=237%:%
%:%723=238%:%
%:%724=238%:%
%:%725=238%:%
%:%726=238%:%
%:%727=238%:%
%:%728=239%:%
%:%729=239%:%
%:%730=240%:%
%:%731=240%:%
%:%732=241%:%
%:%733=241%:%
%:%734=242%:%
%:%735=242%:%
%:%736=243%:%
%:%737=243%:%
%:%738=244%:%
%:%739=244%:%
%:%740=245%:%
%:%741=245%:%
%:%742=246%:%
%:%743=247%:%
%:%744=247%:%
%:%745=248%:%
%:%746=248%:%
%:%747=249%:%
%:%748=249%:%
%:%749=250%:%
%:%750=250%:%
%:%751=250%:%
%:%752=250%:%
%:%753=251%:%
%:%754=251%:%
%:%755=251%:%
%:%756=251%:%
%:%757=251%:%
%:%758=252%:%
%:%759=252%:%
%:%760=252%:%
%:%761=252%:%
%:%762=252%:%
%:%763=253%:%
%:%764=253%:%
%:%765=253%:%
%:%766=253%:%
%:%767=253%:%
%:%768=254%:%
%:%769=254%:%
%:%770=254%:%
%:%771=254%:%
%:%772=255%:%
%:%773=255%:%
%:%774=256%:%
%:%775=256%:%
%:%776=256%:%
%:%777=256%:%
%:%778=256%:%
%:%779=257%:%
%:%780=257%:%
%:%781=258%:%
%:%782=258%:%
%:%783=259%:%
%:%784=259%:%
%:%785=259%:%
%:%786=259%:%
%:%787=259%:%
%:%788=260%:%
%:%789=260%:%
%:%790=260%:%
%:%791=260%:%
%:%792=261%:%
%:%793=261%:%
%:%794=261%:%
%:%795=261%:%
%:%796=262%:%
%:%797=262%:%
%:%798=262%:%
%:%799=262%:%
%:%800=262%:%
%:%801=263%:%
%:%802=263%:%
%:%803=263%:%
%:%804=263%:%
%:%805=263%:%
%:%806=264%:%
%:%807=264%:%
%:%808=264%:%
%:%809=264%:%
%:%810=264%:%
%:%811=265%:%
%:%812=265%:%
%:%813=265%:%
%:%814=265%:%
%:%815=266%:%
%:%816=266%:%
%:%817=267%:%
%:%818=267%:%
%:%819=267%:%
%:%820=267%:%
%:%821=267%:%
%:%822=268%:%
%:%823=268%:%
%:%824=269%:%
%:%830=269%:%
%:%833=270%:%
%:%836=271%:%
%:%841=272%:%




\bibliographystyle{abbrv}
\bibliography{root}


\end{document}
